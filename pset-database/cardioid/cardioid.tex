% Copyright (c) 1990 Massachusetts Institute of Technology
% 
% This material was developed by the Scheme project at the Massachusetts
% Institute of Technology, Department of Electrical Engineering and
% Computer Science.  Permission to copy this material, to redistribute
% it, and to use it for any non-commercial purpose is granted, subject
% to the following restrictions and understandings.
% 
% 1. Any copy made of this material must include this copyright notice
% in full.
% 
% 2. Users of this material agree to make their best efforts (a) to
% return to the MIT Scheme project any improvements or extensions that
% they make, so that these may be included in future releases; and (b)
% to inform MIT of noteworthy uses of this material.
% 
% 3. All materials developed as a consequence of the use of this
% material shall duly acknowledge such use, in accordance with the usual
% standards of acknowledging credit in academic research.
% 
% 4. MIT has made no warrantee or representation that this material
% (including the operation of software contained therein) will be
% error-free, and MIT is under no obligation to provide any services, by
% way of maintenance, update, or otherwise.
% 
% 5. In conjunction with products arising from the use of this material,
% there shall be no use of the name of the Massachusetts Institute of
% Technology nor of any adaptation thereof in any advertising,
% promotional, or sales literature without prior written consent from
% MIT in each case. 

% for yTeX
\typesize=11pt
\hsize=34pc
\vsize=50pc
\parskip 6pt plus 2pt

\def\psetheader{
\centerline{MASSACHUSETTS INSTITUTE OF TECHNOLOGY}
\centerline{Department of Electrical Engineering and Computer Science}
\centerline{6.001 Structure and Interpretation of Computer Programs}
\centerline{Fall Semester, 1987}}

\def\code#1{\beginlisp
#1
\endlisp

\vskip .1in}

\rectoleftheader={6.001 -- Fall Semester 1987}
\rectorightheader={Problem Set 1}
\onheaders
\onfooters

\null
\vskip 1truein

\psetheader

\vskip .25truein

\centerline{Problem Set 1}

\vskip 0.25truein

\vpar
Issued:  September 10, 1987


\vpar
Due: in recitation on September 18, 1987 {\it for all sections}

\vpar
Reading: From text, Chapter 1, section 1.1.

Problem sets should always be handed in at recitation.  Late work will
not be accepted.
%  As explained in the Course Organization handout, we
%are staggering the due-dates for problem sets, so as to
%minimize crowding in the lab.  The Wednesday/Friday due dates will be
%reversed during the second half of the semester to assure equitable
%treatment of students in both morning and afternoon recitations.
Specific due dates will be announced as each problem set is handed
out.

Before you can begin work on this assignment, you will need to buy
a copy of the textbook and to pick up
the manual package.  To obtain the manual package, take the coupon
attached to the ``Red Tape'' handout, and follow the instructions
given in the handout marked ``6.001: Course Organization.''

\chapter{1. Homework exercises}

Write up and turn in the following exercises:

\beginbullets

\bpar Exercise 1.  Evaluation of expressions.

\ftpar{\hskip \parindent}
Below is a sequence of expressions.  What is the result printed by the 
interpreter in response to each expression?  Assume that the sequence is to be
evaluated in the order in which it is presented.  You should be able to do this
without the help of a Chipmunk.

\beginlisp

==> 7
\pbrk
==> (+ 4 2 5)
\pbrk
==> (/ 10 2)
\pbrk
==> (define foo 6)
\pbrk
==> foo
\pbrk
==> (define bar (+ 12 foo))
\pbrk
==> (+ foo (* bar foo))
\pbrk
==> (< bar foo)
\pbrk
==> (if (= (/ bar 3) foo)
        (- bar foo)
        0)
\pbrk
==> (cond ((>= foo 7) bar)
          ((= bar (+ foo 10)) 0)
          (else (abs (- foo bar))))
\endlisp

\bpar Exercise 2.

\ftpar{\hskip \parindent}
In exercise 1.3 in the textbook, Ben Bitdiddle devised a way of
testing whether an interpreter uses applicative or normal order of evaluation.
In fact, when Ben evaluates the expression (test 0 (p)), an applicative 
evaluation would put the interpreter into an infinite loop, whereas a normal
evaluation would let the interpreter return 0.  (If you agree with us, then you
have understood the difference between these 2 kinds of evaluations.)  In this
exercise, we would like you to design another method for such kind of test -- 
namely, write a little procedure which, when run with certain arguments, would
behave differently on applicative and normal order of evaluation.  Your answer
must include the procedure, the expression you typed in to run the procedure,
and a brief explanation of how it works.

\bpar
Exercise 3 -- do exercise 1.2 in the textbook

\bpar
Exercise 4 -- do exercise 1.8 in the textbook

\chapter{2. Programming Assignment: Graphing Cardioids and Nephroids}

A substantial part of each problem set consists of a laboratory assignment that
should be done with the aid of the 6.001 computing facility.  Performance on 
problem sets, and laboratories in particular, is a major factor in determing
grades for 6.001.  Remember that the laboratory room tends to get very crowded
on the night before assignments are due, so it will be to your advantage to
complete your computer work early.

The laboratory assignments have been planned 
on the assumption that the required reading has been done before you
come to the laboratory.   You should also read over and think about the
laboratory assignment  before you sit down in front of a Chipmunk.   It
is generally much more efficient to use a Chipmunk to test, debug, and
run a program you have planned before coming into lab than to try to do
the planning ``online''.  Students who have taken the subject in
previous  terms report that failing to prepare ahead for laboratory
assignments generally ensures the assignments take much longer than
necessary.   

Before coming to the laboratory room for the first time, you should read
the "Chipmunk Manual" and you should bring it with you to use for
reference.  You should also glance  through the "Scheme Manual," but
almost all the information you will need about the Scheme language
itself has been included in the text. 

Each time you begin using one of the Chipmunk computers, you must insert a 
floppy disk in the drive.  Each disk is identified by a name, which is chosen
by you at the time you initialize the disk.  If you place an uninitialized disk
in the drive, the system will offer to initialize it for you.  Before beginning
work on the first assignment, you should use the system to initialize your
disk.  More information is given in the Chipmunk Manual, in the section called
"starting and leaving the system."  Additional help is available from the
laboratory teaching assistants at popular hours.

The first laboratory assignment has been designed to be very simple, so that
you will have a chance to become familiar with the Chipmunk system.


\vskip .5in

In this assignment, you are to experiment with simple procedures that plot
curves called cardioids and nephroids.  These are defined formally as follows.

Consider a circle whose center is at the origin of some coordinate system, and
whose radius is $r$ (call this the base-circle.)  We select a fixed
point $A$ at coordinate $(-r, 0)$.  With center at any point $Q$ on the
circle, and radius $QA$, we draw another circle.  
We repeat this for a large number of positions of $Q$, spread evenly
round the  base-circle.  The heart-shaped curve which all these circles touch is the
{\bf cardioid}.  The pointed part at $A$ is called the {\bf cusp}. (See
Figure 1) 

\beginfigure
\vskip 5truein
\begincaption
Figure \figdef{1}.  The Cardioid.
\endcaption
\endfigure


For a nephroid, we take the same base-circle, with any point $Q$ on the base-
circle as center, and draw a circle tangent to the y-axis.  We repeat
this for many positions of $Q$, spreading evenly round the base-circle.
The kidney-shaped
curve which all these circles touch is the {\bf nephroid}.  (See Figure
2) 

\beginfigure
\vskip 5truein
\begincaption
Figure \figdef{2}.  The Nephroid.
\endcaption
\endfigure

You are given the following procedure which draws cardioids:

\beginlisp
(define (cardioid r dx)
  (define fixed-x (- r))			;set up constant
  (define (new-y x)				;compute new y coordinate
    (sqrt (- (square r) (square x))))
  (define (new-r x)				;compute new radius
    (sqrt (abs (+ (square (new-y x)) (square (- x fixed-x))))))
  (define (iter x1)				;iteration procedure
    (circle x1 (new-y x1) (new-r x1))
    (circle x1 (- (new-y x1)) (new-r x1))
    (iter (+ x1 dx)))
  (clear-graphics)				;initialize graphics screen
  (iter fixed-x))
\endlisp

The arguments to the procedure are the radius $r$ of the base-circle and
an amount to increment the $x$ coordinate, which we call $dx$. 
The procedure contains internally defined
procedures {\tt fixed-x, new-y, new-r} and {\tt iter}.  The first three
compute the $x$ 
coordinate of the first $Q$ point, the $y$ coordinate of a selected $Q$ point, and
the radius of $QA$, respectively. The {\tt iter} procedure repeatedly
draws the 
circles with radius $QA$, and centers $(x1, y)$ and $(x1, -y)$, and increments the $x$ 
coordinate to get a new $Q$ point.  When cardioid starts, it clears the graphics
screen first by using the {\tt clear-graphics} Scheme primitive.

Two procedures which cardioid calls, the {\tt circle} procedure and the
{\tt square} procedure, are not available in Scheme.  They are given to you as follows.

\beginlisp
(define (circle x y r)
  (define da 0.2)				;set up constant
  (define 2pi (* 2 3.151492))			; ditto
  (define (iter a)				;iteration procedure
    (cond ((<= a (+ da 2pi))			;if within bounds, draw and interate
           (draw-line-to (+ x (* r (cos a))) (+ y (* r (sin a))))
           (iter (+ a da)))))
  (position-pen (+ x r) y)			;position pen initially
  (iter 0))					;start at 0
\pbrk
(define (square x) (* x x))
\endlisp

The {\tt circle} procedure simply draws a circle which has $(x, y)$ as its
center and $r$ as its radius.  
The primitive {\it position-pen}
moves the graphics pen to the point indicated by the given
coordinates.  The primitive {\it draw-line-to} moves the graphics pen to
the indicated coordinates, drawing a line from the current point.

{\bf Exercise 1}

Use the NMODE editor to type in the above procedure definitions.  (Subsequent
laboratory assignments will include large amounts of code, which will be made
available for you to load when you begin work on the assignment.  This time,
however, we are asking you to type in the definition yourself, so that you 
will get used to the editor.)  Notice that the editor automatically ``pretty 
prints'' your procedure as you type it in, indenting lines to the position
determined by the number of unclosed left parentheses.  Notice also
that, when you type a right parenthesis, the matching left parenthesis
is briefly high-lighted.
In the code above, words to the right of semicolons are
comments, and need not be typed.  After you've typed in everything, save the
procedure on your disk in a file named {\tt ps1.scm}. 

``Zap'' your procedures from the editor into Scheme.\footnote{$^1$}{Use
the SEND BUFFER command (SHIFT-k$_7$ on the Chipmunk keyboard).
This transmits the entire buffer to Scheme, and is the easiest thing
to use when you only have one or two procedures in your buffer.  With
larger amounts of code, it is better to use other ``zap'' commands,
which transmit only selected parts of the buffer, such as individual
procedure definitions.  For a description of these commands, see the
description of NMODE in the Chipmunk manual, under the heading
``Commands for use with Scheme.''}
First, run the {\tt circle}
procedure by typing in the expression:

\beginlisp
	(circle 0 0 20)
\endlisp

(Press EXECUTE after typing in the expression.)  A circle should appear
on your screen.  To see the drawing, press the key marked GRAPHICS at the
upper right of the keyboard.  The Chipmunk system enables you to view
either text or graphics on the screen, or to see both at once.  This
is controlled by the keys marked GRAPHICS and ALPHA.  Pressing
GRAPHICS shows the graphics screen superimposed on the text.  Pressing
GRAPHICS again hides the text screen and shows only graphics.  ALPHA
works similarly, showing the text screen and hiding the graphics
screen.

Try different values for $x, y$ and $r$.   To clear the screen between trials you may execute
the primitive procedure

\beginlisp
(clear-graphics)
\endlisp


If you typed in your definitions incorrectly, you will need to debug the 
definitions by returning to the editor and make corrections.

Note that if your computation is taking a long time to complete, you can
abort
execution by typing control-G.  (The CONTROL key at the upper left
of the keyboard is used like a SHIFT key.  To type control-G, hold
down the CTRL key and press G.)

In doing this problem, you can save typing by taking advantage of the
``line edit'' and ``recall'' features of Scheme.  Pressing the
RECALL key will redisplay the last Scheme expression evaluated.
This can then be edited using many of the NMODE edit commands.\footnote{$^2$}{In
general, you can use all the edit commands that keep the cursor within
a single line.  An exception: Do not use CTRL-B if you want to
move the cursor to the left.  Use the left-arrow key.} For instance,
suppose you have just evaluated {\it (circle 0 0 100)} 
and you now want to evaluate {\it (circle 0 0 20)}.   You can press RECALL,
use the arrow keys to move the cursor to the character to be changed,
make the edit, and finally press EXECUTE to evaluate the modified
expression.  RECALL always redisplays the previous expression.
Once you have pressed RECALL, you can repeatedly press
CTRL-RECALL to e
procedure {\tt new-y} is called many times with the same argument in each
iteration.
Modify the {\tt cardioid} procedure so that {\tt new-y} is called only
once in each  iteration.

Hand in a listing of your modified {\tt cardioid} procedure.  How many
times was {\tt new-y} called before your modification??

{\bf Exercise 4}

We can also graph the cardioid envelope itself without using circles.  The
following is a pair of parametric equations of its envelope.

\math{\eqalign{
	x &= 2a (\cos t) + a (\cos 2t)\cr
	y &= 2a (\sin t) + a (\sin 2t) \qquad 0 \leq t \leq \infty\cr}}

Write a procedure called {\tt envelope}, which takes 2 arguments: {\tt a}
and {\tt dt}, where {\tt a} determines the size of the envelope, and
{\tt dt} determines the increment of $t$, and which draw the envelope of
the corresponding cardioid.
[Hint: use the Scheme primitives {\tt position-pen}  and {\tt
draw-line-to}.  Refer to the {\tt circle} procedure for hints on how to use them]. 

Hand in a listing of your procedure, and a cardioid envelope created by it.

{\bf Exercise 5}

If we vary the above parametric equations by
\math{\eqalign{
	x &= 3a (\cos t) + a (\cos 3t)\cr
	y &= 3a (\sin t) + a (\sin 3t) \qquad 0 \leq t \leq \infty\cr}}
where we have changed all the 2's to 3's, we get an envelope of the
nephroid.  Modify the {\tt envelope} procedure
so that the constant 2 becomes a variable, which is identified in the 
parameter-list of the {\tt envelope} procedure.  Use this new {\tt
envelope} procedure to graph a nephroid.

Hand in a listing of your new {\tt envelope} procedure, a picture of a
nephroid
envelope and the expression you entered in order to graph the
envelope. 

{\bf Exercise 6}

In this exercise, you are asked to write a procedure which graphs the nephroid
using the circle method described earlier before exercise 1 (see figure 2).  
This procedure should take 2 arguments: r and da, where r is the radius of the
base-circle, and da is the increment of an angle.  Recall the equations of a
circle:
\math{\eqalign{
	x & = r (\cos a)\cr
	y & = r (\sin a)\qquad 0 \leq a \leq 2\pi\cr}}

By  gradually iterating over {\tt a} from 0 to $2\pi$, we can locate the $x$
and $y$ coordinates on the base-circle.  The strategy is to use every $(x,y)$ pair thus 
generated as the $Q$ point of the circles.  [Hint: this procedure should look
very similar to the {\tt cardioid} procedure.  If you found yourself
doing something
much more complicated, you are probably on the wrong track.]  Test out your
procedure by executing the following expression:

\beginlisp
	(nephroid 60 0.1)
\endlisp

Hand in a listing of the {\tt nephroid} procedure and a picture created
by it. 

{\bf Exercise 7}

What is the relationship between the parameter {\tt a} in the {\tt
envelope} procedure and
the parameter {\tt r} in the {\tt cardioid} procedure??  In the {\tt
nephroid} procedure?? Graph a few pictures to find out.

{\bf Exercise 8 (for fun)}

Sketch what you see when you execute the following:

\beginlisp
	(envelope 4 20 0.1)
	(envelope 5 20 0.1)
	(envelope 6 20 0.1)
	(envelope 10 20 0.1)
\endlisp


\chapter{3. Adventures in Debugging}

During the semester, you will often need to debug programs that
you write.  This section contains an exercise that you
should work through {\it at the terminal} to acquaint you with some
of the features of Scheme to aid in debugging.  There is
nothing you need turn in for this part of the problem set, but we
strongly suggest that you do this exercise.   Learning to use the
debugging features will save you much grief on later problem
sets.

The file {\it ps1-debug.scm} is a short file that has been provided for
you to load into Scheme.  To do this, use the NMODE extended
command ``load problem set'' and specify that you want to load
the code for Problem Set 1.  This will make a copy of the file on
your disk.

The file contains three tiny procedures called {\it p1, p2} and
{\it p3}.  You can examine them using NMODE, but there is no need to do
that now -- you will see them later on, during debugging. Zap the
procedures into Scheme.  Now evaluate the expression {\it (p1 1 2)}.
This should signal an error, with the message

\beginlisp
Wrong Number of Arguments 1
within procedure \#[COMPOUND-PROCEDURE P2]
There is no environment available;
using the current read-eval-print environment.
\pbrk
2 Error-$>$
\endlisp

Don't panic.  Beginners have a tendency, when they hit an error,
to quickly type CTRL-G or press EDIT. Then they stare at
their code in the editor until they see the what the bug is.
Indeed, the example here is simple enough so that you probably
can see the bug just by reading the code.  But don't do
this.  Let's instead see how Scheme can be coaxed into producing
some helpful information about the error.

First of all, there is the error message itself.  It tells you
that the error was caused by a procedure being called with 1
argument, which is the wrong number of arguments for that
procedure.  The next line tells you that the error occurred
within the procedure
{\it P2}, so the actual error was that {\it P2} was called with 1
argument, which is the wrong number of arguments for {\it P2}.

The other information about ``current environment,'' and
``read-eval-print environment'' you can ignore.  We will
learn about environments later in the semester.  Notice, however,
that the prompt now reads
\beginlisp
2 Error-$>$
\endlisp
\vpar
which is Scheme's way of indicating that you can now
evaluate expressions within the context of the error, and that
this context is ``at level 2,'' namely, one level down from the
``top level,'' which is the initial Scheme command level.

Unfortunately, the error message alone doesn't say where in
the code the error occurred.  In order to find out more, you need
to use the debugger.  To do this execute the expression {\it (debug)}.

\section{Using the debugger}

The debugger allows you to grovel around
examining pieces of the execution in progress, in order to learn
more about what may have caused the error.
When you typed {\it (debug)} just above, Scheme should have responded:
\beginlisp
Subproblem Level: 0  Reduction Number: 0
Expression:
(P2 B)
within the procedure P3
applied to (1 2)
Debugger
\pbrk
3 Debug--$>$
\endlisp

This says that the expression that caused the error was
{\it (P2 B)}, evaluated within the procedure {\it P3}, which was
called with arguments 1 and 2.  That's probably enough
information to let you fix the error, but let's pretend we need
more information, just to learn more about how the debugger works.

Ignore the stuff about Subproblem and Reduction numbers.  (You can
read about them in the Chipmunk manual.)  The prompt says that you are
in the debugger (at level 3, one down from the error level, which was
level 2).  At this point, you could type CTRL-G to exit the whole
mess and get back to Scheme's ``top level,'' but let's instead explore
the debugger.

The debugger differs from an ordinary Scheme command level, in
that you use single-keystroke commands, rather than typing expressions
and pressing EXECUTE.


One thing you can do is move ``backwards'' in the evaluation
sequence, to see how we got to the point that signaled the error.
To do this, type the character {\it b}.  Scheme should respond:
\beginlisp
Subproblem level: 1 Reduction number: 0
Expression:
(+ (P2 A) (P2 B))
within the procedure P3
applied to (1 2)
\pbrk
3 Debug--$>$
\endlisp
Remember that the expression evaluated to cause the error was {\it (P2
B)}.  Now that we have moved ``back,'' we learn that this expression
was being evaluated as a subproblem of evaluating the expression
\beginlisp(+ (P2 A) (P2 B))\endlisp, still within procedure {\it P3}
applied to 1 and 2.

Press {\it b} again, and you should see
\beginlisp
Subproblem Level: 2  Reduction Number: 0
Expression:
(+ (P2 X Y) (P3 X Y))
within the procedure P1
applied to (1 2)
\endlisp
which tells us that got to the place we just
saw above as a result of trying to evaluate an expression in
{\it P1}.

Press {\it b} again and you should see a horrible mess.  What you
are looking at is some of the guts of the Scheme system -- the part
shown here is a
piece of the interpreter's read-ees that in this case
Scheme was evaluating the arguments to {\it +} in right-to-left
order, which is something you may not have expected. You should
never write code that depends for its correct execution on the
order of evaluation of the arguments in a combination.  The
``official definition'' of Scheme (whatever that means) does not
guarantee that that any particular order will be followed, nor
even that this order will be the same each time a combination is
evaluated.} You can type CTRL-G to return to the Scheme top level
interpreter.

\section{The stepper}

Even with such a simple error, debugging was not so simple.
What do you expect?  After all, an error occurred and you
have to grovel through the mess.  An alternative debugging
strategy is to re-evaluate the expression that produced the error, only
this time to do the evaluation ``step by step,'' and observe the
precise point where the error happens.  The Scheme program that
lets you do this is called the {\it stepper}.

Let's try to debug again, this time using the stepper.  Clear the
screen (by typing CTRL-L) and evaluate the expression

\beginlisp
(step '(p1 1 2))
\endlisp

Don't omit the quotation mark after {\it step}.   (You'll learn
about what quotation is for when we get to Chapter 2.)

If you did this correctly, Scheme should type, halfway down the screen,
\beginlisp

(DEFINE (P1 X Y) (+ (P2 X Y) (P3 X Y)))
Stepping: (P1 1 2) Eval Combination
\endlisp

What you are seeing is the text of the procedure being evaluated,
namely {\it P1}.  The next line tells you that the expression being
stepped that called this procedure is {\it (P1 1 2)} and that the
interpreter is at the point ``Eval combination'' which is the
point where the interpreter sets up to evaluate a
combination.  (Later in the semester, we will learn about the
various elements of the evaluation process.)  The cursor is
flashing at the beginning of the combination to be evaluated.

Type a space.  This proceeds the evaluation ``one step.''  Notice
that nothing much happens, except that the cursor moves to the
beginning of the right-hand inner combination, which is the next
thing the interpreter is going to work on.\footnote{$^4$}{There's that right-to-left
evaluation order again.  See previous footnote.}

Type another space and the interpreter starts working on the
subexpressions of this combination, starting with the symbol {\it Y}.
Accordingly, the stepper prints
\beginlisp

Eval Irreducible
Y has the value
2
\endlisp
telling you that {\it Y} was irreducible (no subexpressions) and
that it has the value 2.

Type another space and you will see that {\it X} has the value 1.
Another space shows the interpreter determining that {\it P3} is the
name of a procedure.
\beginlisp

P3 has the value
\#[COMPOUND-PROCEDURE P3]
\endlisp

When you type another space, the message changes to show that the
interpreter is about to apply {\it P3} to the arguments 1 and 2.

You now have a choice: you can ask the stepper to do the
application in one step, or you can go step-by-step through the
application.

If you press space, the stepper will do the application in one step and
show the returned value.  Try this.

Ugh!  That signaled the error.  So we know that the error happens
while evaluating {\it (P3 1 2)}.  We probably should have gone step
by step through this part of the evaluation.

So go back and try again.  Press CTRL-G, clear the screen,
execute
\beginlisp

(step '(p1 1 2))
\endlisp
and press space until you get back to the point where the
interpreter is about to apply {\it P3} to 1 and 2.

This time, instead of pressing the space bar, type {\it s}.
The stepper should now show the application of the internal procedure:
\beginlisp

(DEFINE (P3 A B) (+ (P2 A) (P2 B)))
\endlisp

Another space will get to the application {\it (P2 B)}, which gives
the error.

Notice that this is the same information we found with the
debugger above.  Only this time we are working ``forwards'' until
the error occurs instead of letting the error happen and then
working backwards.  Both of these techniques should be part of
your debugging repertoire.


\section{For your amusement}

Edit {\it P3} to eliminate the error by calling {\it P2} with two
arguments each time.  (It doesn't matter what these
arguments are, so long as the applications don't produce errors.) 
Now step through the original call to {\it P1}, and get to the point in
{\it P3} where the addition operator {\it +} is about to be applied
to 2 arguments.  If you were to type a space at this point, the
addition would be performed and the result shown.  But, instead
of typing a space, type an {\it s}, and you will see some incredibly
obscure stuff.

What you are looking at are the internals of the addition
operation, implemented as a Scheme procedure.  You see, we lied
to you in lecture: {\it +} is not a primitive in the Scheme system
you are using.

The lesson to draw from this is not that we are liars, but that
the meaning of ``primitive'' must always be taken to be relative
a given level of detail at which one wants to look.  Since we
have more than enough material to cover this semester without
worrying about how the 6.001 Scheme system is implemented in
detail, it is convenient for us to consider {\it +} to be a primitive.

For fun, you can start stepping through the application of {\it +}.
Each time you get to an internal application, type {\it s} to go
inside of it.  Eventually you will get to a ``real primitive.''
Of course, all that a ``real primitive'' is is something that the
stepper won't let you look inside of.  In terms of the assembly
language ``kernel'' of Scheme, a ``real primitive'' might be a
complex program made up of lots of instructions in assembly
language.  And any single assembly language instruction might be
a complex program written in microcode.

This illustrates a major idea that we will see thoughout 6.001: One
does {\it not} create complex systems by focusing on how to construct
them out of ultimate primitive elements.  Rather, one proceeds in
layers, erecting a series of levels, each of which is the
``primitives'' upon which the next level is constructed.

\end

