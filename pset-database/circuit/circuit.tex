% Copyright (c) 1990 Massachusetts Institute of Technology
% 
% This material was developed by the Scheme project at the Massachusetts
% Institute of Technology, Department of Electrical Engineering and
% Computer Science.  Permission to copy this material, to redistribute
% it, and to use it for any non-commercial purpose is granted, subject
% to the following restrictions and understandings.
% 
% 1. Any copy made of this material must include this copyright notice
% in full.
% 
% 2. Users of this material agree to make their best efforts (a) to
% return to the MIT Scheme project any improvements or extensions that
% they make, so that these may be included in future releases; and (b)
% to inform MIT of noteworthy uses of this material.
% 
% 3. All materials developed as a consequence of the use of this
% material shall duly acknowledge such use, in accordance with the usual
% standards of acknowledging credit in academic research.
% 
% 4. MIT has made no warrantee or representation that this material
% (including the operation of software contained therein) will be
% error-free, and MIT is under no obligation to provide any services, by
% way of maintenance, update, or otherwise.
% 
% 5. In conjunction with products arising from the use of this material,
% there shall be no use of the name of the Massachusetts Institute of
% Technology nor of any adaptation thereof in any advertising,
% promotional, or sales literature without prior written consent from
% MIT in each case. 

% -*- LaTeX -*-

\input /zu/6001/6001mac

\def\resistor{$\wedge\wedge\wedge\wedge$}

\begin{document}

\psetheader{Fall Semester, 1989}{Problem Set 5}

\medskip

\begin{flushleft}
Due: In recitation, Wednesday 25 October

Reading: Finish chapter 2, then read through section 3.1.
\end{flushleft}

\mbox{}\hrulefill\mbox{}

\begin{center}
{\bf Quiz I Reminder}
\end{center}

Quiz I is on Thursday, October 19, in 50-340 (Walker Memorial Gymnasium), from
5-7 PM xor 7-9 PM.  You may take the quiz during either of the two sessions,
but students taking it in the first session must stay until 7 PM.  The quiz is
open book.

The quiz covers everything up to and including recitation Friday, October 6
(symbols and quotation).

\mbox{}\hrulefill\mbox{}

\section{Series-Parallel Resistive Networks}

Louis Reasoner, appalled by his tuition bill, is trying to put his computer
background to good use by writing an electrical circuit analyzer program that
he will rent out to 6.002 students.  Louis's partner in this scheme is his
roommate, Lem E. Tweakit, a 6-1 major with a flair for hardware.  Lem suggests
that they begin with a program that computes resistances of simple ``linear
resistive networks.''\footnote{In this problem set, we use the words
``network'' and ``circuit'' interchangeably.  In the context of Louis's
program, there is no distinction.} These are constructed from primitive
elements called {\it resistors}.  A resistor is characterized by a number {\it
R} called the {\it resistance} of the resistor.  It can be depicted as follows:

$$\vbox{\halign{#\cr $_o$\vbox{\hrule width5em} \resistor \vbox{\hrule
width5em} $_o$\cr
\hfil R\hfil\cr}}$$

For computational purposes, it is also convenient to consider the
{\it conductance} of a resistor, which is defined to be the reciprocal of
the resistance. (Conductance is traditionally denoted by the letter
{\it G}.)

$$\vbox{\halign{#\cr
$_o$\vbox{\hrule width5em} \resistor \vbox{\hrule width5em} $_o$\cr
\hfil G = 1/R\hfil\cr}}$$

A resistor is the simplest example of a kind of element called a
{\it two-terminal network}, i.e., a network that has exactly two terminals to
which other objects can be connected:

$$\vbox{\halign{#&#&#&#&#\cr
&&\vbox{\hrule width5em}&&\cr
&\vrule height11pt depth11pt& &\vrule height11pt depth11pt&\cr
$_o$\vbox{\hrule width5em}&\vrule height11pt depth11pt&\hfil N\hfil &\vrule height11pt depth11pt&\vbox{\hrule width5em} $_o$\cr
&\vrule height11pt depth11pt& &\vrule height11pt depth11pt&\cr
&&\vbox{\hrule width5em}&&\cr}}$$


Networks can be combined by attaching their terminals together.  A
network has a resistance (and conductance) that is determined by the
resistances of its parts and the ways in which they are
interconnected.  Louis's and Lem's initial system will provide two
basic methods of combination for constructing two-terminal networks
from simpler two-terminal networks.  The first method is called {\it series}
combination:

$$\vbox{\halign{#&#&#&#&#\cr
&\vbox{\hrule width5em}& &\vbox{\hrule width5em}& \cr
&\vrule height11pt depth11pt\hfil\vrule height11pt depth11pt& &\vrule height11pt depth11pt\hfil\vrule height11pt depth11pt& \cr
$_o$\vbox{\hrule width5em}&\vrule height11pt depth11pt\hfil A\hfil \vrule height11pt depth11pt&\vbox{\hrule width5em} &\vrule height11pt depth11pt\hfil B \hfil\vrule height11pt depth11pt&\vbox{\hrule width5em}$_o$ \cr
&\vrule height11pt depth11pt\hfil\vrule height11pt depth11pt& &\vrule height11pt depth11pt\hfil\vrule height11pt depth11pt& \cr
&\vbox{\hrule width5em}& &\vbox{\hrule width5em}& \cr}}$$

\centerline{$R = R_A + R_B$}

The combined resistance of two networks connected in series is the sum
of the resistances.

The second method is {\it parallel} combination: 

\begin{minipage}[t]{\linewidth}
\vspace{3in}

\centerline{$G = G_A + G_B$}
\end{minipage}

The combined conductance of two networks joined in parallel is the sum
of the conductances of the pieces.

After a few hours of work, Louis and Lem have quite an
elegant program that handles such series and parallel
combinations.\footnote{Networks that can be constructed as series and
parallel combinations of primitive two-terminal elements are called
{\it series-parallel networks}.} The program is provided in the first
appendix to this handout, labeled {\it ps5-res.scm}.

You should read through this code now to get some idea of how it
works.  There are constructors {\it make-resistor}, {\it make-series}, and
{\it make-parallel}, and an operation called {\it resistance} that
computes the resistance of a network by finding the resistances of the
parts of a network and combining them appropriately.

Louis and Lem test their program by computing the resistance of the
following network:

\vspace{3in}

They type:

\beginlisp
(define r1 (make-resistor  5))
(define r2 (make-resistor 15))
(define r3 (make-resistor 10))
(define r4 (make-resistor 10))

(define N
  (make-parallel (make-series r1 r2)
                 (make-series r3 r4)))

-->(resistance N)
10.000
\endlisp
Indeed, the answer is correct (as Lem verifies).

{\bf Exercise 1:}
Consider the data structure representing the network {\it N}.

{a.}
Draw the box-and-pointer structure that represents the object {\it N}.

{b.}
How many times was {\it resistance} called in computing the resistance of
{\it N}?  What were the arguments to {\it resistance} for each call?  (To
specify the arguments, you can sketch the part of the network that
each argument represents, or you can give a description in English,
e.g., ``{\it R3} in parallel with {\it R4}''.)

\begin{minipage}[t]{\linewidth}
\section{L-extensions}

Given a 2-terminal network {\it B}, we can form a new two-terminal
network by attaching an ``L-section,'' which is itself constructed
from two two-terminal networks {\it S} and {\it P} as follows:

\vspace{3.5in}
\end{minipage}

This operation is called ``L-extension'' of a base {\it B} by a ``series
part'' {\it S} and a ``parallel part'' {\it P}.

{\bf Exercise 2:}
Write a procedure {\it L-extend} that takes three two-terminal networks --
{\it base}, {\it series-part}, and {\it parallel-part} -- and combines them
using {\it make-series} and {\it make-parallel} to produce the extended
network as shown above.


{\bf Exercise 3:}
Use your {\it L-extend} procedure to make a new network that extends the
network {\it N} from Exercise 1 (as the base) by a 5-ohm resistor (the
series part) and a 10-ohm resistor (the parallel part).

Verify that the resulting network has a resistance of 10 ohms.  Show
the expressions that you typed in order to generate the network and to
check its resistance.



By repeatedly L-extending a base by a given series and parallel
part, we obtain a circuit called a ``ladder.''  The following diagram
shows a 4-stage ladder.

\vspace{3in}

Louis decides that he can easily implement a ladder by making use of
his {\it L-extend} procedure, together with the {\it repeated} procedure
below.  Note that {\it identity} is a procedure that simply returns its
argument, and {\it compose} is a procedure that, given two procedures
{\it f} and {\it g} as arguments, creates a new procedure that applies
the composition of {\it f} and {\it g} to its argument.

\beginlisp
(define identity (lambda (x) x))

(define compose (lambda (f g)
                  (lambda (x) (f (g x)))))

(define (repeated f n)
  (if (= n 0)
      identity
      (compose f (repeated f (- n 1)))))
\endlisp

He defines the following procedure to construct a ladder with a
given number of stages:

\beginlisp
(define (ladder-extension stages base series-part parallel-part)
  ((repeated {\it $<$EXP-1$>$} stages) {\it $<$EXP-2$>$}))
\endlisp

{\bf Exercise 4:}
Complete the {\it ladder-extension} procedure.  What are the missing
expressions
{\it $<$EXP-1$>$} and {\it $<$EXP-2$>$}?

{\bf Exercise 5:}
A classical network-theory problem used to plague 6.002 students is to
compute the resistance of an infinitely long ladder of 1-ohm
resistors:  

\vspace{3in}

The procedure {\tt make-tolerance-check} below creates a procedure of two
parameters which checks whether those parameters are within {\tt tolerance}:

\beginlisp
(define make-tolerance-check
  (lambda (tolerance)
    (lambda (number1 number2)
      (< (abs (- number1 number2)) tolerance))))
\endlisp

Now, fill in the missing parts of the procedure {\tt solve-infinite-ladder}
below, which takes a {\tt tolerance} and calculates the resistance of longer
and longer ladders until it finds two answers within {\tt tolerance} of each
other.  The missing parts are labaled {\tt <exp1>} and {\tt <exp2>}.

\beginlisp
(define (solve-infinite-ladder tolerance)
  (let ((close-enough? (make-tolerance-check <exp1>))
        (one-ohm-resistor (make-resistor 1)))
    (define (loop circuit circuit-resistance)
      (let ((new-circuit (l-extend circuit
                                   one-ohm-resistor
                                   one-ohm-resistor)))
        (let ((new-resistance (resistance new-circuit)))
          (if <exp2>
              new-resistance
              (loop new-circuit new-resistance)))))
    (loop one-ohm-resistor (resistance one-ohm-resistor))))
\endlisp

Include the expressions you used for {\tt <exp1>} and {\tt <exp2>} in your
problem set solutions.

Using your completed {\tt solve-infinite-ladder}, calculate the resistance of
an infinite ladder of one-ohm resistors within a tolerance of 0.0001.  To what
value does the resistance appear to converge?  Have you seen this number
before?

{\bf Exercise 6:}
Louis and Lem demonstrate their system to Ben Bitdiddle.  As an
example, they try to compute the resistance of a long ladder.  To
their surprise, they find that although the program gets correct
answers, it seems to run more slowly than before.

Taking a careful look at the code, they find that Louis has changed
the definition of {\it resistance-parallel} so that it now reads:
\beginlisp
(define (resistance-parallel ckt)
  (/ (* (resistance (left-branch ckt))
        (resistance (right-branch ckt)))
     (+ (resistance (left-branch ckt))
        (resistance (right-branch ckt)))))
\endlisp
Explain why this change makes the program run so slowly.  As an
example, consider a ladder where each of the series, parallel, and
base pieces is a simple resistor.  If the ladder has {\it N} stages:

(a) How many resistors are contained in the ladder?

(b) In computing the resistance of the ladder, how many times will the
procedure {\it resistance-resistor} be run if the system uses the
original version of the {\it resistance-parallel} procedure?

(c) After Louis installs the new version of {\it resistance-parallel}, how
many times will {\it resistance-resistor} be run in computing the resistance
of the {\it N}-stage ladder?  (You should be able to obtain an exact answer
in terms of {\it N}.  However, partial credit will be given for good partial
answers.  Show your work.)

(d) How has Louis's change affected the running time of the system?
(E.g., slowed it down by a constant factor?  slowed it down
quadratically? slowed it down exponentially?)


\section{General Series-Parallel Circuits}

Louis goes off to try to rent out the system in the 6.002 lab in
building 38.  His first potential customer is Anna Logg, a sophomore
who is busily doing her 6.002 problem set.  ``Foo,'' says Anna, ``this
is useless unless it can handle capacitors and inductors as well as
resistors.''

Louis goes back home and asks Lem for a short course on circuit
theory.  Lem says that indeed, interesting circuits have elements
called capacitors and inductors, and solving such circuits involves
(ugh!) differential equations, and Lem is too tired to
give a course in circuit theory (Louis has to take 6.002 next semester
anyway).  But luckily, Lem explains, it is unnecessary to understand
this in order to make a useful system.  In particular, explains Lem,
capacitors and inductors can be thought of as ``resistors'' whose
``resistance'' (actually called {\it impedance}) is a complex number that
varies with a parameter {\it s}, called the ``complex frequency.''

Lem says that for a capacitor of capacitance {\it C} (measured in farads)
the ``resistance'' is 1/{\it sC}.  For an inductor of inductance {\it L} (measured
in henrys) the ``resistance'' is {\it sL}:


\vspace{2in}

For example, a so-called ``parallel resonant circuit'' can be thought
of as a network of 3 ``resistors'' and analyzed in the same way as
before.  Only this time, the ``resistance'' of the network will be a
function of {\it s}.

\vspace{2in}

Louis meditates on this for a few days and turns for help to the great
wizard Alyssa P. Hacker.  Alyssa suggests a clever scheme that allows
Louis to handle these things without greatly changing his program.
The idea is that the ``resistance'' of a circuit (or a circuit
element), instead of being a simple number,
will now be a procedure that takes some {\it s} as argument.
For instance, the ``resistance'' of a
(real) resistor of declared resistance {\it R} is represented by a
procedure that computes a constant function whose value is {\it R}, for
any {\it s}.  For a capacitor of capacitance {\it C}, the ``resistance''
is the procedure that takes {\it s} as argument and returns 1/{\it sC}.

The data structure that Alyssa suggests for representing a primitive
electrical element is a list of three elements

\beginlisp
($<$type$>$ ($<$parameter name$>$ $<$value$>$) $<$procedure$>$)
\endlisp

This includes a {\it type} as in Louis's original program, a
``documentation field'' that makes printouts of circuits easier to
interpret, and the actual ``resistance'' procedure of the element.

Thus, a (real) resistor is constructed by

\beginlisp
(define (make-resistor resistance)
  (attach-type 'resistor
               (list (list 'resistance resistance)
                     (lambda (s) resistance)))))
\endlisp

A capacitor is constructed by

\beginlisp
(define (make-capacitor capacitance)
  (attach-type 'resistor
               (list (list 'capacitance capacitance)
                     (lambda (s)
                       (/ 1 (* capacitance s))))))
\endlisp

(Note that, as far as the {\it type} is concerned, a capacitor is a type
of resistor, although we can tell the difference by looking at the
documentation field.)  Inductors are constructed similarly.

Then, for example, the ``resistance'' of a series combination should
be a procedure that, given an argument {\it s}, computes the
resistances of the branches for that {\it s}, and returns their sum:

\beginlisp
(define (resistance-series ckt)
  (lambda (s)
    (+ ((resistance (left-branch ckt)) s)
       ((resistance (right-branch ckt)) s))))
\endlisp

Actually the procedures given above are not quite correct, because in
order to use this idea to get information about circuits, one needs to
use inputs {\it s} that are complex numbers.  (Ask Lem if you want to
know why, or take 6.002.)

So Louis has to make a further modification: All arithmetic must be
changed so that it works with complex numbers.  He builds a
representation for complex numbers and complex number operations and
modifies the code accordingly.  The result of his labors (which works,
amazingly!) is in the file {\it ps5-imp.scm}.  This is attached as the
second appendix to this problem set.  You should study the code and
use it to do the following exercises.

{\bf Exercise 7} Load Louis's new program into Scheme.\footnote{A version of
this file should have been copied to your disk and into an Edwin buffer when
you invoked the ``load problem set'' command.} Build a parallel resonant
circuit (as shown above) using a 2-ohm resistor, a 1-farad capacitor, and a
1-henry inductor.  Call the circuit {\it N2}.  What
is the impedance (``resistance'') of {\it N2} at $s=2+0j$?, at $s=0+2j$?  (Note
that Louis's code contains a constructor that makes complex numbers of the form
$A+Bj$, where $j$ is the square root of -1.)  Show the expressions you
evaluated in order to generate the circuit and to find the impedance.

Lem has suggested that it is very useful to look at a plot of the
magnitude of the impedance of a circuit as a function of imaginary
values of the complex frequency {\it s} (i.e., values of {\it s} for which
the real part is 0).  Such a plot is called a {\it frequency response}
plot of the circuit.

We have supplied a program that plots functions on the Chipmunk screen.  To get
an idea of how the program works, type {\it (line-plot square -2 2 .1)}.  This
plots values of {\it square} in the interval from -2 to 2 with an increment of
0.1.  Notice that {\it line-plot} prints the {\it x} and {\it y} bounds of the
plot, in the format {\it (xmin xmax ymin ymax)}.  (The file {\it
ps5-graph.scm}, automatically loaded when you load the problem set, implements
the {\it line-plot} procedure.  You need not read or understand the code in the
file in order to use {\it line-plot}.)

\begin{minipage}[t]{\linewidth}
{\bf Exercise 8}
Make a plot of the frequency response of the circuit {\it N2} from
$s=0+.001j$ to $s=0+2j$, stepping $s$ by $.05j$.  (Note that
the {\it ps5-imp} file contains a {\it magnitude} procedure for computing
magnitudes of complex numbers.)  You should get a curve like the one
shown in the figure.  The sharp peak is characteristic of a ``resonant
circuit,'' and the value of $s$ where the peak occurs is called the
{\it resonant frequency}.  Where is the resonant frequency for the
circuit {\it N2}?  What expressions(s) did you type to Scheme to
generate the plot?

\vspace{4.5in}
\end{minipage}

{\bf Exercise 9}
Use Louis's program to determine how the frequency response changes
if you change the resistance of the resistor from 2 to 10 ohms.
Sketch the new frequency-response curve, indicating how it differs
from the original curve.
Does the resonant frequency change?  Does the height of the peak
change?

\fbox{\huge NOTE: Everything after this point is OPTIONAL.}

\section{Generating Circuits}

Louis's program is so successful that it becomes standardly used by
almost all students taking 6.002.  Eventually, the Course 6 faculty
realizes that students are doing circuit-analysis homework simply by
using the program, resulting in the shocking state of affairs that
most students are able to spend less than 30 hours per week doing
6.002 problem sets!  At an emergency meeting of the Course 6
Undergraduate Educational Policy Committee, Prof. Gould suggests that
the 6.002 staff can increase the difficulty of the problem sets by
asking students to {\it design} circuits rather than simply to analyze
them: Students will be given a list of parts (resistors, capacitors,
and inductors) and asked to use these to build a circuit that has some
desired behavior, such as resonance in a given frequency range.

When Louis hears of this plan, he decides to augment his program so
that it can do this kind of problem as well.  His idea is to write a
program that will randomly generate series-parallel circuits using the
specified parts, then analyze the circuits to find ones with the desired
behavior.  (Each of the specified parts will be used exactly once in
each circuit.)

We can generate a random circuit as follows: Suppose that {\it S} is the
set of parts we are supposed to use.  We split {\it S} (randomly) into
two subsets {\it A} and {\it B}.  Then, recursively, we generate a random
circuit {\it CA} using the parts in {\it A} and a random circuit {\it CB}
using the parts in {\it B}.  Finally, we return either the series
combination of {\it CA} and {\it CB} or the parallel combination of {\it CA}
and {\it CB} (each with probability 1/2).

{\bf Exercise 10:}
Write a procedure {\it random-circuit} that uses this strategy to
generate random series-parallel circuit.  {\it Random-circuit} should
take as argument a set of parts, represented as a list.  You should
use a separate procedure called {\it split-list}, which takes as
argument a list {\it S} and returns a pair of lists {\it A} and {\it B},
such that each item in {\it S} appears either in {\it A} or in {\it B} (with
equal probability).  {\it Random-circuit} is then a recursive procedure
that uses {\it split-list}.  (Think about how to stop the recursion:
What do you do if one of the lists {\it A} or {\it B} is empty, or has
only one element?)  Turn in a listing of your procedure(s).

{\bf Exercise 11:}
Use your program to generate five different circuits whose parts are a
1-farad capacitor, a 2-farad capacitor, a 1-henry inductor, a 1-ohm
resistor, and a 2-ohm resistor.  Turn in the (pretty-printed) results.


{\bf Exercise 12:}
Using the plotter, determine which (if any) of the circuits you made has a
resonant frequency between $s=0$ and $s=0+2j$.  (Resonant behavior
is characterized by a sharp peak or a sharp dip at one or more
frequencies.)

{\bf Program design:} Do either one of the next two exercises.

{\bf Exercise 13:}
Describe carefully how you would write a program that automatically
tests circuits for resonant behavior.  You needn't actually implement
the program, but if you do, and have a lot of time, search for
series-parallel circuits with four components to find those with
resonant behavior.  Take the four components to be a 10-ohm resistor,
a 2-farad capacitor, a 1-farad capacitor, and a 2-henry inductor.
Search the region from $s=0$ to $s=0+2j$, as before.

{\bf Exercise 14:}
One problem with the random method of generating circuits is that the
same circuit might be generated many times, and some circuits might
not be tried at all.  How would you write a program to generate
{\it all} the circuits that can be constructed from a given list of
parts (with each part used exactly once)?  Notice that there are
really two problems here: one is to make sure that each circuit is
generated, and the other is to eliminate duplicates.  Eliminating
duplicates can be tricky: for instance, given parts {\it p1}, {\it p2},
and {\it p3}, the combination
\beginlisp
(make-series p1 (make-series p2 p3))
\endlisp
represents the same circuit as
\beginlisp
(make-series p2 (make-series p1 p3))
\endlisp
You needn't write any code for this exercise, but be sure to give a
good description of how you would go about solving the problem.


Warning/Hint: There is a clever way to eliminate the duplicates, but a
straightforward approach can become extremely complicated.  Don't be
afraid to turn in a partial solution if you don't discover a complete
method.  Also, if you actually implement a program and want to test
it, you should know that there are 8 distinct circuits that can be
constructed with 3 parts, 52 with 4 four parts, and 472 with 5 parts.

\end{document}
