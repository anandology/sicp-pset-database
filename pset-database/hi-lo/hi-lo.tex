% -*- Mode: TeX -*-
% for yTeX
% Copyright (c) 1990 Massachusetts Institute of Technology
% 
% This material was developed by the Scheme project at the Massachusetts
% Institute of Technology, Department of Electrical Engineering and
% Computer Science.  Permission to copy this material, to redistribute
% it, and to use it for any non-commercial purpose is granted, subject
% to the following restrictions and understandings.
% 
% 1. Any copy made of this material must include this copyright notice
% in full.
% 
% 2. Users of this material agree to make their best efforts (a) to
% return to the MIT Scheme project any improvements or extensions that
% they make, so that these may be included in future releases; and (b)
% to inform MIT of noteworthy uses of this material.
% 
% 3. All materials developed as a consequence of the use of this
% material shall duly acknowledge such use, in accordance with the usual
% standards of acknowledging credit in academic research.
% 
% 4. MIT has made no warrantee or representation that this material
% (including the operation of software contained therein) will be
% error-free, and MIT is under no obligation to provide any services, by
% way of maintenance, update, or otherwise.
% 
% 5. In conjunction with products arising from the use of this material,
% there shall be no use of the name of the Massachusetts Institute of
% Technology nor of any adaptation thereof in any advertising,
% promotional, or sales literature without prior written consent from
% MIT in each case. 


\typesize=11pt
\hsize=34pc
\vsize=50pc
\parskip 6pt plus 2pt

\def\fbox#1{%
  \vtop{\vbox{\hrule%
              \hbox{\vrule\kern3pt%
                    \vtop{\vbox{\kern3pt#1}\kern3pt}%
                    \kern3pt\vrule}}%
        \hrule}}

\def\psetheader{
\centerline{MASSACHUSETTS INSTITUTE OF TECHNOLOGY}
\centerline{Department of Electrical Engineering and Computer Science}
\centerline{6.001 Structure and Interpretation of Computer Programs}
\centerline{Fall Semester, 1988}}

\def\code#1{\beginlisp
#1
\endlisp

\vskip .1in}

\rectoleftheader={6.001 -- Fall Semester 1988}
\rectorightheader={Problem Set 2}
\onheaders
\onfooters

\null
\vskip 1truein

\psetheader

\vskip .25truein

\centerline{Problem Set 2}

\vskip 0.25truein

\vpar
Issued:  September 20, 1988


\vpar
Due: in recitation on September 30, 1988 {\it for all sections}

\vpar
Reading: From text, reminder of Chapter 1.


\vpar 
{\bf Reminder:}  There will be a mini-quiz in recitation on Wednesday,
September 28.

\chapter{1. Homework exercises}

Work the following exercises.  Write up your
solutions and submit them as part of your homework to your TA in
recitation.

\beginbullets

\bpar
Exercise 2.1: Do Exercise 1.25 in Abelson and Sussman, p. 56.

\bpar
Exercise 2.2: Do Exercise 1.26 in Abelson and Sussman, p. 56.

\bpar
Exercise 2.3: Do Exercise 1.32 in Abelson and Sussman, p. 69.

\bpar
Exercise 2.4: Do Exercise 1.33 in Abelson and Sussman, p. 70.

\bpar
Exercise 2.5:  Using the RSM (revised substitution model), show how the
following expressions evolve when being evaluated.  Indicate which rule
you used to get from one step to another.  You may assume that {\tt
square} has been defined by 

\beginlisp
(define square (lambda (x) (* x x)))
\endlisp

\atpar{a.} {\tt (define pass (lambda (m) (lambda (z) (m z))))}

\atpar{b.} {\tt ((lambda (x) ((pass x) 3)) square)}

\atpar{c.} {\tt (define use-pass (pass (lambda (a) (square (+ a 1)))))}

\atpar{d.} {\tt (use-pass 2)}



\endbullets

\chapter{2. Programming Assignment:  Big Bucks for Stuck Trucks}

Ben Bitdiddle has been spending some of his spare time working for
WMBR (Walker Memorial Basement Radio).  Concerned that the size
of station's listenership is declining (it used to be bigger when the station was
WTBS, before the call letters were ``sold'' to  Ted Turner), Ben
reasons that holding a promotional event is a good way to increase
listenership.  He decides to run a contest in which, at particular times
during the day, listeners can call in for the opportunity to be selected
as a contestant.  Each contestant has a limited time period (10 seconds)
in which to guess an integer known to the radio station.  If the
contestant correctly guesses the hidden integer, she wins a cash prize,
if not she wins an MIT tee shirt.

To add some spice to the contest, Louis decides that the number to be
guessed by the participants will be the amount of money spent by
You-Hall to repair all of its rental trucks that become stuck under the
bridges on Memorial and Storrow Drive during the fall term, taken to the
nearest dollar.  To make the contest a little easier, the radio station
announces that this hidden amount is bounded between a given upper and
lower limit, whose values the station also announces.  Since contestants
will only have 10 seconds to guess the amount, it is important to guess
the hidden sum in as few guesses as possible.

When he hears about the contest, Louis Reasoner becomes very excited.
``What a great chance to earn some serious bucks!'' he exclaims.  ``In
fact, I can use the computer to help me figure out the best way to guess
the hidden number.''

Louis decides to write a procedure that will count the number of guesses
it takes to guess a hidden number.  His idea is that by trying this
procedure out on a number of different examples of hidden numbers, he
can determine how many guesses, on average, are needed to guess the
number, and from this he can determine how quickly he must be able to
enunciate guesses in order to get in that many guesses in 10 seconds.

We have helped Louis a little by providing a procedure called {\tt
setup}.  {\tt Setup} takes two arguments, a lower limit and an upper
limit in that order, and generates two global variables, {\tt
high-limit} which is bound to the value of the argument supplied as an
upper limit to {\tt setup}, and {\tt low-limit} which is bound to the
value of the argument upplied as a lower limit to {\tt setup}.  It also
chooses a random integer between the two limits, and creates three
special procedures, {\tt =-hidden}, {\tt <-hidden} and {\tt >-hidden},
each of which takes a single argument and evaluates to {\bf true} if the
value of the argument is equal to, less than, or greater than the value
of the random integer, respectively.

Louis' first idea is to simply guess integers at random between the
given limits until the correct answer is found.  Below is a procedure,
called {\tt strategy-random}, that guesses integers between {\tt
low-limit} and {\tt high-limit} until it guesses the integer represented
by {\tt hidden}, at which point it returns the number of guesses it made
in order to find that value.

\beginlisp
(define strategy-random
  (lambda ()
    (define helper                             ; internal helper procedure
      (lambda (n)                              ; argument is no. of trials
        (if (=-hidden (+ low-limit
                         (random (+ (- high-limit low-limit)
                                 1))))         ; to get random number in range
                                               ; have found right number
            n                                  ; return no. of trials
            (helper (+ n 1)))))                ; else do again with new guess
    (helper 1)))                               ; start off with first try
\endlisp


To load these procedures, and
to set up a file in which you can put your solutions to this problem
set, follow the directions for ``loading problem sets'' in the System
Manual, to load the code for Problem Set 2. 

To test out the procedures, evaluate {\tt setup}  applied to some simple
arguments, e.g. {\tt (setup 1 10)}, to initialize
the global variables, including the hidden number.  Then evaluate
{\tt (strategy-random)} to count the number of guesses needed to find the hidden number.

{\bf Exercise 1.}  Is {\tt helper}, as defined above, a recursive
procedure?  Clearly state why or why not.  Does it give rise to a
recursive or an iterative process?  Again, clearly state why.


{\bf Exercise 2.}
After playing  with his procedure for awhile, Louis decides that random
guessing is not very useful, especially when the radio station announces
that the upper and lower limits on their hidden value are $10000$ and
$1$ respectively.  Instead, he decides to be a bit more reasoned about
his approach.  In particular, he decides that a better way to guess the
hidden number is to start at the given lower limit, and increment by one
until the correct value is reached.

Write a procedure, called {\tt strategy-increment}, 
that starts at {\tt
low-limit} and increments by 1 until it reaches the hiddeen integer, at
which point it returns the number of guesses it made 
in order to find that value.

%{\bf following code is one solution, will go away in pset}
%
%\beginlisp
%(define strategy-increment
%  (lambda ()
%    (define helper 
%      (lambda (n g)
%        (if (= g hidden)
%            n
%            (helper (+ n 1) (+ g 1)))))
%    (helper 1 low-limit)))
%\endlisp

To test out your procedure, evaluate {\tt setup} applied to some simple arguments
to initialize
the global variables, including the hidden number, and then evaluate
{\tt strategy-increment} applied to whatever arguments are appropriate to
count the number of guesses needed to find the hidden number.


{\bf Exercise 3:}  What is the order of growth of {\tt
strategy-increment} measured in terms of the range of possible values
for the hidden number, $[\hbox{\rm high-limit} - \hbox{\rm lower-limit}
+ 1]$.  Please justify your answer, clearly and concisely.

{\bf Exercise 4:}
As Louis is experimenting with his code, Alyssa P. Hacker wanders by.
She looks over his shoulder, and suggests that his approach can be
improved upon considerably, using the idea of binary search.  In
particular, she suggests that a better way of narrowing down the range
of possibilities for the hidden value is to choose a number in the
middle of the range of possibilities.  After glancing at Alyssa's
favorite bed-time reading -- {\bf Struggle and Intimidation of Computer
Programming} --  Louis sets down the following rules for designing such
a procedure.


\atpar{1.} Begin by choosing a number halfway (to the nearest integer)
between the low and high limit.

\atpar{2.} If the correct number is guessed, stop, and return the number
of guesses made.

\atpar{3.} Otherwise, test to see if the guess is too high.  If it is,
then choose a new guess that is halfway between the previous guess and the
low limit given by the radio station.  If it is too low, then choose a new
guess that is halfway between the previous guess and the high limit
given by the radio station.

\atpar{4.} Continue at 2 until the correct number is found.

Write a procedure called {\tt strategy-bin-1}, which implements the
above rules.  Since we want our guesses to be integers, and finding the
midpoints of ranges may involve fractions, you may use the procedure
{\tt flip}, defined below, and which has been loaded into the
system.

\beginlisp
(define flip
   (lambda (n)
     (cond ((integer? n) n)
           ((= (random 2) 1) (floor n))
           (else (ceiling n)))))
\endlisp

%{\bf following code is one solution, will go away in pset}
%
%\beginlisp
%(define strategy-bin-1
%  (lambda ()
%    (define helper 
%      (lambda (n g)
%        (cond ((= g hidden) n)
%              ((< g hidden)
%               (helper (+ n 1)
%                       (flip (/ (+ g high-limit) 2))))
%              (else
%               (helper (+ n 1)
%                       (flip (/ (+ g low-limit) 2)))))))
%    (helper 1 (flip (/ (+ low-limit high-limit) 2)))))
%\endlisp

{\bf Exercise 5.}  Louis experiments with his new strategy by evaluating
{\tt setup} applied to some arguments, and then evaluating his procedure, but
notices
that for some hidden values, his procedure seems to be taking forever to
return a value, and that he has to abort the evaluation of his procedure
application in those case.  He asks Alyssa about it.  She looks at his
rules, given above, and the procedure he has generated and exclaims,
``Of course, you silly nit, you've set up a procedure that loops
infinitely.''  To show this, she suggests that Louis examine the case in
which the hidden value is 6, his initial guess is 5 and the limits are 1
and 10.  Using this example and the rules Louis listed, explain why Alyssa is
correct. 

{\bf Exercise 6.}  Alyssa explains to Louis that the correct way to use
binary search is to keep track of the highest number known to be smaller
than the hidden value (called the {\bf glb} for greatest lower bound)
and the lowest number known to be larger than the hidden value (called
the {\bf lub} for least upper bound).  In this way, if the
current guess is too small, one can take the average (using {\tt flip}
to get an integer)
of the guess and the current value for the lub as the new guess, while
updating the lub and glb.  If the current guess is too high, one can
take the average of the guess and the current value for the glb as the
new guess, while updating the lub and the glb.  Write a procedure called
{\tt strategy-bin-2} that implements this idea.

%{\bf following code is one solution, will go away in pset}
%
%\beginlisp
%(define strategy-bin-2
%  (lambda ()
%    (define helper 
%      (lambda (n g u l)
%        (cond ((= g hidden) n)
%              ((< g hidden)
%               (helper (+ n 1)
%                       (flip (/ (+ g u) 2))
%                       u
%                       g))
%              (else
%               (helper (+ n 1)
%                       (flip (/ (+ g l) 2))
%                       g
%                       l)))))
%    (helper 1 (flip (/ (+ low-limit high-limit) 2)) high-limit low-limit))) 
%\endlisp


{\bf Exercise 7:}  What is the order of growth of {\tt
strategy-bin-2} measured in terms of the range of possible values
for the hidden number, $[\hbox{\rm high-limit} - \hbox{\rm lower-limit}
+1]$.  Please justify your answer, clearly and concisely.


{\bf Exercise 8:}  Louis decides that since he is trying out so many
different strategies that it would be helpful to capture the commonality
of the procedures in a higher order procedure, that encompassed all of
them.  He begins by writing the following fragment:

\beginlisp
(define general-strategy
  (lambda (next-guess next-lub next-glb trial guess lub glb)
    (if (=-hidden guess)
        trial
        (general-strategy .........))))
\endlisp

You need to help Louis out by completing this code fragment.  Louis'
idea is that {\tt next-guess} should be a procedure for generating the
next guess, and {\tt next-lub} and {\tt next-glb} should be
procedures for generating the new lub and glb, respectively, if those
values are actually used by the strategy. {\tt Trial} is the number of
the current guess, {\tt guess} is the value of the guess itself, and
{\tt lub} and {\tt glb} are the appropriate bounds.  Please complete the
fragment for Louis.

%{\bf one solution, remove from pset}
%
%\beginlisp
%(general-strategy next-guess next-lub next-glb
%                  (+ 1 trial) (next-guess guess lub glb)
%                  (next-lub lub guess)
%                  (next-glb glb guess))
%\endlisp

Once you have completed {\tt general-strategy}, show how you would
rewrite {\tt strategy-random}, {\tt strategy-increment} and {\tt
strategy-bin-2} in terms of {\tt general-strategy}.

{\bf Exercise 9:}  Finally, Louis is ready to test out the different
procedures, in order to find the best one.  His idea is to have a
procedure that will run $n$ trials of a strategy and return information
about the number of guesses needed to find the correct solution, such as
the average number of guesses or the maximum number of guesses.  He writes the 
following code fragment:   

\beginlisp
(define tester
  (lambda (no-of-trials strategy type)
    (define final-stats (make-statistic type))
    (define updater (make-updater type))
    (define run-it
      (lambda (n stat)
        (if (= n no-of-trials)
            \fbox{\hbox{1}}
            (sequence (setup 1 1000)
                      \fbox{\hbox{2}}))))
    (run-it 0 0)))
\endlisp


Louis idea is that {\tt make-updater} is to be a procedure that returns
a procedure for updating the statistic to be computed.  If we want to
compute the average number of guesses, then this procedure should take
as argument the {\tt strategy} and the current {\tt stat} (in this case
the sum of the number of guesses taken on all the previous trials) and
return the sum of that previous sum and the number of guesses required
on the current application of the strategy.  If we want to compute the
maximum number of guesses taken on a trial, then this procedure should
take as argument the {\tt strategy} and the current {\tt stat} (in this
case the maximum number of guesses used on any of the previous trials)
and return the new maximum after another application of the {\tt
strategy}.  Similarly, {\tt make-statistic} is to be a procedure that
returns a procedure for computing the final statistic.  If we want to
compute the average number of guesses, this procedure should take as
arguments the total number of guesses over all the trials and the number
of trials, and return the average.  If we want to compute the maximum
number of guesses, this procedure should return that value.

You need to help poor Louis out.   In particular, you need to fill in
the appropriate code for \fbox{\hbox{1}} and \fbox{\hbox{2}}.  Note that
a version of this code fragment has been loaded into your problem set
file. You also need to write the procedures {\tt make-updater} and {\tt
make-statistic}.

Since both {\tt make-updater} and {\tt make-statistic} are to return
different procedures, based on the type of statistic we want to collect,
we need a convention for specifying which type.  Louis decides to use
numbers to tell the procedures which type he wants.  You may do the same
thing, although later in the term we will see much cleaner methods for
specifying arguments such as which type to use.

Turn in copies of your code.  Also, using {\tt tester}, fill out a
small chart, in which
you indicate the average and maximum number of guesses observed for 10
trials , using {\tt strategy-increment} and {\tt strategy-bin-2}.
Finally, 
answer the following questions. 

Does {\tt run-it} give rise to an iterative or a recursive process?
Why?

\end


\beginlisp
(define tester
  (lambda (no-of-trials strategy type)
    (define final-stats (make-statistic type))
    (define updater (make-updater type))
    (define run-it
      (lambda (n stat)
        (if (= n no-of-trials)
            (final-stats stat n)
            (sequence (setup 1 1000)
                      (run-it (+ 1 n) (updater stat strategy))))))
    (run-it 0 0)))
\pbrk
(define make-statistic 
  (lambda (type)
    (cond ((= type 1)
           (lambda (stat strat)
             (+ stat (strat))))
          ((= type 2)
           (lambda (stat strat)
             (max stat (strat)))))))
\pbrk
(define make-updater
  (lambda (type)
    (cond ((= type 1)
           (lambda (stat trials)
             (/ stat trials)))
          ((= type 2)
           (lambda (stat trials)
             stat)))))
\endlisp

