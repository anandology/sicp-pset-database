% -*- LaTex -*-
% Copyright (c) 1990 Massachusetts Institute of Technology
% 
% This material was developed by the Scheme project at the Massachusetts
% Institute of Technology, Department of Electrical Engineering and
% Computer Science.  Permission to copy this material, to redistribute
% it, and to use it for any non-commercial purpose is granted, subject
% to the following restrictions and understandings.
% 
% 1. Any copy made of this material must include this copyright notice
% in full.
% 
% 2. Users of this material agree to make their best efforts (a) to
% return to the MIT Scheme project any improvements or extensions that
% they make, so that these may be included in future releases; and (b)
% to inform MIT of noteworthy uses of this material.
% 
% 3. All materials developed as a consequence of the use of this
% material shall duly acknowledge such use, in accordance with the usual
% standards of acknowledging credit in academic research.
% 
% 4. MIT has made no warrantee or representation that this material
% (including the operation of software contained therein) will be
% error-free, and MIT is under no obligation to provide any services, by
% way of maintenance, update, or otherwise.
% 
% 5. In conjunction with products arising from the use of this material,
% there shall be no use of the name of the Massachusetts Institute of
% Technology nor of any adaptation thereof in any advertising,
% promotional, or sales literature without prior written consent from
% MIT in each case. 

% This is based on the Spring Semester, 1985 version

\input ../formatting/6001mac.tex

\evensidemargin 35pt

\begin{document}

\psetheader{Spring Semester, 1985}{Problem set 2}

\medskip

\begin{tabular}{ll}
Issued: & Tuesday, February 12 \\
Due: & in recitation on Friday, February 22 {\em for all sections} \\
Reading: & Text, Chapter 1, Sections 1.2 and 1.3 
\end{tabular}

\section{Homework exercises}

Write up and turn in the following exercises from the text:

\begin{itemize}
\item Exercise 1.25: Products
\item Exercise 1.26: Accumulate
\item Exercise 1.26: Filters
\item Exercise 1.28: {\tt (f f)}
\item Exercise 1.32: Repeated application
\item Exercise 1.33: Repeated smoothing
\end{itemize}

\section{Laboratory Assignment: Testing for Primality}

This laboratory assignment deals with the orders of growth in the
number of steps required by some algorithms that test whether a given
number is a prime.  It is based upon Section 1.2.6 of the text.  You
should read that material before beginning work on this assignment.

\subsection{To Do at the Laboratory}

All the procedures from section 1.2.6 are installed on the
Chipmunk shared resource manager, and can be loaded onto your floppy
disk.  To do this, follow the instructions given in the Chipmunk
manual in the section on ``Loading Problem Set Files,'' to load the
code for problem set 2.  (The code for this problem set is
sufficiently short that we have not separated out the procedures that
you are expected to modify, as explained in the Chipmunk manual.)

There are slight differences between the {\tt timed-prime-test} and
{\tt fermat-test} procedures we have given you and the ones in the book.
{\tt timed-prime-test} uses {\tt princ} instead of {\tt print} to avoid
starting a new line until it gets to the next number being tested.
{\tt fermat-test} uses {\tt big-random} instead of {\tt random} because we
will be testing some very big numbers for primality, and the
{\tt random} procedure built in to Scheme cannot generate extremely
large random numbers.  {\tt big-random} is defined as

\beginlisp
(define (big-random n)
  (random (min n (expt 10 10))))
\endlisp

This procedure can be called with arbitrarily large numbers, but will
only generate numbers less than or equal to $10^{10}$.

\labpart Do exercise 1.16 from the text.

\labpart Define the following procedure, which, when called
with an odd integer $n$, tests the primality of consecutive odd integers
starting with $n$.  (Obviously, there is no point checking if even
integers are prime.)  The procedure will keep running until you
interrupt it by typing ctrl-G.

\beginlisp
(define (search-for-primes n)
  (timed-prime-test n)
  (search-for-primes (+ n 2)))
\endlisp

Transfer your new procedure to Scheme and test it.  When you are
satisfied that it is working, use {\tt search-for-primes} to find the 3
smallest primes that are larger than larger than 1,000; larger
than 10,000; larger than 100,000; larger than 1,000,000.

Prepare a chart as follows:

\begin{center}
Range: 1000

\begin{tabular}{|r|c|c|c|}  \hline
Primes: & \fillin{prime1} & \fillin{prime2} & \fillin{prime3} \\ \hline
Time1:  &                 &                 &                 \\ \hline
Time2:  &                 &                 &                 \\ \hline
Time3:  &                 &                 &                 \\ \hline
\end{tabular}

\end{center}

and make similar charts for the ranges 10,000, 100,000, and 1,000,000.
In the spaces marked \fillin{prime$n$}, fill in the three smallest primes you
found.  (You will fill in 12 primes in all, 3 in the chart for each
range.)  Under each prime, in the row labelled Time1, fill in the
time used by the prime testing procedure to determine that the number
was prime.  (The Scheme primitive procedure {\tt runtime} used in
{\tt timed-prime-test} returns the amount of time in hundredths of a
second that the system has been running.)\footnote{Be careful here.
{\tt runtime} counts not only actual compute time, but also garbage
collection time.  Garbage collection is a process that Scheme goes
through as part of its memory management.  (We will discuss this near
the end of the term.)  When Scheme is garbage-collecting, the letter G
appears at the lower right-hand corner of the screen.  (Ordinarily,
the Greek letter pi is shown there.)  If a garbage collection happens
during one of the computations whose time you care about, the
{\tt runtime} number will be too large.  If you see a garbage collection
happening during one of the computations for a number in your chart,
it's best to retime the computation for that number.}

\labpart Modify {\tt smallest-divisor} as described in exercise
1.18 in the text.  With {\tt timed-prime-test} using this modified
version of {\tt smallest-divisor}, run the test for each of the 12
primes listed in your table.  Enter the times required by the tests in
your chart in the rows labelled Time2.

\labpart Back in the editor, modify {\tt timed-prime-test} to use
{\tt fast-prime?} in place of {\tt prime?}.  You can assume that a number
is prime if it passes the Fermat test two times.  Now test each of
your 12 primes as in Part 3 above and enter the required times in the
spaces marked Time3.

\labpart Now let's test some really big numbers for primality.
In 1644, the French mathematician Marin Mersenne published the claim
that numbers of the form $2^{p}-1$ are prime for the following
values of $p$ and for no other $p$ less than 257.

\begin{center}
2, 3, 5, 7, 13, 17, 19, 31, 67, 127, 257
\end{center}

It turns out that Mersenne missed a few values of $p$, and that some of
the values in his list do not give primes.  Use {\tt timed-prime-test}
(modified as in Part 4 to use {\tt fast-prime?}) to determine which of
the values in Mersenne's list give primes, and which do not.  (Prime
numbers of the form $2^{p}-1$ are known as {\em Mersenne primes}.)
Record the time required for each test.  To aid in testing, it will
help to define the following procedure:

\beginlisp
(define (mersenne p) (- (expt 2 p) 1))
\endlisp

\labpart Let's try to find which values of $p$ Mersenne missed.
A straightforward way to do this is to write an iterative procedure
that checks, for all integers $p$ in a given range, whether {\tt
(mersenne $p$)} is prime.  If we blindly test all integers in a given
range, however, then we will be doing a lot of needless checking,
because $2^{p}-1$ cannot be prime unless $p$ is prime.  (Sketch of
proof: Show that $2^{ab}-1$ is divisible by $2^{a}-1$.)  Write a
procedure {\tt mersenne-range} that works as follows: For each $p$ in
a given range of integers, the program should first use {\tt prime?}
to test whether p is prime.\footnote{If you like, you can use {\tt
fast-prime?} here.  But the numbers $p$ will be rather small so there
is not much advantage.  Also, {\tt fast-prime?} with only a couple of
iterations is unreliable for very small numbers.  Also be careful with
$p = 2$, for which {\tt fast-prime?} does not work.} If so, it should
use {\tt fast-prime?} to test whether {\tt (mersenne $p$)} is prime.
As the program runs, it should print $p$, {\tt (mersenne $p$)}, and
the result of the prime test.  Use {\tt mersenne-range} to find all
values of $p$ less than or equal to 257 for which $2^{p}-1$ is prime.

\labpart Generalize {\tt mersenne-range} to a procedure called
{\tt prime-filter-check}, which takes as input two integers {\tt a} and
{\tt b}, a predicate called {\tt filter}, and a procedure of one argument
called {\tt term}.  For each integer $n$ in the range from {\tt a} to
{\tt b} for which {\tt (filter $n$)} is true, the program should check (using
{\tt fast-prime?}) whether {\tt (term $n$)} is prime.  As a test, calling the
procedure with {\tt filter} as {\tt prime?} and {\tt term} as {\tt mersenne}
should do the same thing as the {\tt mersenne-range} procedure of Part
6.

\labpart It is a curious fact that numbers of the form $n^2 + n + 41$
are prime for small values of $n$.  Use the procedure
you wrote in Part 7 to answer the following questions: 

\begin{enumerate}

\item What are the ten smallest positive integers $n$ for which $n^2 + n + 41$
is not prime? 

\item What are the ten smallest {\em primes} $p$ for which $n^2 + n + 41$
is not prime?
\end{enumerate}

\subsection{Post-lab Write-up}

After you are through at the lab, you are to write up and
hand in answers to the following questions:

\begin{enumerate}

\item What is the smallest divisor of each of the numbers 
that you tested in Part 1?

\item Prepare a neat copy of the table you made, listing the 3 primes in
each of the 4 ranges and the corresponding timings for each of the
three primality tests.

\item The number of steps in the first prime test should grow as fast as
the square root of the number being tested.  We should expect
therefore that testing for primes around 10,000 should take about
$\sqrt{10}$ times as long as testing for primes around 1000.  Does your
timing data for Time1 bear this out?  How well does the data for 100,000
and 1,000,000 support the $\sqrt{n}$ prediction?

\item The modification you made in Part 3 was designed to speed up the
prime test by halving the number of test steps.  So you should expect
it to run about twice as fast.  Does your data bear this out?  For
each of the primes in your table, compute the ratio Time1/Time2.
Does this ratio appear to be constant (i.e., more or less independent
of the number being tested)?  Does the ratio indicate that the new
test runs twice as fast as the old?  If not, what is the actual ratio
and how do you explain the fact that it is different from 2?

\item The number of steps required by the {\tt fast-prime?} test has $O(\log n)$
growth.  How then would you expect the time to test primes near
100,000 to compare with the time needed to test primes near 1000?  How
well does your data bear this out?  Can you explain any discrepancy
you find?

\item Which of the numbers in Mersenne's list do in fact yield primes?
What numbers did Mersenne miss?

\item In Part 5, how long did your program take to test primality of
$2^{127}-1$?  Estimate how long the first prime testing algorithm (Part
1) would require to check the primality of this number?  To answer
this question, extrapolate from the data you collected in the Time1
entries of your table, together with the fact that the number of steps
required grows as $\sqrt{n}$.  Give the answer, not in seconds, but
in whatever unit seems to most appropriately express the amount of
time required (e.g., minutes, hours, days, ...).  Be sure to explain
how you arrived at your estimate.

\item Turn in listings of the {\tt mersenne-range} and
{\tt prime-filter-check} procedures that you wrote in Parts 6 and 7.

\item What are the answers you obtained in Part 8?  How did you use the
procedure in Part 7 to find these answers?

\item Do Exercise 1.20 from the text.

\item Do Exercise 1.21 from the text.
\end{enumerate}

\end{document}
