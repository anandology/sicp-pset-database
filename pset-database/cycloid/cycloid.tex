% Copyright (c) 1990 Massachusetts Institute of Technology
% 
% This material was developed by the Scheme project at the Massachusetts
% Institute of Technology, Department of Electrical Engineering and
% Computer Science.  Permission to copy this material, to redistribute
% it, and to use it for any non-commercial purpose is granted, subject
% to the following restrictions and understandings.
% 
% 1. Any copy made of this material must include this copyright notice
% in full.
% 
% 2. Users of this material agree to make their best efforts (a) to
% return to the MIT Scheme project any improvements or extensions that
% they make, so that these may be included in future releases; and (b)
% to inform MIT of noteworthy uses of this material.
% 
% 3. All materials developed as a consequence of the use of this
% material shall duly acknowledge such use, in accordance with the usual
% standards of acknowledging credit in academic research.
% 
% 4. MIT has made no warrantee or representation that this material
% (including the operation of software contained therein) will be
% error-free, and MIT is under no obligation to provide any services, by
% way of maintenance, update, or otherwise.
% 
% 5. In conjunction with products arising from the use of this material,
% there shall be no use of the name of the Massachusetts Institute of
% Technology nor of any adaptation thereof in any advertising,
% promotional, or sales literature without prior written consent from
% MIT in each case. 

\input ../formatting/6001mac

\begin{document}

\psetheader{Fall Semester, 1989}{Problem Set 1}

\medskip

\begin{flushleft}
Issued:  12 September 1989 \\
\smallskip
Due:  In recitation, 20 September 1989
\end{flushleft}

\noindent
{\small
Problem sets should always be handed in at recitation.  Late work will
not be accepted.}

\begin{flushleft}
Reading: 
\begin{tightlist}

\item ``Course Organization'': Review this if you haven't
already done so. 

\item Text: section 1.1.

\item  ``A 6.001 User's Guide to the Chipmunk System'': Chapters 1
and 2.  Bring this manual to lab with you to use as a reference.
\end{tightlist}
\end{flushleft}

Before you can begin work on this assignment, you will need to get a copy of
the textbook and to pick up the manual package.  To obtain the manual package,
follow the instructions given in the Course Organization memo.

The laboratory assignments for 6.001 have been designed with the
assumption that you will do the required reading and textbook
exercises before you come to the laboratory.  You should also read
over and think through the entire assignment before you sit down at
the computer.  It is generally much more efficient to test, debug, and
run a program that you have planned before coming into lab than to try
to do the planning ``online.''  Students who have taken 6.001 in
previous terms report that failing to prepare ahead for laboratory
assignments generally ensures that the assignments will take much
longer than necessary.


\section{Part 1: Homework exercises}

Do the following exercises before going to the lab.  Write up your
solutions and include them as part of your homework.

\paragraph{Exercise 1}
Exercise 1.1 on page 19 of the text.  You may use the computer to check your
answers, but first do the evaluation by hand so as to get practice with
understanding Scheme.  Typing expressions such as these to the interpreter and
observing its behavior is one of the best ways of learning about the language.
Experiment and try your own problems (you need not turn in your own problems).
Though the examples shown are often indented and printed on several lines for
readability, an expression may be typed on a single line or on several, and
redundant spaces and carriage returns are ignored.  It is to your advantage to
format your work so that you (and others) can easily read it.


\paragraph{Exercise 2}
For each of the following four expressions, show a sequence of
steps---using the substitution model of evaluation---that reduce the
expression to a number.  In each expression, assume that $exp_1$
and $exp_2$ are subexpressions that evaluate to 5 and 12,
respectively.  (Of course, you can't show the details of evaluating
these two subexpressions.)  Also, for each expression, tell whether
applicative-order evaluation stipulates that $exp_1$ is evaluated
before $exp_2$, or vice versa, or whether the order of evaluating
these two subexpressions is unspecified.
\beginlisp
((lambda (x) ((lambda (y) (- y x)) $exp_1$)) $exp_2$)
\null
((lambda (y) ((lambda (x) (- y x)) $exp_2$)) $exp_1$)
\null
((lambda (x y) (- y x)) $exp_2$ $exp_1$)
\null
((lambda (y) (lambda (x) (- y x)) $exp_1$) $exp_2$)
\null
(((lambda (x) (lambda (y) (- y x))) $exp_1$) $exp_2$)
\endlisp

\paragraph{Exercise 3}
Translate the following arithmetic expression into Scheme prefix notation:
\begin{displaymath}
\frac{3+2+(1-(5-(2+\frac{3}{4})))}{4 (7-3) (1-6)}
\end{displaymath}

\paragraph{Exercise 4}
Do Exercise 1.4 on page 23 of the text.


\section{Part 2: Getting started in the lab and using the debugger}

When you come to the lab, find a free computer and log in and initialize one of
your disks, as described at the beginning of chapter 1 of the Chipmunk manual.
You should also write your name and address on a label affixed to the jacket of
the floppy disk.  Floppy disks are notoriously unreliable storage media, so it
is a good idea to copy your data onto a second disk (that is used only for this
purpose) when you have completed each laboratory assignment.  See the
description of how to copy disks in the Chipmunk manual, and do not hesitate to
ask the lab assistants for help.

After you have successfully logged in, load the material for problem set 1
using the method described in chapter 1 of the Chipmunk manual---namely, press
\key{EXTEND} (i.e., the key marked \kkey{2}) then type {\tt load problem set}
and press \key{ENTER}.  When the system asks for the problem set number, type
{\tt 1} and press \key{ENTER}.  The system will load some files and will leave
you connected to the Scheme interaction buffer, where Scheme prompts you for an
expression to evaluate.

\subsection{Evaluating expressions} 

To get used to typing at the Chipmunks, check your answers to exercise
1.1.  After you type each expression, press the \key{EXECUTE} key (at
the lower right side of the keyboard) to evaluate the expression and
see the result.  Notice that when you type a right parenthesis, the
cursor briefly highlights the matching left parenthesis.  You can
correct typing errors with the \key{BACKSPACE} key.\footnote{Edwin is
a version of Emacs, the editor used on Athena workstations.  It has
many powerful editing features, which you can read about in chapter 2
of the Chipmunk manual.  Although simple cursor motion will be your
primary editing tool at first, it will be greatly to your advantage to
learn to use some of the more sophisticated editing commands as the
semester progresses.}

To type expressions that go over a single line, use the \key{ENTER}
key (not the \key{EXECUTE} key) to move to the next line.  Notice that
the editor automatically ``pretty prints'' your procedure as you type
it in, indenting lines to the position determined by the number of
unclosed left parentheses.

\subsection{Using the debugger}

When you program, you will certainly make errors, both simple typographical
typographical mistakes and more significant conceptual bugs.  Even expert
programmers produce code that has errors; superior programmers make use of
debuggers to identify and correct the errors efficiently.  At some point over
the next month\footnote{Don't do it now---you have enough to do now.}, you
should read chapter 3 of the Chipmunk manual.  This chapter describes Scheme's
debugging features in detail.  For now, we've provided a simple exercise to
acquaint you with the debugger.

Loading the code for problem set 1, which you did above, defined three
tiny procedures called {\tt p1, p2} and {\tt p3}.  We won't show you
the text of these procedures---the only point of running them is to
illustrate the debugger.

Evaluate the expression {\tt (p1 1 2)}.  This should signal an error.
The screen splits into two windows, with the ordinary interaction
buffer at the top, and a {\tt *Scheme Error*} window at the bottom.
The error window contains the message
\beginlisp
Wrong Number of Arguments 1
within procedure \#[COMPOUND-PROCEDURE P2]
\endlisp
At the bottom of the screen is a question asking whether or not you
want to debug the error.

Don't panic.  Beginners have a tendency, when they encounter an error,
to immediately respond ``No'' to the offer to debug, sometimes without
even reading the error message.  Let's instead see how Scheme can be
coaxed into producing some helpful information about the error.

First of all, there is the error message itself.  It says that the
error was caused by a procedure being called with 1 argument, which is
the wrong number of arguments for that procedure.  The next line tells
you that the error occurred within the procedure {\tt P2}, so the
actual error was that {\tt P2} was called with 1 argument, which is
the wrong number of arguments for {\tt P2}.

Unfortunately, the error message alone doesn't say where in the code
the error occurred.  In order to find out more, you need to use the
debugger.  The debugger allows you to grovel around, examining pieces
of the execution in progress in order to learn more about what may
have caused the error.

Type {\tt y} in response to the system's question.
Scheme should respond:
\beginlisp
Subproblem Level: 0  Reduction Number: 0
Expression:
(P2 B)
within the procedure P3
applied to (1 2)
\endlisp

This says that the expression that caused the error was {\tt (P2 B)},
within the procedure {\tt P3}, which was called with arguments 1 and
2.  Ignore the stuff about subproblem and reduction numbers.  You can
read about them in chapter 4 of the Chipmunk manual.

The debugger differs from an ordinary Scheme command level, in that you use
single-keystroke commands, rather than typing expressions and pressing
\key{EXECUTE}.  One thing you can do is move ``up'' in the evaluation sequence
to see how the program reached the point that signaled the error.  To do this,
type the character {\tt u}.  The debugger should show:
\beginlisp
Subproblem level: 1 Reduction number: 0
Expression:
(+ (P2 A) (P2 B))
within the procedure P3
applied to (1 2)
\endlisp
Remember that the expression evaluated to cause the error was {\tt (P2
B)}.  Now that we have moved ``up,'' we learn that this expression was
being evaluated as a subproblem of evaluating the expression {\tt (+
(P2 A) (P2 B))} still within procedure {\tt P3} applied to 1 and 2.

So we've learned from this that the bug is in {\tt P3} where the
expression {\tt (+ (P2 A) (P2 B))} calls {\tt P2} with the wrong
number of arguments.  At this point, one would normally quit the
debugger and edit {\tt P3} to correct the bug.

Before leaving the debugger, let's explore a little more.  Press {\tt
u} again, and you should see
\beginlisp
Subproblem Level: 2  Reduction Number: 0
Expression:
(+ (P2 X Y) (P3 X Y))
within the procedure P1
applied to (1 2)
\endlisp
which tells us that the program reached the place we just saw above as
a result of trying to evaluate an expression in {\tt P1}.

Press {\tt u} again and you should see some mysterious stuff.  What
you are looking at is some of the guts of the Scheme system---the part
shown here is a piece of the interpreter's read-eval-print loop.
In general, backing up from any error will eventually land you in the
guts of the system.  At this point you should stop backing up unless,
of course, you want to explore what the system looks like.  (Yes:
almost all of the system is itself a Scheme program.)

In the debugger, the opposite of {\tt u} is {\tt d}, which moves you
``forward.'' Go forward until the point of the actual error,
which is as far as you can go.

Besides {\tt b} and {\tt f}, there are about a dozen debugger single-character
commands that perform operations at various levels of obscurity.  You can see a
list of them by typing {\tt ?} at the debugger.  For more information, see the
Chipmunk manual.

Type {\tt q} to quit the debugger and return to ordinary Scheme.  (If you are
left in the *Help* buffer, just press the \key{scheme} key (labeled \kkey{7})
to return to the Scheme interaction buffer.)

\section{3. Programming Assignment: \\ Graphing Epicycloids and Hypocycloids}

Now that you've gained some experience with the Chipmunks, you should be ready
to work on the programming assignment.  When you are finished in the lab, you
should write up and hand in the numbered problems below.  You may want to
include listings and/or pictures in your write-up.  Chapter 1 of the Chipmunk
manual explains how to use the lab printers.

In this assignment, you are to experiment with simple procedures that plot
curves called epicycloids and hypocycloids.  These are defined formally as
follows.

Consider a circle of radius {\tt b} and a disk of radius {\tt a} rolling on the
circumference of the circle.  Consider a point {\tt p} that rotates with the
disk at a distance {\tt a1} from the center of the disk.  The curve traced by
this point as the disk rolls around the circle is called an {\it epicycloid} if
the disk is rolling around the outside of the circle, and a {\it hypocycloid}
if the disk is rolling around the inside of the circle.

\begin{figure}
\vspace{4 in}
\caption{Constructing epicycloids and hypocycloids}
\end{figure}

If the equation of the circle with center point at $(0, 0)$ is

\beginlisp
x$^2$ + y$^2$ = b$^2$
\endlisp
then the locus of the point {\tt p=(x,y)} can be described for the
epicycloid as follows:

\[
\begin{array}{lll}
x(t) & = & (a+b) sin (\frac{a}{b} t)  -  a_1 sin (\frac{a+b}{b} t) \\
y(t) & = & (a+b) cos (\frac{a}{b} t)  -  a_1 cos (\frac{a+b}{b} t)
\end{array}
\]

The corresponding equations for the hypocycloid are

\[
\begin{array}{lll}
x(t) & = & (b-a) sin (\frac{a}{b} t)  -  a_1 sin (\frac{b-a}{b} t) \\
y(t) & = & (b-a) cos (\frac{a}{b} t)  -  a_1 cos (\frac{b-a}{b} t)
\end{array}
\]

We can graph epicycloids on the Chipmunk using the procedure
below.  The arguments to the procedure are the radii
{\tt a} and {\tt b} of the disk and the circle, the distance {\tt a1} from
the center of the disk
to the plotted point, and a parameter {\tt dt} that determines the
increment at which points are plotted.

\begin{minipage}[t]{\linewidth}
\beginlisp
(define (epicycloid a b a1 dt)
  (define (horiz t)
    (- (* (+ a b) (sin (/ (* a t) b)))
       (* a1 (sin (/ (* (+ a b) t) b)))))
  (define (vert t)
    (- (* (+ a b) (cos (/ (* a t) b)))
       (* a1 (cos (/ (* (+ a b) t) b)))))
  (define (iter t)
    (draw-line-to (horiz t) (vert t))
    (iter (+ t dt)))
  (iter 0))                            ; start drawing
\endlisp
\end{minipage}

The {\tt epicycloid} procedure contains internally defined
procedures that compute the horizontal and vertical coordinates
corresponding to a given value of {\tt t}, and a procedure {\tt iter}
that repeatedly draws lines connecting the points for
consecutive values of {\tt t}.
The primitive {\tt draw-line-to}, used by {\tt iter}, moves the
graphics pen to the indicated {\tt x} and {\tt y} coordinates, drawing a line
from the current point.


\subsection{Trying the program}

Start a new file on your disk to hold the code for your program by
pressing {\key{FIND FILE} (\shift \kkey{0})} and enter a name for this
file: {\tt ps1.scm}.  You can use some other name besides {\tt ps1} if
you like, but you should end the name in {\tt .scm}.  This tells the
editor that the file contains Scheme code.  Edwin will inform you that
this is a new file, and will create an empty edit buffer for it.
Subsequent laboratory assignments will include large amounts of code,
which will be loaded automatically into an edit buffer when you begin
work on the assignment.  This time, however, we are asking you to type
the code yourself, to give you practice with the editor.

Type in the definitions of the procedure {\tt epicycloid}.  Use the
\key{ENTER} key to go from one line to the next, so that Edwin will
automatically indent the code.  If Edwin's indentation does not match
what you expected, you have probably omitted a parenthesis---observing
Edwin's indentation is an excellent way to catch parenthesis errors.
It's also a good idea to leave a blank line between procedure
definitions.

``Zap'' your procedures from the editor into Scheme, using the
\key{EVALUATE BUFFER} key ({\shift \kkey{7}}).\footnote{\key{EVALUATE
BUFFER} transmits the entire buffer to Scheme, and is the easiest thing to use
when you only have one or two procedures in your buffer.  With larger amounts
of code, it is better to use other ``zap'' commands, which transmit only
selected parts of the buffer, such as individual procedure definitions.  For a
description of these commands, see Chapter 3 of the Chipmunk manual under the
heading ``Scheme Interaction.''}.  If you get an error at this point, you
probably have a badly formed procedure definition.  Ask for help.  Otherwise,
test your program by evaluating
\beginlisp
==> (clear-graphics)
\null
==> (epicycloid 40 35 60 0.1)
\endlisp
to draw an epicycloid curve.

\begin{figure}
\vspace{4 in}
\caption{{\tt (epicycloid 40 35 60 0.1)}}
\end{figure}

To see the drawing, press the key marked \key{GRAPHICS} at the upper
right of the keyboard.  The Chipmunk system enables you to view either
text or graphics on the screen, or to see both at once.  This is
controlled by the keys marked \key{GRAPHICS} and \key{ALPHA}.
Pressing \key{GRAPHICS} shows the graphics screen superimposed on the
text.  Pressing \key{GRAPHICS} again hides the text screen and shows
only graphics.  \key{ALPHA} works similarly, showing the text screen
and hiding the graphics screen.

Try drawing some epicycloid curves with various parameters.  Use
{\tt clear-graphics} to clear the screen and {\tt position-pen} to change
the initial position from which the curve is drawn.

To save your work on your disk, press {\key{SAVE FILE} (\kkey{0})}.
It's a good idea to save your work periodically, to protect yourself
against system crashes.


If you typed the program incorrectly, you will need to debug the
procedure and return to the editor to make corrections.

Observe that the {\tt epicycloid} procedure does not terminate.  It
will keep running until you stop execution by typing \ctrl\key{g} or
\key{ABORT}.  (The \key{CTRL} key at the upper left of the keyboard is
used like a \key{SHIFT} key.  To type \ctrl\key{g}, hold down the
\key{CTRL} key and press \key{g}.)

\subsection{Exercise 5}

Define a procedure {\tt radius-scale} that takes as argument a number
{\tt m}, and calls {\tt epicycloid} with ${\tt a}=30$, ${\tt b}=20$,
${\tt a1}={\tt m}\times 60$, and ${\tt dt}=0.1$.

Explore the family of figures generated by {\tt radius-scale} as {\tt m}
varies smoothly from 0 to 1. Turn in two or three sketches showing how
the figures change, together with a listing of the procedure
{\tt radius-scale}.

\subsection{Exercise 6}

A hypocycloid is similar in definition to an
epicycloid.  Using the editor, make a copy of your code for
{\tt epicycloid}, and modify it to implement a definition for
{\tt hypocycloid}.  Turn in a listing of the procedure.

\subsection{Exercise 7}

Investigate the family of hypocycloids generated with ${\tt a}=60$,
${\tt b}=30$, ${\tt dt}=0.2$, and
{\tt a1} varying from 0 to 70. All these figures should have 3-fold
symmetry. Turn in sketches of some of the figures to indicate how
the shape varies as {\tt a1} changes.

\subsection{Exercise 8}

Investigate the hypocycloids that are generated with ${\tt a}=60$,
${\tt a1}=60$, ${\tt dt}=0.2$, and {\tt b} taking on values that are integer
multiples of 5 that are greater than 0 and less than 60.  Pay
particular attention to the symmetry of the figures.  (For example,
${\tt b}=30$ produces 3-fold symmetry and {\tt b}=40 produces 4-fold
symmetry.)  Find a value of {\tt b} that produces a figure with 5-fold
symmetry; with 11-fold symmetry.  What kind of symmetry is produced by
${\tt b}=35$?  Turn in a sketch of one of the more interesting figures
you discover.

\subsection{Exercise 9}

\newcommand{\twopib}{$2\pi\times b$}

If we restrict {\tt a} and {\tt b} to be integers, then the epicycloid
will be completely drawn by the time that {\tt t} reaches \twopib.
Explain why.  (In general, waiting until {\tt t} reaches \twopib\  may
cause the figure to be traced more than once.  The number of times
that the figure will be retraced can be specified in terms of {\tt a}
and {\tt b}.  You might think about how to do this.)  Modify the {\tt
epicycloid} program so that it stops when {\tt t} becomes greater than
\twopib.  (Let the procedure return 0 when it stops).  This requires
only a small modification to the {\tt iter} procedure.  In order to do
this conveniently, in Scheme define {\tt pi} to be 3.14159 (or \hbox{\tt
(* 4 (atan 1 1))} for a more accurate value).  Also define {\tt
2pi} to be a constant equal to twice {\tt pi}.  Turn in a listing of
your modified procedure, as well as the expressions you type to Scheme
to define {\tt pi} and {\tt 2pi}.

\subsection{Exercise 10}

({\it Design problem\/}) The {\tt epicycloid} procedure given above does a
lot of unnecessary computation.  Each time it plots a point, it
computes the sum of {\tt a} and {\tt b} four times.  In computing the
argument to {\tt sin} and {\tt cos} it multiplies {\tt t} by a constant and
then divides the result by {\tt b}, rather than just multiplying by a
single constant that could have been computed just once at the
beginning of the plot.  It takes no advantage of the fact that many of
the same quantities, such as \mbox{\tt (/ (* (+ a b) t) b)}, are used in
computing both the {\tt x} and the {\tt y} coordinates.

Redesign the procedure to eliminate as many of these redundant
computations as you can.  (You may need to define extra internal
variables or procedures.)  Implement your design as a new procedure
{\tt epicycloid1}.  Test your procedure to verify that it draws the same
figures as does the original {\tt epicycloid}.  Does your modification
improve the speed by a noticeable amount?  (Use a wristwatch, or the
clock at the lower right of the screen, to determine approximate
timings.)  Turn in a listing of {\tt epicycloid1}, together with the
expressions you ran to make timing tests, and the comparative timings
for {\tt epicycloid} and {\tt epicycloid1}.


\subsection{Finally}

Please indicate the names of any students you cooperated with in doing
this assignment.

How much time did you spend in lab on this assignment?  How much time
did you spend in total on this assignment?



\end{document}
