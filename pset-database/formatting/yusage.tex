% for yTeX
\typesize=11pt
\runninghead={How to Use \ytex}
\makelgtbrackets

% small caps font
\font\sc=amcsc10

% easy indexing of macros described
\let\+=\relax                                   % commands
\let\==\relax                                   % parameters
\let\:=\relax                                   % switches

% easy verbatim stuff
\catcode`\"=12\relax
\def\"{\hbox{\tt "}}
\def\verbq{\verb"}
\addspecial\"
\catcode`\"=\active \let"=\verbq

% macro descriptions
\newelement\gab
  \setup \unvpar
         \advance\parskip by2pt plus1pt
         \leftskip=\parindent
         \clubpenalty=10000
         \def\*{\noindent \hskip-\leftskip\ignorespaces}%
         \def\\{\linebreak\ignorespaces}%
         \def\##1{{\tt \char`\{}#1{\tt \char`\}}}%
  \above \minbreak \pregabpenalty \abovegabskip
  \below \dobreak \postgabpenalty \belowgabskip \vpar
  \begin
  \end \endgraf
  \endnew
\abovegabskip=\abovebulletsskip
\belowgabskip=\belowbulletsskip
\pregabpenalty=\prebulletspenalty
\postgabpenalty=\postbulletspenalty

% names for TeXs
\let\tex=\TeX
\def\LaTeX{L\kern-.2em\raise.3ex\hbox{a}\kern-.09em\TeX}
\let\latex=\LaTeX
\def\Tbase{\sy{TBASE}}
\let\tbase=\Tbase
\def\Plain{{\rm Plain}}
\let\plain=\Plain
\def\lisp{L{\smllrm ISP}}

\begintitlepage

\tcenter{\sc The Artificial Intelligence Laboratory\\ Massachusetts
Institute of Technology}

\vskip 1pc

\spread {Working Paper 273}{May 1985}
\spread {}{Revised, June 1986}

\vskip 6pc

\ctitle {How to Use \ytex}

\vskip 3pc

\cauthor {Daniel Brotsky\\ June 9, 1986}

\vskip 3pc

\beginabstract

\ytex---pronounced {\it why}-\tex\ or {\it oops}-\tex---is a
\tex\ macro package.  \ytex\ provides both an easy-to-use
interface for \tex\ novices and a powerful macro-creation library
for \tex\ programmers.  It is this two-tier structure that makes
\ytex\ more useful to a diverse \tex\ user community than other
macro packages such as \plain\ or \latex.

This paper contains \ytex\ instructions intended for novice
users.  It summarizes the facilities provided in \ytex\ and
concludes with a table of useful commands.

The version of \ytex\ documented here is release \fmtversion.

\endabstract

\beginbottomtext

Work on \ytex\ was supported by a desire to avoid doing real
work, like research.

\smallskip

A.I. Laboratory Working Papers are produced for internal
circulation, and may contain information that is, for example,
too preliminary or too detailed for formal publication.  It is
not intended that they should be considered papers to which
reference can be made in the literature.

\endbottomtext

\endtitlepage

\section {What is \ytex?}

\ytex---usually pronounced {\it why}-\tex\ but see below---is a
\tex\ macro package.  \ytex\ evolved from the author's experience
with other \tex\ macro packages, primarily \tbase, \plain,
\latex, and the \sy{PHW} book macros.  This experience showed
that, on the one hand, packages as complete as \latex\ or \tbase\
tend to be inflexible while, on the other hand, packages as
flexible as \plain\ tend to be incomplete.  \ytex\ ties both
hands together by aiming for inflexibility and incompleteness
simultaneously.

Like most \tex-related programs and macro packages, \ytex\ has a
``cute'' name that attempts a dual-language pun in English and
pseudo-Greek.\footnote*{The result is about as weak the premise
suggests.} The letter preceding the \tex\ in \ytex\ is not just a
`Y' but also an {\it upsilon}---the Greek antecedent for the
English letters `u' and `y'.\footnote\dag{Experienced \TeX\ math
hackers may know that \TeX\ makes a version of capital upsilon
available as $\Upsilon$, and they may wonder why the name \yTeX\
is not spelled as \lower.5ex\hbox{$\Upsilon$}\kern-.1667em\TeX.
That's because \TeX's version of $\Upsilon$ is not a Greek
version (one of which really looked like `Y'---trust the Romans
to be accurate in their borrowing) but actually the Uncial
version popular among middle-ages religious circles.  Hey, what
can I say?  Trust \TeX\ to be medieval whenever possible.} Thus
you may pronounce \ytex\ as the expected {\it why}-\tex, the
erudite {\it upsilon}-\tex, or the shorter {\it oops}-\tex.  But
when you spell it to the computer, you have to spell it
\sy{YTEX}.

\section {About This Document}

This document contains basic instructions for using \ytex.  It
describes the things you can do using the user interface that
\ytex\ always provides.  It does {\it not} explain how to use the
\ytex\ library facilities to add to and replace parts of that
interface.  Readers who are interested in such matters should
look at the \ytex\ sources.

If you are a complete \tex\ beginner you may find this document
hard to follow.  In fact, almost everyone will find parts of the
presentation hard to follow.  My advice to those who are confused
is to not worry about it: most of the time you will be confused
because you are reading about a feature you have no use for and
thus no experience with.  If you need to use something and don't
understand what is said about it here, just try the examples I
give in one of your manuscripts and see if you can figure it out
from there.  If that doesn't work, try finding someone who does
use the feature and ask them.  As a last resort, try finding a
\tex\ wizard and asking them.

If you wish a quick summary of most of the \ytex\ commands, you
should turn to the tables of commands, parameters, and switches
at the very end of this document.

\section {Some Notation and Conventions I Use Throughout}

In what follows, I try to use notation and conventions very
similar to those Knuth uses in the \tex book.  For
convenience,\footnote*{(and also for those many of us who find
the \tex book completely confusing)} I summarize most of
the conventions here.

\beginbullets

\bpar Text set in "typewriter" type is suitable for \ytex\ input.

\bpar Commands---also called {\it control sequences} or {\it
macros}---all start with a backslash.  {\it Alphabetic commands}
such as "\section" and "\it" have one or more letters after their
backslash, while {\it non-alphabetic commands} such as "\'" and
"\&" have a single non-letter after their backslash.  When you
use an alphabetic command, you must have a space or other
non-letter right after the command name (as in "{\it italics}"),
and any spaces after the command are ignored by \tex.  When you
use a non-alphabetic command, you can have anything at all after
the command name (as in the Spanish "m\'agico"), and spaces you
put there are {\it not} ignored by \tex.

\bpar Commands sometimes need {\it arguments}, such as the stuff
typed in braces in
\beginverb
\section {This is the Argument.}
\endverb
Most commands which use arguments---a process also called {\it
taking} or {\it reading} arguments---expect their arguments to be
enclosed in braces (as they were in this example).  But there are
other ways of specifying arguments, and these are detailed in
what follows.

\bpar \ytex\ and \tex\ both have the notion of {\it parameters}
which have {\it values}.  For example, the value of the "\hsize"
parameter is used as the normal length of lines in paragraphs.
To set a parameter, you give the name of the parameter, an
optional `"="' sign, and then the desired new value for the
parameter.  For example, to set the normal line length to 5
inches, you could say either of the following things:
\beginverb
\hsize=5in
\hsize 5in
\endverb

\bpar There are several kinds of parameters: <integer> parameters
hold integers (such as 3 or $-1$), <dimen> parameters hold
distances (such as "2in" or "5pt"), <glue> parameters hold
variable distances (such as "6pt plus 2pt minus 1pt"), and <toks>
parameters hold token lists (such as
"{\bigsize\bf}").\footnote\dag{There are also <muglue>
parameters, but only math wizards worry about those.} The
different kinds of parameters are used for different purposes;
for example, the "\hsize" parameter mentioned above is a <dimen>
because line length is a distance, and the "\pageno" parameter
that holds the current page number is---as expected---an
<integer> parameter.  But no matter what their type, you assign
values to parameters in the manner described above.

\bpar If you ever want to examine the value of a parameter---for
example, if you want to know what it is so you can change it
appropriately---you can get \ytex\ to show you the value by
putting "\showthe" in front of the parameter name.  For example,
if you said "\showthe\parindent" in your input file, \ytex\ would print
\beginverb
>20.0pt.
...  \showthe\parindent
\endverb
on your terminal screen when it processes your input.  It would
then pause and wait for you to type a carriage return before
continuing.

\endbullets

\section {Format of a \ytex\ Input File}

\ytex\ input is just normal \tex\ input.\footnote*{Of course,
this means \tex82!  \tbase\ input, for example, will not work in
\ytex.}  It is customary, however, to start \ytex\ input files
with a comment indicating that \ytex\ is expected to process the
file.  For example, the manuscript file for this document starts
with the line
\beginverb
% for yTeX
\endverb
so anyone who looks at the file will realize this.

The first command in every \ytex\ manuscript should be a
\="\typesize"\= command specifying the desired type size for the
text of the document.  For example, after the comment shown
above, the manuscript for this document has the line
\beginverb
\typesize=11pt
\endverb
As you can see, a "\typesize" command looks like a command for
setting one of \tex's <dimen> parameters, such as "\parindent".
However, the dimension that follows the equal sign must be one
of 10pt, 11pt, or 12pt, since these are the only sizes that
\ytex\ supports for normal text.

If you forget to put a "\typesize" command at the beginning of
your manuscript, \ytex\ will complain with the error message
\beginverb
! You never gave a \typesize command.
\endverb
Don't panic; just type a carriage return and \ytex\ will use
10~point type.  But you should add a "\typesize" command before
you run the file again.

You can't give more than one "\typesize" command in a single
manuscript.  This is because \ytex\ implements your desired size
by magnifying a 10pt document by the right amount, and \tex\ does
not like to change magnification in the middle of a run.  This
may seem like a drawback, but it's really a feature, because it
means that the line and page breaks will not change if you run a
manuscript through \ytex\ a second time using a different
typesize!  So you can proof your document in one size and run it
off in a different size without worrying about formatting
changes.

\section {How To Invoke \ytex}

To invoke \ytex\ on \sy{OZ} you give the command \sy{YTEX} to the
\sy{EXEC}.\footnote*{The commands \sy{YTEXHE} and \sy{YTEXTR}
also exist for those who prefer the Helvetica or Times Roman font
families.} \ytex\ starts up by printing a release number and
reading in any recent fixes to itself.  It then looks for two
files called \sy{YMATH.TEX} and \sy{YLOCAL.TEX} in your
connected directory (and on \sy{TEXINPUTS:} if they are not
found locally).  You can customize your \ytex\ by putting
appropriate commands in these two files.

The \sy{YMATH.TEX} files are intended for favorite mathematics
macros while the \sy{YLOCAL.TEX} macros are for macros or
parameter settings of any kind.  Of course, all kinds of \tex\
commands can be put in either of these files, but keep the
following in mind: separating math macros from others can make
the job of merging different authors' manuscripts a lot easier.

\section {Where are the Sources}

The source for \ytex\ lives in the directory \sy{KS:<TEX.YTEX>}
on \sy{OZ} and is split primarily among the files \sy{YTEX.MAC},
\sy{YBASE.MAC}, \sy{YFONTS.MAC}, and \sy{YUSER.MAC}.  The file
\sy{YLOG.MAC} contains a log of essentially all the changes made
to \ytex\ since its early stages; the first few lines of this
file declare the current version number.  The file \sy{YSITE.MAC}
contains definitions and parameter settings (``site changes'')
appropriate to the local installation.  The file \sy{YUSAGE.TEX}
contains the source for this document.

The file \sy{KS:<TEX.YTEX>YFIX.MAC} is where \ytex\ looks for
revisions made to the macros since the last version was dumped.
Finally, an archive of correspondence about \ytex\ can be found
in the file \sy{KS:<TEX.YTEX>YTEX.MAIL} on \sy{OZ}.

\section {The \ytex\ View of Documents}

\ytex\ believes that documents are made up of chunks of text
called {\it elements} which are laid out on pages.  For example,
each paragraph is an element, as is each chapter or section
heading, each figure, each footnote, and so on.  Many of the
\ytex\ commands you will use most often serve to tell \ytex\
which text belongs to which element.  For example, the command
"\footnote" tells \ytex\ that its argument is a footnote
element.

As \ytex\ encounters elements of various types in its input, it
first tries to format the content of each element in the desired
way.  For example, when it finishes reading the contents of a
paragraph, it first tries to break the contents into lines
according to the rules currently in effect for paragraph
elements.  When \ytex\ finishes reading the text of a table, it
tries to format that text according to the current rules for
table elements.

Once \ytex\ accumulates enough elements to fit on a page, it
tries to lay those elements out on a page according to the layout
rules that apply to each element.  For example, header elements
appear as a single line alone at the very top of a page, while
footnote elements appear tacked onto the bottoms of pages.

Page layout is the hardest part of \ytex's job.  This is because
the rules for laying out various elements may come into conflict.
For example, the rules for figures elements that appear in text
may say that the figure should be separated from the surrounding
text by one pica.\footnote*{There are six picas to the inch.}
But a figure may be placed just before a section heading, and the
rules for section heading elements may say that they should be
preceded by two picas.  In cases like this, \ytex\ does the best
it can, but it may not do exactly what you want.  So if you are
very picky about the page layout of a particular document, you
may have to tell \ytex\ exactly what you want at each place a
conflict like this arises.  But don't worry: usually only people
who are publishing books need to be this picky.

\section {The Five Kinds of \ytex\ Commands}

There are five kinds of \ytex\ commands:

\beginbullets

\bpar {\it Delimiters\/} are commands that demarcate the text of
particular elements: they do what is commonly known as
``typemarking.''  For example, "\section" and "\footnote" are
both delimiters.  All delimiters read arguments (in ways
explained below); it is the contents of these arguments that
become the contents of the elements they delimit.

\bpar {\it Parameters\/} are commands that affect the formatting
and layout rules for particular elements.  For example, "\hsize"
and "\parskip" are parameters that affect the width and layout of
paragraphs and pages.  Parameters don't take arguments; instead,
they have values which can be assigned in the manner described
above.

\bpar {\it Switches\/} are commands like parameters that affect
the formatting and layout of particular elements.  For example,
"\noindent" and "\offheaders" are switches.  Switches come in two
kinds---on/off switches and true/false switches---which are
described in detail later.  Switches never take arguments.

\bpar {\it Abbreviations\/} are commands that expand into common
sequences of low-level \tex\ commands.  For example, "\padline"
and "\linebreak" are abbreviations.  Some abbreviations have
arguments, and some don't.  Any abbreviations that take arguments
take them in braces in the normal \tex\ way.

\bpar {\it Descriptors\/} are commands that create new element
types with particular formatting and layout rules.  For example,
the commands "\newfloat" and "\newtitle" are descriptors that
are used to create elements which float to the top of pages or
which title things.  Descriptors take very complex arguments.

\endbullets

Normally, you use only delimiters, parameters, switches, and
abbreviations in your manuscripts.  Descriptors are for
compulsive \tex\ hackers who feel the need for new element types,
or for wizards who are preparing documents with special
requirements (such as this manual).

\section {More About Delimiter Commands}

Delimiters come in three kinds:

\beginbullets

\bpar {\it Standard\/} delimiters---such as "\section", "\footnote",
or "\spread"---take their arguments in the standard \tex\ way:
the arguments are enclosed in braces and appear immediately after
the command.  For example, this is how you give two arguments
to the "\spread" command:
\beginverb
\spread {First argument}{Second argument}
\endverb

\bpar {\it Paired\/} delimiters---such as
"\begintext"/\allowbreak"\endtext" and
"\beginabstract"/\allowbreak"\endabstract"---surround their
argument.  For example, this is how you delimit the text of an
abstract:
\beginverb
\beginabstract
This is a very short abstract
used only for expository purposes.
I wish all my abstracts were this simple.
\endabstract
\endverb

\bpar {\it Paragraph\/} delimiters---such as "\bpar" and
"\ftpar"---are kind of a cross between standard and paired
delimiters.  Their most important argument---the text of the
paragraph---is delimited by the paragraph delimiter in front and
the command "\par" (or a blank line) in back.  But they may have
additional arguments (such as tag text) which are given in braces
immediately following the paragraph delimiter.  For example, this
paragraph starts with
\beginverb
\bpar {\it Paragraph\/} delimiters---
\endverb
and ends with a blank line.  But it would look exactly the same
if it had started with
\beginverb
\ftpar{$\bullet$}  {\it Paragraph\/} delimiters---
\endverb
instead.

\endbullets

Most paired delimiters are also available in a standard form.
For example, you can use the command "\abstract" instead of
"\beginabstract" and "\endabstract", in which case you would
enclose the text of the abstract in braces and place it right
after the command.\footnote*{The only paired delimiters that are
not available in standard form are those in which the delimited
text is in ``nofill'' mode, such as "\beginverbatim" and
"\endverbatim".  This has to do with subtle differences between
the effect of the two forms, differences that you needn't care
about and which can confuse even \tex\ wizards.}  As you may have
guessed from the example, you get the standard form of paired
delimiters by removing the "begin" and "end" prefixes.

Each argument to a standard or paired delimiter is a local group;
that is, font and other parameter changes inside the element do
not have any effects outside.  So if you want the text of your
abstract to be in italics, you can just say
\beginverb
\beginabstract
\it This abstract is in italics.
\endabstract
\endverb
instead of the (also correct but) tedious
\beginverb
\beginabstract
{\it This abstract is in italics.}
\endabstract
\endverb
Similarly, you can just say
\beginverb
\section {\it An Italic Section Header}
\endverb
instead of
\beginverb
\section {{\it An Italic Section Header}}
\endverb

\section {More About Switches}

Switches come in two kinds:

\bpar {\it On/off} switches are usually used to control whether
things such as headers and footers appear on a page.
For example, whether headers appear or not is controlled by the
two commands "\onheaders" and "\offheaders".  These two commands
together are said to control the "\headers" switch.

\vpar All on/off switches have variations in a {\it yes/no} form.
For example, the commands "\yesheaders" and "\noheaders" are like
"\onheaders" and "\offheaders", except that they {\it only}
control headers on the current page, not on future pages as well.

The point of having both on/off and yes/no versions of switches
can be seen from the following: If you want none of the pages of
a letter numbered, you would put "\offfooters" at the beginning of
the letter.  But if you want numbers on the bottoms of all pages
except the first, you would put "\nofooters" at the beginning.
The effect of "\nofooters" is to turn off the "\footers" switch
for the first page {\it only}.

\bpar {\it True/false} switches are mostly used by wizards, so I
won't say much about them here.  They are much like off/on
switches except that they have a different form.  For example,
the "\vpar" switch controls whether the first paragraph after a
section heading is left-flush or not; the "\vpar" switch is
turned on with "\vpartrue" and turned off with "\vparfalse".  You
can see the effects of saying "\vpartrue" in this document.

\section {A List of the \ytex\ User Macros}

This section contains descriptions of the most useful \ytex\
macros.  The listing is broken down by subject area; for example,
all the commands relevant to footnotes can be found in one place,
and all the commands relevant to sectioning in another.  The next
section gives an alphabetized list of all the commands.

In the following descriptions, only the "begin" half of paired
delimiters is mentioned.  All the paired delimiters are available
in standard forms unless the description mentions otherwise.
Also, only the "on" form of on/off switches is mentioned.  Of
course, the "off", "yes", and "no" forms are also available.

\heading {Titling Pages}

\ytex\ provides two kinds of titling pages: {\it part pages}
which are delimited by \+"\beginpartpage"\+ and {\it title pages}
which are delimited by \+"\begintitlepage"\+.  The only
difference between the two is that title pages are always
numbered 0 while part pages have whatever number they would
normally be assigned.  The contents of titling pages are centered
by default, and titling pages have no runners unless you ask for
them.

The following commands are useful both on titling pages and
elsewhere:

\begingab

\*\+"\begintitle"\+ and \+"\beginctitle"\+\\set their arg as
titles.  "\begintitle" gives a left-flush ragged-right paragraph
and "\beginctitle" gives a ragged-center paragraph.  Inside of
titles, the commands \+"\\"\+, \+"\cr"\+, and \+"\crcr"\+ force
line breaks.

\*\+"\beginauthor"\+ and \+"\begincauthor"\+\\are like
"\begintitle" and "\beginctitle" but use \="\authorfont"\=
(default "\regsize\rm") instead of \="\titlefont"\= (default
"\bigsize\bf").

\*\+"\beginabstract"\+\\delimits abstract text which is set in
\="\abstractfont"\= (default "\smlsize\rm") and indented from
both margins by "\abstractindent" (default "0pt").

\endgab

The following command is available only on titling pages:

\begingab

\*\+"\beginbottomtext"\+\\delimits text which is set at the
bottom of the page in \="\bottomtextfont"\= (default
"\smllsize\rm").  If you use this command, the text of the page
will no longer be centered.  Instead, the delimited material
appears at the bottom of the page and material above it will
start at the top of the page.

\endgab

\heading {Chapters and Sections}

These macros produce titles at the top of chapters, sections, and
so on.  Titles are normally set as left-flush, ragged-right
paragraphs.  Inside of titles, the commands \+"\\"\+, \+"\cr"\+,
and \+"\crcr"\+ force line breaks.


\begingab

\*\+"\beginchapter"\+ <title> "\endchapter"\\starts a new
chapter.  The <title> is set in the \="\chapterfont"\= (default
"\bigsize\bf") as a left-flush ragged-right paragraph.  The title
is preceded by \="\prechapterpenalty"\= and
\="\abovechapterskip"\= and followed by \="\postchapterpenalty"\=
and \="\belowchapterskip"\=.  Most people prefer the standard
form "\chapter"~\#{<title>}.

\*\+"\beginpchapter"\+\\is like "\beginchapter" but it forces the
title to the top of a new page and turns off runners for that
page.  In two-sided matter, it forces the title to the top of a
recto page.

\*\+"\beginsection"\+\\is like "\beginchapter" but uses smaller
font and spacing.  Neither chapters nor sections do any automatic
numbering for you.  But there are no-op commands \+"\secdef"\+
and \+"\secref"\+ which, like their figure and table cousins, can
be used to help keep track of numbering.  There are commands
\+"\beginsubsection"\+ and \+"\beginsubsubsection"\+ but they are
just aliases for "\section".

\*\+"\beginheading"\+\\is like "\beginsection" but with smaller
font and spacing.

\endgab

The true/false switch \:"\centerheadingstrue"\: controls whether
chapter, section, and other titles are set as left-flush or
centered ragged-margin paragraphs.  The default is
"\centerheadingsfalse"; that is, headings are left-flush as
mentioned above.

\heading {Page Layout}

Pages have headers and footers, jointly called runners.  When
they are turned on, runners are single lines that appear
separated from the main text by the distances \="\headerdrop"\=
and \="\footerdrop"\=.\footnote*{These dimensions are not
measured baseline-to-baseline; rather, they are the appearing
space between the runners and the text.} By default, headers are
turned on and footers are turned off.

\begingab

\*\:"\onrunners"\:, \:"\onheaders"\:, and \:"\onfooters"\:\\are
switches that turn the various runners on.  Recall that the
effects of these commands are permanent; the effects of their
yes/no counterparts are confined to the current page.

\endgab

There are actually two versions of headers and footers---one for
recto (right-hand or odd-numbered) pages and one for verso
(left-hand or even-numbered) pages.  By default, all pages are
considered to be recto pages, but you can use the switch
\:"\twosidedtrue"\: to get two-sided output and "\twosidedfalse"
to go back to one-sided.

The contents of headers and footers are the values of the <toks>
parameters \="\versoleftheader"\=, \="\rectoleftheader"\=, and so
on.  By default, the \="\rectorightheader"\=, the
"\versoleftheader", and both center footers contain a boldface
page number (called a {\it folio}); these come out in roman
numerals if they are negative.  (The page number is the parameter
\="\pageno"\= which you can set.)  Also by default, the
"\rectoleftheader" is the "\firstmark".

For example, if you want your pages numbered at the bottom, you
could say
\beginverb
\offheaders     % turn off the header lines
\onfooters      % turn on the footer lines
\endverb
because the footers contain page numbers by default.  If you want
each page to have a title and a page number, you can say
\beginverb
\rectoleftheader={The Title of My Paper}
\endverb
Since this is done so often, there is a synonym
\="\runninghead"\= for "\rectoleftheader" that lets you say
\beginverb
\runninghead={The Title of My Paper}
\endverb
instead.

The default font for runners is kept in the <toks> parameter
\="\runnerfont"\=, which starts out as "\smlsize\rm".

\heading {Paragraphs}

Each paragraph is an element to \ytex.  Paragraphs are considered
to consist of a crown (the first line) and a vest (all the other
lines).  By default, the crown is indented "\parindent" and the
vest is not.  But there are a variety of paragraph delimiters
that give you paragraph elements that are formatted differently.
(Recall that paragraph delimiters start paragraphs that are
ended by "\par" or a blank line.)

\begingab

\*\+"\ivpar"\+\\delimits an {\it inverted} paragraph: the crown
is not indented but the vest is.

\*\+"\ipar"\+\\delimits an {\it indented} paragraph: both crown
and vest are indented.

\*\+"\ftpar"\+ \#{<tag text>}\\delimits a {\it flush-tagged}
paragraph: both crown and vest are indented and the first
argument (in braces) to "\ftpar" is set flush on the left margin
of the crown.  For example,
\beginverb
\ftpar {1.} First line ...\linebreak second line ...\par
\endverb
produces a paragraph that looks like this:
\begintextlines
\parindent=2em%
\ftpar{1.}First line \dots\linebreak{}second line \dots
\endtextlines

\*\+"\atpar"\+ \#{<tag text>}\\delimits an {\it adjoint-tagged}
paragraph: both crown and vest are indented and the first
argument (in braces) to "\atpar" is set so its right edge is
\="\partagsep"\= away from the left edge of the crown line.
(This is like the \plain\ \tex\ "\item" command.)  For example,
\beginverb
\atpar {a)} First line ...\linebreak second line ...\par
\endverb
produces a paragraph that looks like this:
\begintextlines
\parindent=2em%
\atpar{a)}First line \dots\linebreak{}second line \dots
\endtextlines

\*\+"\vtpar"\+ \#{<tag text>}\\delimits a {\it variably-tagged}
paragraph: the crown, vest, and tag are set just like an
adjoint-tagged paragraph but the width of the indent is not the
"\parindent" but instead is the width of the tag text.  For
example,
\beginverb
\vtpar {Keywords:} This ...\linebreak and that ...\par
\endverb
produces a paragraph that looks like this:
\begintextlines
\vtpar{Keywords:}This \dots\linebreak{}and that \dots
\endtextlines

\*\+"\bpar"\+\\delimits a {\it bullet} paragraph: a flush-tagged
paragraph whose tag is a bullet symbol.

\endgab

If bullet paragraphs appear next to each other, it is good
practice to put a \+"\beginbullets"\+ before the first one and an
\+"\endbullets"\+ after the last one.  This will add a little
space before and after so as to visually cluster the paragraphs.

If you interpose math or a figure or a title between paragraphs,
it is standard typesetting practice not to indent the paragraph
immediately following the interposed element.  For example,
the paragraphs which immediately follow section titles
in this document are not indented, nor are those which follow
figures or textual displays.

To follow this practice in your own documents, you could
use "\noindent" to start the paragraphs which follow interposed
elements.  But \ytex\ does this for you automatically.  The
command \+"\vpar"\+ effectively forces the next paragraph to
start with "\noindent", and all \ytex\ commands which interpose
material between paragraphs---such as "\section"---end with
"\vpar".  Thus you will not have to use "\noindent" after
figures and so on.

If for some reason you want a particular paragraph that follows
interposed text to be indented normally, just give the command
\+"\unvpar"\+ in front of that paragraph.  "\unvpar" cancels the
effect of "\vpar" and allows the paragraph to be indented.

\begingab

\*\:"\vpartrue"\: and "\vparfalse"\\allow and disallow the
effects of "\vpar".  If you don't want the first paragraphs of
sections and so on to be left-flush, say "\vparfalse".

\*\+"\linebreak"\+\\ends the current line of a paragraph, filling
out to the right hand margin with blank space.

\*\:"\onindent"\:\\turns the default indentation of crown lines
on.  If you want your paragraphs to normally have no indentation,
say "\offindent".  But keep in mind that this will break commands
like "\bpar" which rely on indentation.

\*\+"\beginquote"\+\\delimits a quote.  Quotes are set in the
\="\quotefont"\= (default "\smlsize\rm") with the left and right
margins narrowed by "\parindent".  Inside of quotes paragraph
indentation is turned off (with "\offindent").

\endgab

\heading {Fonts and Sizes}

To change type face in a document, you use the normal \tex\
commands "\it", "\bf", "\tt", "\sl", and "\rm".  To change to
increasingly bigger sizes of type, you can use the commands
\+"\bigsize"\+, \+"\biggsize"\+, and \+"\bigggsize"\+.  For
increasingly smaller sizes, you can use \+"\smlsize"\+,
\+"\smllsize"\+, and \+"\smlllsize"\+.  You can switch to the
regular size with \+"\regsize"\+.  The size-changing commands all
switch to the roman face.

There are commands \+"\smlrm"\+, \+"\smllrm"\+, and \+"\smltt"\+
which are like face-changing commands in that the baseline
spacing does not change, but the ``typeface'' they switch to are
actually smaller sizes of the indicated face.  For example, to
get the effect of what printers call ``small caps'', you can use
the "\smllrm" face.  To get \lisp, you would say
"L{\smllrm"~"ISP}".

If you ever want to set material in the font selected by a font
parameter such as "\titlefont", you can select that font by
saying "\the" in front of the parameter name.  For example, to
set three words in the "\captionfont", I could say
\beginverb
Here are {\the\captionfont three small words}.
\endverb
which gives me
\begintext
Here are {\the\captionfont{}three small words}.
\endtext
\vpar

The command \+"\singlespace"\+ causes lines to be single spaced,
and the command \+"\doublespace"\+ causes lines to be double
spaced.  Single spaced lines are the default.  These commands
affect all text, including footnotes and captions, so that if you
use "\doublespace" you might want to add a "\singlespace" command
to the font parameters for footnotes, captions, and so on.  If
you don't like the exact spacing values used by "\singlespace" or
"\doublespace", read what it says below about line spacing for
wizards and define your own versions of these commands.

\heading {Figures}

You can ask \yTeX\ to lay out your figures in one of four ways:
{\it stationary}, which means that figures are boxes that appear
where they are defined; {\it section}, which are like stationary
except that they are followed by "\vfil" glue which makes them
suitable for putting all in one section with no intervening text;
{\it top}, which means all figures float to the top of a page; or
{\it floating}, which means they appear where they are defined
unless they can't fit on the page there in which case they float
to the top of the next page they fit on.  The current layout
discipline (default {\it floating}) can be set with
\+"\stationaryfigures"\+, \+"\sectionfigures"\+,
\+"\topfigures"\+, and \+"\floatingfigures"\+.

\begingab

\*\+"\figdef"\+ and \+"\figref"\+\\are no-ops which are useful
for keeping track of figure numbers.

\*\+"\beginfigure"\+\\delimits a figure definition.  You get a
figure layed out according to the current discipline.  Each
figure is enclosed in a "\vbox" so you don't have to worry about
breaks.

\begingroup \figurelinetrue
\beginfigure

\noindent This figure was defined with
\beginverb
\beginfigure
\noindent This figure was defined with ...
   ...
\begincaption
Figure~\figdef{1}.  A figure which is ...
... it will float.
\endcaption
\endfigure
\endverb
and this was the result.

\begincaption
Figure~\figdef{1}.  A figure which is also an example of how to
prepare a figure.  This figure will appear where it was defined
unless it can't fit on that page, in which case it will
float.
\endcaption

\endfigure
\endgroup

\*\+"\beginpagefigure"\+\\delimits a full-page figure definition.
These figures go on pages by themselves.

\beginpagefigure

\vfil

\noindent This full-page figure was defined with
\beginverb
\beginpagefigure
\vfil
\noindent This full-page figure was defined with ...
  ...
\begincaption
Figure~\figdef{2}.  A figure which is ...
... it will float.
\vfil
\endcaption
\endpagefigure
\endverb
and this was the result.

\vfil

\begincaption
Figure~\figdef{2}.  A figure which is also an example of how to
prepare a figure.  This figure will appear where it was defined
unless it can't fit on that page, in which case it will float.
\endcaption

\endpagefigure

\*\+"\beginstationaryfigure"\+\\gives a stationary figure no
matter what the current layout discipline is.  The paired
delimiter "\beginstationarypagefigure" gives a full-page
stationary figure, and there are similar explicit commands
provided for all four of the layout disciplines.

\*\+"\begincaption"\+\\is defined only within figures and
delimits captions, which are set as unindented paragraphs in
\="\captionfont"\= with margins narrowed by \="\captionindent"\=.
Captions are preceded by \="\abovecaptionskip"\=.

\endgab

For examples, see figures~\figref{1} and~\figref{2}, which I
referred to here by saying
\beginverb
... see figures~\figref{1} and~\figref{2}, ...
\endverb
Note, in particular, that figure~\figref{1} is bracketed by
horizontal lines.  \ytex\ will put out such lines if you say
\:"\figurelinetrue"\: at the front of your document; the default
is \:"\figurelinefalse"\: (no lines).  (The lines themselves are
put \="\figurelinedrop"\= away from the body of the figure.)

The easiest way to leave space for a paste-in figure is to define
a figure whose body consists of a vertical skip command, as in
\beginverb
\beginfigure
\vskip 2 true in
\begincaption
This will be a 2 inch high pasted in figure.
\endcaption
\endfigure
\endverb
Note that the amount to skip was given in {\tt true} inches, not
just inches.  This is because typesizes larger than ten point are
implemented by magnifying the entire document: if the dimension
had been given as "2in" and the typesize were given as "11pt",
\yTeX\ would leave 2.2 inches of space instead of 2.  The use of
"true" in the dimension specification prevents \yTeX\ from
magnifying the space it leaves for the paste-in
material.\footnote*{Keep in mind that specifying sizes in true
dimensions may cause the page breaks to change when the type
size changes.  If you are proofing in a larger typesize and you
want page breaks to remain consistent, you should specify
paste-in sizes {\it without} the "true" specification, basing the
declared size on the magnification which will be used in {\it
final} output.  Then don't worry about the fact that the spaces
in the proof version are a little large.}

There is a <toks> parameter \="\topsep"\= whose contents are
inserted between the floating figures and text on any page where
floating figures appear.  By default, "\topsep" is empty, so only
the normal below-figure space appears between floating figures
and text.  If you want more space, you can say, for example,
\beginverb
\topsep={\bigskip}
\endverb

\heading {Tables}

Tables are delimited with \+"\begintable"\+ and "\endtable".
Right after the "\begintable" command you must have a {\it
preamble specification} enclosed in brackets ("[" and "]"), such
as in
\beginverb
\begintable [l"r|lc]
\topline
Item&&        Price&&  User&          Use\cr
\dmidline
Widgets&&    \$2.50&&  D. Brotsky&    \TeX\ work\cr
\midline
Grommets&&\$2500.50&&  The Pentagon&  none\cr
\botline
\endtable
\endverb
which produces table~\tabref{1}.

\beginfigure

{\catcode`\"=12\relax
 \centerline {%
   \begintable [l"r|lc]
     \topline
     Item&&     Price&&     User&          Use\cr
     \dmidline
     Widgets&&    \$2.50&&  D. Brotsky&    \TeX\ work\cr
     \midline
     Grommets&&\$2500.50&&  The Pentagon&  none\cr
     \botline
   \endtable}}

\begincaption
Table~\tabdef{1}.  This table is produced by the example in the
text.  Well, actually, this table is {\it almost} produced by
that example; in fact, this one has also been centered on the
page by putting the entire text of that example into the argument
of "\centerline".
\endcaption

\endfigure

Inside of a preamble specification, each character stands for one
column in the table.  The meanings of the characters are:
\begindisplaytable [cl]
"l"&   left-aligned column\cr
"L"&   left-aligned math column\cr
"c"&   center-aligned column\cr
"C"&   center-aligned math column\cr
"r"&   right-aligned column\cr
"R"&   right-aligned math column\cr
"|"&   single-thickness vertical line\cr
 \"&   double-thickness vertical line\cr
"&"&   repeat specification marker\cr
\enddisplaytable
Math columns are just like their non-math counterparts except
that the entries are set in math mode in \="\tablestyle"\=
(default "\displaystyle").  The "&" specification allows you to
take advantage of \tex's automatic iteration of column
specifications: if you have more entries in a line than you have
columns in the preamble, the preamble columns will repeat from
wherever the "&" was specified.

The thickness of a "|" line is the <dimen> parameter
\="\vbarwidth"\= (default ".4pt") and the thickness of a \" line
is the <dimen> parameter \="\dvbarwidth"\= (default "1pt").  Keep
in mind that "|" and \" columns must be delimited by "&" like
other columns.  It's just that normally the contents of these
columns are empty, so you end up with "&&" in the rows of your
input.

Inside tables the command "\\" ends rows like "\cr".  In
addition, you get access to a variety of useful commands:

\begingab

\*\+"\inline"\+\\puts a horizontal line in the table.  Better to
use are \+"\topline"\+, \+"\midline"\+, and \+"\botline"\+ (which
you normally use at the top, middle, and bottom of the table)
because they put padding around the line which looks good.

\*\+"\dinline"\+, \+"\dtopline"\+, \+"\dmidline"\+, and
\+"\dbotline"\+\\are the double-width counterparts of the
commands mentioned above.  "\inline"-type lines have thickness
\="\hbarheight"\=, and "\dinline"-type lines have thickness
\="\dhbarheight"\=.

\*\+"\padline"\+ <dimen>\\puts <dimen> extra space between lines
of the table with the nice feature that vertical bars specified
in the preamble extend right through the extra space.  The
<dimen> is not an argument: don't put it in braces.

\endgab

In addition, you get \+"\tabdef"\+ and \+"\tabref"\+ which, like
"\figdef" and "\figref", are no-ops useful for keeping track of
table numbers.  Also, there are the following table-affecting
parameters:

\begingab

\*\="\pretabskip"\=, \="\intabskip"\=, and
\="\posttabskip"\=\\are the glue parameters that "\tabskip" glue
is set to before the first column, between columns, and after the
last column of a table.  Defaults are "\hfil" glue for
"\pretabskip" and "\posttabskip", "1em" glue for "\intabskip".

\*\="\tablewidth"\=\\is settable like a <dimen>, and setting it
has strong side effects.  If you set it positive, all lines in
tables are set to that width.  If you set it to zero, tables have
their natural width.  If you set it negative, tables are expanded
by the negative of what you set it to.  The default is "0pt";
that is, tables have their natural size.

\endgab

Finally, in addition to the "\begintable" command which puts a
box around its table, there is \+"\beginopentable"\+ which does
not put a box and there is \+"\begindisplaytable"\+ which centers
the whole table in a "$$"-style display.

\heading {Footnotes}

\ytex\ gives you a footnote mechanism essentially equivalent to
\plain's, but footnotes are output as adjoint-tagged paragraphs
in the \="\footnotefont"\= with an indent of
\="\footnotemarkerwidth"\=.

\begingab

\*\+"\footnote"\+\#{<tag>}\#{<footnote text>}\\puts <tag> where
you call "\footnote" and also gives you a footnote marked with
<tag>.  You use "\footnote" inside paragraphs.

\*\+"\nfootnote"\+\#{<tag>}\#{<footnote text>}\\is like
"\footenote" but the <tag> must be a number.  The tag is set as a
superscript.

\*\+"\vfootnote"\+\#{<tag>}\#{<footnote text>}\\is like
"\footnote" but it doesn't put the tag where you call it, only on
the footnote text itself.  You use "\vfootnote" between
paragraphs.

\*\+"\beginfootmatter"\+\\gives you a footnote with no marker and
with no left margin indentation where the marker would go.  I'm
not sure why you would want this.

\endgab

The no-ops \+"\footdef"\+ and \+"\footref"\+ are provided to help
keep track of footnote numbering.  There is a <toks> parameter
\="\botsep"\= whose contents are inserted between the text and
footnotes on any page where footnotes appear.  By default,
"\botsep" puts in some space and a short line.  If you just want
space, you can say, for example,
\beginverb
\botsep={\bigskip}
\endverb

\heading {Textual Displays}

These macros give you "$$"-like displays containing text and
other useful textual blocks.  There are no standard forms for any
of the "begin" and "end" paired delimiters that produce textual
displays.

\begingab

\*\+"\beginnofill"\+\\delimits text that is not made into filled
lines, but output in lines as it appears in the input.
Indentation is turned off.

\*\+"\begintext"\+\\gives you a "$$"-like display containing the
nofill material it delimits.  The displayed lines have width
"\displaywidth" and are indented "\parindent" from the normal
"\displayindent" of the display.

\*\+"\begintextlines"\+\\is like "\begintext" but the displayed
lines are the full "\hsize" wide and the normal "\displayindent"
is ignored.

\endgab

\hangindent 2\parindent \hangafter0
\noindent For example, this is a paragraph which has both its
crown and vest indented twice the normal paragraph indentation.
If, inside this paragraph, we produce a display with
\beginverb
\begintext
Here is the first line of the display
Here is the second.
\endtext
\endverb
then the result appears to have a left margin even with the
indentation of the paragraph, as in:
\begintext
Here is the first line of the display
Here is the second.
\endtext
(Each line of the display is indented "\parindent" from the
margin because each line starts a paragraph.)  But if we specify
the text of the display using the following commands:
\beginverb
\begintextlines
Here is the first line of the display
Here is the second.
\endtextlines
\endverb
then the display ignores the indentation of the paragraph,
producing the following:
\begintextlines
Here is the first line of the display
Here is the second.
\endtextlines

\begingab

\*\+"\begincode"\+ and \+"\begincodelines"\+\\are like
"\begintext" and "\begintextlines" but the material is set in the
typewriter font.

\*\+"\verb"\+\\is used as
\begintext
\noindent"\verb"<char><text><char>
\endtext
and gives you <text> verbatim in the typewriter font.  All \tex\
special characters (such as "\" and "&") are treated as normal
characters.

\*\+"\beginverb"\+ and \+"\beginverblines"\+\\are like
"\begincode" and "\begincodelines" except the intervening text is
taken verbatim.

\endgab

\begingroup

\newverbatim\myverb
  \begincs \beginmyverb
  \readcs \readmyv@rb
  \endcsname endmyverb%
  \display
  \setup \hsize=\displaywidth \advance\leftskip\displayindent \tt
  \begin
  \end
  \endnew
  
For example
\beginmyverb
\beginverb
\noindent This is some \verbatim \text.
\centerline {Note that commands are ignored.}
\endverb
\endmyverb
produces
\beginverb
\noindent This is some \verbatim \text.
\centerline {Note that commands are ignored.}
\endverb

\endgroup

\begingab

\*\+"\beginlisp"\+ and \+"\beginlisplines"\+\\delimit \lisp\
programs.  Programs are printed in the \="\lispfont"\= (default
"\regsize\tt") in nofill mode; the \tex\ control characters (such
as "#") often found in \lisp\ code are turned off; and comments
are printed in roman.  Programs are boxed to prevent breaks; the
sequence ";\pbrk"\+ at the start of a line makes that line an
escape into surrounding vertical mode: you can put in breaks and
glue.  If you just specify \+"\pbrk"\+ after the semicolon, with
nothing else on the line, you get a space suitable for insertion
between function definitions.

\endgab

For example, this input
\beginverb
\beginlisp
;;Good Old Factorial
(defun fact (n)
  (cond ((zerop n) 1)                ;base case
        (t (* n (fact (1- n))))))    ;recursive case
;\pbrk
(defun myfact (n)
  (cond ((> n 0) (fact n))           ;OK if pos
        (t (- (fact (- n))))))       ;invert if neg
;\pbrk \medskip
(fact -5)                            ;sample usage
    -125
\endlisp
\endverb
produces this output
\beginlisp
;;Good Old Factorial
(defun fact (n)
  (cond ((zerop n) 1)                ;base case
        (t (* n (fact (1- n))))))    ;recursive case
;\pbrk
(defun myfact (n)
  (cond ((> n 0) (fact n))           ;OK if pos
        (t (- (fact (- n))))))       ;invert if neg
;\pbrk \medskip
(fact -5)                            ;sample usage
    -125
\endlisp

\heading {Two-Column Text}

The paired delimiter \+"\begintwocolumntext"\+ sets its argument
material in two columns, with the columns separated by
\="\columnskip"\=.  Top figures and footnotes in this material
are still set in a single column extending the full width of the
page.  Note that you can not use the floating figure style in
two-column text because there is no way of knowing in advance
whether figure material set in this style should be one- or
two-columns wide.  Also, full-page stationary or section figures
are liable not to work.

The paired delimiter \+"\begindoublecolumn"\+ sets its argument
material two columns per page, just as if the "\hsize" had been
narrowed and the results pasted up.  Thus, every page becomes a
single column, and commands like "\eject" refer to columns, not
pages.  The command \+"\pageeject"\+ will force a true page
break.  {\it Warning}: The "\begindoublecolumn" command will
throw away everything on the current page, so it is only safe to
use at the start of a fresh page.

\heading {Formatting Small Pieces of Text}

These commands produce little chunks of text in useful shapes, or
combine small blocks of text in handy ways.

\begingab

\*\+"\spread"\+\\takes two or more arguments and spreads them out
evenly spaced on a line.  For example, the input
\beginverb
\spread{Piece1}{Piece2}{}{Piece4 (3 was empty)}
\endverb
produces
\begintextlines
\spread{Piece1}{Piece2}{}{Piece4 (3 was empty)}
\endtextlines
Each piece of text except the leftmost and rightmost are centered
on their appropriate position.  Notice that the empty argument
took up a position even though it didn't put anything in that
place.

\*\+"\begintleft"\+, \+"\begintcenter"\+, and
\+"\begintright"\+\\set their arguments as left-flush, centered,
and right-flush titles in the current font.  Titles are allowed
to contain more than one paragraph.  Inside of titles, the
commands \+"\\"\+, \+"\cr"\+, and \+"\crcr"\+ force line breaks.

\*\+"\sy"\+ and \+"\sybox"\+\\set their arguments in a slightly
smaller typewriter font.  They differ only in that "\sybox" also
puts the argument in an hbox so it can't be broken across lines.

\*\+"\ucsy"\+ and \+"\ucsybox"\+\\are relatives of "\sy" and
"\sybox" that put their arguments in upper case.

\*\+"\ignore"\+\\does just that to its single argument.

\*\+"\filpage"\+\\fills up the rest of the page with white space
and does a page break.

\endgab

\heading {The Date}

\begingab

\*\="\hour"\= and \="\minute"\=\\are <number> parameters that
hold the time \ytex\ started in 24-hour format.  \+"\daytime"\+
outputs this in the form \sybox{15:35}.

\*\+"\monthname"\+ and \+"\monthshortname"\+\\output the name of
the current month; "\monthshortname" is three letters long.

\*\+"\shortyear"\+\\names the current year without the first two
digits, as in 86.

\*\+"\date"\+, \+"\shortdate"\+, and \+"\slashdate"\+\\output the
current date as in June~9,~1986; 9~Jun~86; and 6/9/86.

\endgab

For example, you can get output like
\begintext
Today is day 14 of Aug ({\it{}i.e.} August), %
and the time is 14:34.
\endtext
by saying
\beginverb
Today is day \number\day\ of \monthshortname
({\it i.e.} \monthname), and the time is \daytime.
\endverb

\heading {Miscellany}

\begingab

\*\+"\ytex"\+, "\yTeX", "\YTEX", "\YTeX", and \+"\oopstex"\+\\all
give the \ytex\ symbol.

\*\+"\draft"\+\\adds a line to the bottom of each page that looks
like this:
\begintextlines
\let\message=\ignore%
\draft\line{\the\bottomgloss}%
\endtextlines
This command also sets the "\overfullrule" to "5pt" instead of
the default "0pt" and prints the message "{Draft}" on your
terminal screen.

\endgab

Here are the default values of a few parameters not yet
mentioned:
\beginverb
\hsize=28pc \vsize=44pc
\abovedisplayskip=3pt plus1pt minus2pt
\belowdisplayskip=3pt plus1pt minus2pt
\abovedisplayshortskip=0pt plus1pt
\belowdisplayshortskip=2pt plus1pt minus1pt
\parindent=2em
\parskip=0pt plus1pt
\normallineskip=2pt
\normallineskiplimit=0pt
\clubpenalty=900
\widowpenalty=900
\endverb
The default "\typesize" is ten point.  Keep in mind that, if you
make the "\typesize" "11pt" or "12pt", the "\hsize" and "\vsize"
will be magnified by $1.1$ or $1.2$, so you probably won't need
to change them.  If you have a definite measurement you want used
for the margins independent of what magnification gets used,
specify it in ``true'' units (as in "\hsize=6 true in").  The
output routine will always try to center your output on the
output page; you can use "\hoffset" and "\voffset" to shift it
off center any desired amount.

\section {For Wizards Only}

Here are some features that inexperienced \tex nicians should
probably avoid.  Once again they are grouped by topic: some of
the topics also appeared above.

\heading {Page Glosses}

Pages can have {\it glosses} which are special lines that appear
above the header and below the footer.  The <toks> parameters
\="\topgloss"\= and \="\bottomgloss"\= contain the contents of
the glosses.  You turn glosses off and on just like runners, but
both glosses default off.  The top gloss defaultly contains a
copyright message like the following:
\begintextlines
\copyrightholder={by the author}%
\line{\the\topgloss}%
\endtextlines
while the bottom gloss is defaultly empty.  The words ``by the
author'' are the value of the <toks> parameter
\="\copyrightholder"\=.

\begingab

\*\:"\showcopyrighttrue"\: and "\showcopyrightfalse"\\allow and
disallow the showing of the top gloss.  To actually get the top
gloss to show on a particular page, you also have to say
\+"\copyrightpage"\+ somewhere on that page.

\endgab

The "\draft" macro described above uses the bottom gloss.

\heading {Size Changing}

There are also explicit size-selection commands such as
"\twelevepoint", but these pay no attention to the "\typesize",
so their use is not advised.

The baseline spacing is set whenever a size change command is
given by saying
\beginverb
\normalbaselineskip=\the\baselinefactor em%
\baselineskip\normalbaselineskip
\endverb
Thus, the value of the <toks> parameter \="\baselinefactor"\=
(default "{1.3}") can be used to vary the baseline spacing.  Note
that changes in "\baselinefactor" will not take effect until you
give a size-changing command or say \+"\setnormalbaselines"\+.
For example, these are the default definitions for "\singlespace"
and "\doublespace":
\beginverb
\def\singlespace{\baselinefactor={1.3}\setnormalbaselines}
\def\doublespace{\baselinefactor={2.6}\setnormalbaselines}
\endverb

\heading {Setup Hooks}

A variety of the default user-level macros---such as figures,
captions, footnotes, etc.---provide for user-defined set-up
hooks.  For example, every figure macro calls the macro
"\setupfigure", which in turn calls a macro "\setupfigurehook"
which the user is free to define.  Check the source code for
details as to which of these hooks exist.

\heading {More About Tables}

There is an escape mechanism whereby arbitrary alignment
specifications can be put into the preamble of a table.  A column
is normally specified by a letter, but arbitrary material
enclosed within balanced braces may be used to specify a column
instead.  For example, the preamble specification
\beginverb
[l{\hfil $\displaystyle #$}c]
\endverb
produces the following preamble
\beginverb
\tabskip\intabskip
#\hfil & \hfil $\displaystyle #$& \hfil #\hfil
\tabskip\posttabskip\cr
\endverb

The material within matching braces can only specify one column
in the preamble: you must use more than one set of braces to
specify more than one column.  There is no limit on the number of
columns that can be specified in this way.  When a column is
specified in this way, the entry produced for that column by
"\padline" will contain an "\omit" so as to produce white space.
Thus, if the specified material contains a "\vrule"
specification, that rule will not extend through padding.

If the material in braces starts with a "\tabskip" specification,
\ytex\ assumes that it contains only that specification, and it
merely embeds the specification in the preamble instead of making
it a column specification.  For example, the specification
\beginverb
[l{\tabskip=2em }lr]
\endverb
produces the preamble
\beginverb
\tabskip\intabskip
#\hfil & \tabskip=2em #\hfil& \hfil #%
\tabskip\posttabskip\cr
\endverb
This mechanism can be used to vary the between-column skip within
a line.  Keep in mind, however, that the "\posttabskip" will
override such specifications after the last column.

The interline glue is normally turned off in tables, and struts
are used to keep the line spacing correct.  If all of the
preamble specifications in a table are given with the "{}" escape
mechanism instead of \ytex's abbreviations, you should put struts
in at least one of the columns to force correct line spacing.

If you specify \hbox{"height" <dimen>} as the contents of a "|"
or \" column, that will override the natural height of the
"\vrule".  This is the mechanism used by "\padline".

\section {A List of Many \ytex\ Commands}

Here is a list of all the commands, parameters, and switches
described hereto.  It is not a complete list of all the \ytex\
commands, but it gets most of the useful ones.

\heading {The Commands}

The {\bf type} given in this table is one of {\bf SD} for
standard delimiter, {\bf PD} for paired delimiter, {\bf Par} for
paragraph delimiter, and {\bf A} for abbreviation.  If a paired
delimiter has a standard form, it appears in this table in that
form.  Paired delimiters which do not have standard forms ({\it
i.e.}, those which delimit nofill text) appear in their "begin"
form alphabetized under "begin".

\medskip

\begingroup
  \def\crs{\cr\noalign{\vskip 0pt plus.2pt}}
  \intabskip=.5em
  \beginopentable [lcl]%
    \bf Command&    \bf Type& \bf Summary\cr
    \midline
    "\\"&                 A&  breaks lines in titles, ends rows in tables\crs
    "\abstract"&         PD&  abstract text\crs
    "\atpar"&           Par&  adjoint-tagged paragraph\crs
    "\author"&           PD&  author's name(s)\crs
    "\begincode"&        PD&  typewriter font textual display\crs
    "\begincodelines"&   PD&  unindented typewriter font textual display\crs
    "\beginlisp"&        PD&  \lisp\ program display\crs
    "\beginlisplines"&   PD&  unindented \lisp\ program display\crs
    "\beginnofill"&      PD&  unboxed unfilled text\crs
    "\begintext"&        PD&  unfilled textual display\crs
    "\begintextlines"&   PD&  unindented unfilled textual display\crs
    "\beginverb"&        PD&  typewriter font verbatim display\crs
    "\beginverblines"&   PD&  unindented typewriter font verbatim display\crs
    "\bigsize"&           A&  switch to a larger size of type\crs
    "\biggsize"&          A&  switch to an even larger size\crs
    "\bigggsize"&         A&  switch to an even larger size still\crs
    "\botline"&           A&  horizontal line for bottom of tables\crs
    "\bottomtext"&       PD&  text for bottom of title page\crs
    "\bpar"&            Par&  bullet paragraph\crs
    "\bullets"&          PD&  group of bullet paragraphs\crs
    "\caption"&          PD&  caption (figures only)\crs
    "\cauthor"&          PD&  centered author's name\crs
    "\chapter"&          PD&  chapter title\crs
    "\copyrightpage"&     A&  allow copyright message\crs
    "\cr"&                A&  breaks lines in titles, ends rows in tables\crs
    "\crcr"&              A&  breaks lines in titles, ends rows in tables\crs
    "\ctitle"&           PD&  centered title\crs
    "\date"&              A&  date as June 9, 1986\crs
    "\daytime"&           A&  time of day in 24 hour format\crs
    "\dbotline"&          A&  double width line for bottom of table\crs
    "\dinline"&           A&  double width unpadded table line\crs
    "\displaytable"&     PD&  table display\crs
    "\dmidline"&          A&  double width line for middle of table\crs
    "\doublecolumn"&     PD&  double-column mode\crs
    "\doublespace"&       A&  double space text\crs
    "\draft"&             A&  put draft mark on all pages\crs
    "\dtopline"&          A&  double width line for top of table\crs
    "\figdef"&            A&  no-op for managing figure numbers\crs
    "\figref"&            A&  no-op for managing figure numbers\crs
    "\figure"&           PD&  figure in current style\crs
    "\figurebox"&        PD&  stationary boxed figure\crs
    "\filpage"&           A&  end current page with white space\crs
    "\floatingfigures"&   A&  figure style is {\it floating}\crs
    "\floatingfigure"&   PD&  define floating figure\crs
    "\floatingpagefigure"&PD&  define floating full-page figure\crs
    "\footdef"&           A&  no-op for managing footnote numbers\crs
    "\footref"&           A&  no-op for managing footnote numbers\crs
    "\footnote"&         SD&  specify footnote while in paragraph\crs
    "\footmatter"&       PD&  specify footnote text with no marker\crs
    "\ftpar"&           Par&  flush-tagged paragraph\crs
    "\heading"&          PD&  left-flush heading\crs
    "\ignore"&            A&  ignore argument\crs
    "\inline"&            A&  unpadded horizontal line in table\crs
    "\ipar"&            Par&  indented paragraph\crs
    "\ivpar"&           Par&  inverted paragraph\crs
    "\linebreak"&         A&  force line break in paragraph\crs
    "\midline"&           A&  horizontal line for middle of table\crs
    "\monthname"&         A&  full name of current month\crs
    "\monthshortname"&    A&  first three letters of current month\crs
    "\nfootnote"&        SD&  specify numbered footnote in paragraph\crs
    "\oopstex"&           A&  the \ytex\ symbol\crs
    "\opentable"&        PD&  unboxed table\crs
    "\padline"&           A&  put padding in table\crs
    "\pageeject"&         A&  force a page break in double-column mode\crs
    "\pagefigure"&       PD&  full page figure in desired style\crs
    "\partpage"&         PD&  titling page with usual page number\crs
    "\pbrk"&              A&  escape to vertical mode in \lisp\ programs\crs
    "\pchapter"&         PD&  chapter title on new page\crs
    "\quote"&            PD&  quotation text\crs
    "\regsize"&           A&  switch to the regular size of type\crs
    "\secdef"&            A&  no-op for managing section numbers\crs
    "\secref"&            A&  no-op for managing section numbers\crs
    "\section"&          PD&  section title\crs
    "\sectionfigures"&    A&  figure style is {\it section}\crs
    "\sectionfigure"&    PD&  define section figure\crs
    "\sectionpagefigure"&PD&  define section full-page figure\crs
    "\setnormalbaselines"&A&  use "\baselinefactor" to set baselines\crs
    "\shortdate"&         A&  date as in 9 Jun 86\crs
    "\shortyear"&         A&  last two digits of current year\crs
    "\singlespace"&       A&  single space text\crs
    "\slashdate"&         A&  date as in 6/9/86\crs
    "\smlsize"&           A&  switch to a smaller size of type\crs
    "\smllsize"&          A&  switch to an even smaller size\crs
    "\smlllsize"&         A&  switch to an even smaller size still\crs
    "\spread"&           SD&  spread args evenly on line\crs
    "\stationaryfigures"& A&  figure style is {\it stationary}\crs
    "\stationaryfigure"& PD&  define stationary figure\crs
    "\stationarypagefigure"&PD&  define stationary full-page figure\crs
    "\subsection"&       PD&  synonym for "\section"\crs
    "\subsubsection"&    PD&  synonym for "\section"\crs
    "\sy"&               SD&  set arg in smaller typewriter font\crs
    "\sybox"&            SD&  hboxed "\sy"\crs
    "\tabdef"&            A&  no-op for managing table numbers\crs
    "\table"&            PD&  vboxed table\crs
    "\tabref"&            A&  no-op for managing table numbers\crs
    "\tcenter"&          PD&  centered title\crs
    "\title"&            PD&  title\crs
    "\titlepage"&        PD&  titling page with page number 0\crs
    "\tleft"&            PD&  left-flush title\crs
    "\topfigures"&        A&  figure style is {\it top}\crs
    "\topfigure"&        PD&  define top figure\crs
    "\toppagefigure"&    PD&  define top full-page figure\crs
    "\topline"&           A&  horizontal line for top of table\crs
    "\tright"&           PD&  right-flush title\crs
    "\twocolumntext"&    PD&  two-column text\crs
    "\ucsy"&             SD&  uppercase "\sy"\crs
    "\ucsybox"&          SD&  uppercase "\sybox"\crs
    "\unvpar"&            A&  allow indentation on next paragraph\crs
    "\verb"&              A&  verbatim text\crs
    "\vfootnote"&        PD&  specify footnote text in vertical mode\crs
    "\vpar"&              A&  disallow indentation on next paragraph\crs
    "\vtpar"&           Par&  variably-tagged paragraph\crs
    "\ytex"&              A&  the \ytex\ symbol\cr
  \endopentable
\endgroup

\heading {Parameters}

The {\bf type} given in this table is one of {\bf I} for
<integer>, {\bf D} for <dimen>, {\bf G} for <glue>, and {\bf
T} for <toks>.

\medskip

\begingroup
  \def\crs{\cr\noalign{\vskip 0pt plus.2pt}}
  \intabskip=.5em
  \beginopentable [lcl]%
    \bf Parameter&  \bf Type& \bf Default\cr
    \midline
    "\abovecaptionskip"&   G&  "\medskipamount"\crs
    "\abovechapterskip"&   G&  "18pt", stretch is 3 times the "\parskip"'s\crs
    "\abstractfont"&       T&  "{\smlsize\rm}"\crs
    "\abstractindent"&     D&  "0pt"\crs
    "\authorfont"&         T&  "{\regsize\rm}"\crs
    "\baselinefactor"&     T&  "{1.3}"\crs
    "\belowchapterskip"&   G&  "9pt", stretch is 3 times the "\parskip"'s\crs
    "\bottomgloss"&        T&  <empty>\crs
    "\bottomtextfont"&     T&  "{\smllsize\rm}"\crs
    "\botsep"&             T&  "{\vskip\bigskipamount \footnoterule}"\crs
    "\captionindent"&      D&  "\parindent"\crs
    "\captionfont"&        T&  "{\smlsize\rm}"\crs
    "\chapterfont"&        T&  "{\bigsize\bf}"\crs
    "\columnskip"&         D&  "2pc"\crs
    "\copyrightholder"&    T&  "{by the author}"\crs
    "\dhbarheight"&        D&  "1pt"\crs
    "\dvbarwidth"&         D&  "1pt"\crs
    "\figurelinedrop"&     D&  "4pt"\crs
    "\footerdrop"&         D&  "1.25pc"\crs
    "\footnotefont"&       T&  "{\smlsize\rm}"\crs
    "\footnotemarkerwidth"&D&  "10pt"\crs
    "\hbarheight"&         D&  "0.4pt"\crs
    "\headerdrop"&         D&  "1.25pc"\crs
    "\hour"&               I&  starting hour of job\crs
    "\intabskip"&          G&  "1em"\crs
    "\lispfont"&           T&  "{\smlsize\tt}"\crs
    "\minute"&             I&  starting minute of job\crs
    "\pageno"&             I&  current page number\crs
    "\partagsep"&          D&  separation between tag and crown line\crs
    "\postchapterpenalty"& I&  "10000"\crs
    "\posttabskip"&        G&  "0pt plus 1fil"\crs
    "\prechapterpenalty"&  I&  "-500"\crs
    "\pretabskip"&         G&  "0pt plus 1fil"\crs
    "\quotefont"&          T&  "{\smlsize\rm}"\crs
    "\rectoleftheader"&    T&  "{\firstmark}"\crs
    "\rectocenterheader"&  T&  <empty>\crs
    "\rectorightheader"&   T&  "{\bf folio}"\crs
    "\rectoleftfooter"&    T&  <empty>\crs
    "\rectocenterfooter"&  T&  "{\bf folio}"\crs
    "\rectorightfooter"&   T&  <empty>\crs
    "\runnerfont"&         T&  "{\smlsize\rm}"\crs
    "\runninghead"&        T&  <synonym for "\rectoleftheader">\crs
    "\tablewidth"&         D&  "0pt" (settable only)\crs
    "\tablestyle"&         T&  {"\displaystyle"}\crs
    "\titlefont"&          T&  "{\bigsize\bf}"\crs
    "\topgloss"&           T&  <empty>\crs
    "\topsep"&             T&  <empty>\crs
    "\typesize"&           D&  10pt (settable only)\crs
    "\vbarwidth"&          D&  "0.4pt"\crs
    "\versoleftheader"&    T&  "{\bf folio}"\crs
    "\versocenterheader"&  T&  <empty>\crs
    "\versorightheader"&   T&  <empty>\crs
    "\versoleftfooter"&    T&  <empty>\crs
    "\versocenterfooter"&  T&  "{\bf folio}"\crs
    "\versorightfooter"&   T&  <empty>\cr
  \endopentable
\endgroup

\heading {Switches}

The {\bf type} given in this table is one of {\bf O} for on/off
and {\bf T} for true/false.

\medskip

\begingroup
  \def\crs{\cr\noalign{\vskip 0pt plus.2pt}}
  \intabskip=.5em
  \beginopentable [lcl]%
    \bf Switch&     \bf Type& \bf Default\cr
    \midline
    "\centerheadings"&     T&  false\crs
    "\figureline"&         T&  false\crs
    "\footers"&            O&  off\crs
    "\headers"&            O&  on\crs
    "\indent"&             O&  on\crs
    "\runners"&            O&  on\crs
    "\showcopyright"&      T&  false\crs
    "\twosided"&           T&  false\crs
    "\vpar"&               T&  true\cr
  \endopentable
\endgroup

\bye
