% for yTeX
% Copyright (c) 1990 Massachusetts Institute of Technology
% 
% This material was developed by the Scheme project at the Massachusetts
% Institute of Technology, Department of Electrical Engineering and
% Computer Science.  Permission to copy this material, to redistribute
% it, and to use it for any non-commercial purpose is granted, subject
% to the following restrictions and understandings.
% 
% 1. Any copy made of this material must include this copyright notice
% in full.
% 
% 2. Users of this material agree to make their best efforts (a) to
% return to the MIT Scheme project any improvements or extensions that
% they make, so that these may be included in future releases; and (b)
% to inform MIT of noteworthy uses of this material.
% 
% 3. All materials developed as a consequence of the use of this
% material shall duly acknowledge such use, in accordance with the usual
% standards of acknowledging credit in academic research.
% 
% 4. MIT has made no warrantee or representation that this material
% (including the operation of software contained therein) will be
% error-free, and MIT is under no obligation to provide any services, by
% way of maintenance, update, or otherwise.
% 
% 5. In conjunction with products arising from the use of this material,
% there shall be no use of the name of the Massachusetts Institute of
% Technology nor of any adaptation thereof in any advertising,
% promotional, or sales literature without prior written consent from
% MIT in each case. 


\typesize=11pt
\hsize=34pc
\vsize=50pc
\parskip 6pt plus 2pt

\def\psetheader{
\centerline{MASSACHUSETTS INSTITUTE OF TECHNOLOGY}
\centerline{Department of Electrical Engineering and Computer Science}
\centerline{6.001 Structure and Interpretation of Computer Programs}
\centerline{Fall Semester, 1987}}

\def\code#1{\beginlisp
#1
\endlisp

\vskip .1in}

\rectoleftheader={6.001 -- Fall Semester 1987}
\rectorightheader={Problem Set 2}
\onheaders
\onfooters

\null
\vskip 1truein

\psetheader

\vskip .25truein

\centerline{Problem Set 2}

\vskip 0.25truein

\vpar
Issued: September 17, 1987


\vpar
Due: in recitation on September 25, 1987 {\it for all sections}


\vpar
Reading assignment: Chapter 1, Sections 1.2 and 1.3

\chapter{1. Homework exercises}

Do the following exercises in the textbook:


\beginbullets

\bpar Exercise 1. -- Do exercise 1.12  in the textbook.


\bpar Exercise 2. -- Do exercise 1.13  in the textbook.
      (The answer to exercise 1.11 is given in the appendix to help you
      in doing this exercise.)

\bpar Exercise 3. -- Using the RSM (revised substitution model), show
      how the following expressions evolve when being evaluated.
      Indicate which rule you used to get from one step to another.

\hskip 20pt a. {\tt (define foo (lambda (m) (lambda (z) (m z))))}

\hskip 20pt b. {\tt  ((lambda (x) ((foo x) square)) 3)}

\hskip 20pt c. {\tt (define bar ((lambda (g) (lambda (y) (g y))) foo))}

\hskip 20pt d. {\tt ((bar foo) square)}

\hskip 20pt
-- assume that square is defined as:

\hskip 20pt      {\tt (define (square x) (* x x))}


\bpar Exercise 4. -- Consider the following higher order procedure
      {\tt repeated}, which takes a function {\tt f} of one argument, and
      a number {\tt n}, and returns a procedure which, when applied to an argument
      {\tt a}, computes 
\math{ \underbrace{\hbox{\tt (f (f (f.....(f}}_{\hbox{\rm n times}}
\hbox{\tt a))))}}
      The procedure is given below:
\beginlisp
      (define (repeated f n)
        (if (= n 0)
            (lambda (z) z)
            (compose f (repeated f (- n 1)))))
\pbrk
      (define (compose f g)
        (lambda (x) (f (g x))))
\endlisp

\hskip 20pt
      Assume that we now define a function called add-5 as follows:
\beginlisp
      (define (add-5 y) (+ y 5))
\endlisp
      Using your intuition, predict and describe the object returned
      as a result of evaluating the expression
\beginlisp
      (repeated add-5 2)
\endlisp
      Put the returned object in RSM notation.  (Note: DO NOT show your
      work for this one, because it would be quite disasterous for
      your TA to read.  If you really want to work it out explicitly
      using RSM, try to skip the details and concentrate on the
      returned object at each stage of the recursion, putting
      these returned objects together by carefully using rule C2.)

\endbullets

\chapter{2. Laboratory Assignment: RSA Cryptography System}

Cryptography systems are used to encode and decode information for
secure transmission and storage.  They are typically based around the
notion of using keys for encryption and decryption, either transforming
an English message into a new string of symbols, or vice versa.  Keys
can be as simple as a table that pairs input characters with output
characters, although real encryption systems are much more complex.

The RSA Cryptography System is based on using very large prime numbers
to determine the keys for encryption and decryption of messages.  The
algorithms for testing and thus finding large prime numbers efficiently
are based on Section 1.2.6 of the text.  You should read that material before
beginning work on this assignment.

The RSA scheme belongs to the family of Public Key Cryptography Systems which
are used by corporations, the government or even individuals to send secret
messages to each other via public wires.  A cryptography system must ensure
the privacy of such communication, as well as preventing forgeries.
The RSA scheme is the first and probably still the best way of realizing such
communications.  It was invented by Rivest, Shamir and Adleman (all at the
time at M.I.T., in course 6-3, 18 and 18 respectively.)  It is based on the
following widely accepted hypothesis:

\beginquote
If $n=pq$, where $p$ and $q$ are large prime numbers (say 100 digits long), then
it is in principle and in practice essentially impossible in any reasonable
length of time, given the number $n$, to find its two prime factors $p$ and
$q$.
\endquote

(Note: The above is quoted from Professor Arthur Mattuck's 18.063 notes.)

In the RSA scheme, a user selects two secret 100 digit primes, $p$ and $q$.
He or she then define
\math{\eqalign{n& = pq\cr
        m& = (p - 1)(q - 1).\cr}}

He/she also selects a 100-digit (or larger) encryption number $e$, such that
$e$ and $m$ are co-primes, i.e., the only common factor of e and m is 1.
(Remember that any integer number can be represented uniquely by its
factors, which are those integers that divide the original one without
remainder.)
The
user puts $n$ and $e$ on public file, but keeps $m$ secret.  To encrypt a message,
$s$, which is represented by a number, we perform the following calculation:

\centerline{encrypted message = $s$ to the power of $e$, modulo $n$}

Remember that modular arithmetic means that we only keep the remainder
of a number relative to a modular base.  For example, the number 100
modular 8 is given by 4, since that is the remainder that is left after
dividing 100 by 8.

Given an encrypted message, the receiver wants to decrypt it to obtain
the original message.  In order to do this, he/she wants to
perform the following calculation, with some number $d$:

\centerline{$s^\prime$ = encrypted message to the power of $d$, modulo
$n$.} 

In order for this to work, the receiver needs to find a number $d$ such
that $s = s^\prime$.  You will notice that the requirement for the
number $d$ is that:   

\centerline{$s$ = $s$ to the power of $e$, then to the power of $d$, modulo
$n$.} 

The equation for finding the number $d$ is:

\centerline{$d = (1 - mr)/e$, where $r$ is a negative integer, and $d$ a positive one.}

(Since this course is not 18.063, we won't ask you to understand all of the
above - you'll have to do that when you take that course.)

So much for theory - let's see some practical use of RSA.  Before we take
off, you need to set up a file in which to put in your solutions.  All the
procedures in the appendix of the problem set are installed on the
Chipmunks, and can be loaded onto your floppy disk.  To do this,
follow the instructions given in the Chipmunk manual in the section,
``Loading Problem Set Files,'' to load the code for problem set 2.
You can now use this code, modify it and add your solutions to it.
The {\tt fast-prime?} procedure and all the sub-procedures that it uses
are 
taken from Section 1.2.6 of the textbook. You are also provided with a
public key selection procedure called {\tt select-keys}, which uses
procedures
{\tt select-prime, select-e} and {\tt gcd} (greatest common divisor). 
Take a look at
these procedures and make sure you understand them before trying them out.

\section{The Story}

Our good friend Louis Reasoner has just joined the CIA upon receiving his
B.S. degree from W.I.T. (Worthwhile Institute of Technology).  As a result of
his specialization in cryptography at W.I.T, he was assigned to implement a
new RSA system for the CIA, who is planning to use it for negotiating a
highly secret business deal with Great Britain.

\section{Exercise 1}

The first procedure Louis Reasoner wrote for calculating modulo-m
exponentiation is {\tt expmod-1}.  Smart as he is, he also wrote a
procedure
called {\tt timed-exp}, which times how long it takes the computer to run
a 
procedure {\tt f} on arguments {\tt b e m}.  Use {\tt expmod-1} to
calculate 10 to the power
of 8, modulo 7.  Then use {\tt timed-exp} to time this calculation. 
Prepare a chart as follows:

\centerline{\begintable ["l|c|c|c"]
\dtopline
& &&10e8(mod 7)&&10e16(mod 7)&&10e26(mod 7)&\cr
\dmidline
&{\tt expmod-1}&& && && &\cr
\midline
&{\tt expmod-2}&& && && &\cr
\midline
&{\tt expmod-3}&& && && &\cr
\dbotline
\endtable}

Fill in the time it takes for {\tt expmod-1} to calculate the specified
numbers in the chart.

\section{Exercise 2}

Louis is pleased with the result, but his supervisor Alyssa P.
Hacker, who has an M.S. degree from M.I.T., critized this procedure, saying,
``The order of growth of your procedure can be reduced considerably, using the
idea of successive squaring.''  She then suggested Louis to take a look 
at Section 1.2.4 in her favorite bed-time reading - {\bf Struggle and
Intimidation of Computer Programming}.  Louis quickly read the section, and wrote
{\tt expmod-2}.  Test out the performance of his {\tt expmod-2} and fill
in the results in the prepared chart.

\section{Exercise 3}

Louis showed his {\tt expmod-2} to Alyssa P. Hacker.
Alyssa said,"That's not bad, but you are not taking any advantage
of the fact that we are doing modulo arithmetics!".  She quickly wrote
{\tt expmod-3} for Louis.  Test out {\tt expmod-3}
and complete the chart.  Hand the chart in and answer the following
questions:

(a) What is the order of growth of {\tt expmod-1, expmod-2} and {\tt
expmod-3}?

(b) Explain why {\tt expmod-3} runs faster than both {\tt expmod-1} and
{\tt expmod-2}.

\section{Exercise 4}

``Finally my expmod is working!'', said Louis with great relief.  He then
spent the rest of the day completing the key selection
procedures.
When the moonlight came through his office window at 8:59 pm, he was
already too tired to test out those new procedures, and fell asleep on
his desk.  Please test out these procedures for Louis.  Simply run the
{\tt select-key} procedure and generate some $n$ and $e$ numbers.  Notice
that we
are not really using 100 digit prime numbers because of the limited
capacity of the Chipmunks.  Make a list of 4 pairs of $n$ and $e$ numbers
as follows:

\centerline{\begintable ["l|c|c"]
\dtopline
& &&n&&e&\cr
\dmidline
&{\tt key-pair-1}&& && &\cr
\midline
&{\tt key-pair-2}&& && &\cr
\midline
&{\tt key-pair-3}&& && &\cr
\midline
&{\tt key-pair-4}&& && &\cr
\dbotline
\endtable}

Hand in the list and answer the following question:

(a) Is the {\tt gcd} procedure an iterative or recursive procedure?

(b) Does the {\tt gcd} procedure give rise to an iterative or recursive
process? 

\section{Exercise 5}

.....Meanwhile, on the other side of the globe at exactly 3:59 in the
cold morning in Moscow, two KGB agents, Ordr O. Grofiev and Bloch Schtraktur,
are burning the midnight oil to complete their implementation of the
code cracker for the CIA's new RSA system (their licensed-to-kill agent 6003
Sigmund Proceszeng found out about the CIA's recent activity.)  Here's
what they have so far:

\beginlisp
(define (crack-rsa n e)
   (define start-time (runtime))
   (define (iter guess)   ;;; iterator to find d, given a guessed r
      (........))
   (let ((p (prime-component n)))
      (if (= p 1)
          (print "Sorry - cannot find prime component")
          (.................))))
\pbrk
(define (prime-component n)
   (define (iter guess)
      (...........))
   (iter ...))
\pbrk
(define (improve guess)
   (- guess 1))
\endlisp

The dotted parts are not implemented yet.  In other words, these two nitwits
didn't come up with anything impressive, not to mention useful.  We ask you
in this exercise to complete these procedures, using the information we gave
you earlier.  Here is a summary of the highlights:

\math{\eqalign{
        n& = pq \qquad                         \hbox{\rm ($p$ and $q$ are primes)}\cr
        m& = (p - 1)(q - 1)\cr
        d& = (1 - mr)/e \qquad                  \hbox{\rm ($r$ is a negative integer)}\cr}}

{\tt Crack-rsa} takes the two public keys $n$ and $e$ as arguments, and
returns the decryption key $d$.  It calls {\tt prime-component} to find
out one of $n$'s prime
components.  $q$ and $m$ can subsequently be found.  Finding $d$ requires some
systematic guessing.  Make use of the fact that $d$ must be an integer,
as the stopping condition of the guessing.
The internal variable {\tt start-time} is there
for your convenience in putting timing mechanism into the procedure.  You
marry about changing the
{\tt improve} procedure, for you will be asked to do so in the subsequent
exercises.  The fragmented code of {\tt crack-rsa} and {\tt prime-component}
are not given with the rest of the code, so you will need to type it in.
The next exercise will enable you to test out your procedures.  Be aware
that you can add your own internal defines if you feel appropriate.

Hand in a listing of your procedures.

\section{Exercise 6}

Test out your {\tt crack-rsa} procedure by finding out the $d$ numbers (or
the
secret key) of the following pairs of $n$ and $e$ numbers (or the public keys.)
Before you test out your procedures, make sure that you have install a
timing mechanism in the
{\tt crack-rsa} procedure so that the {\tt crack-rsa} would print out the
elapsed time when it
has found out the answer.  Record the elapsed times and fill them in
the chart below in the ``time1'' column.

After you have found the secret keys to the corresponding public
keys, choose one set of keys to encrypt some random number and then
decrypt the encrypted number.  Use the {\tt encrypt} and {\tt decrypt}
procedures
provided in the code.  You should get back the same number after the
decryption.

\centerline{\begintable ["l|l|l|l|l|l"]
\dtopline
&n&&e&&d&&time1 &&time2&&time3&\cr
\dmidline
&19781&&817&& && && && &\cr
\midline
&27193&&363&& && && && &\cr
\midline
&27257&&169&& && && && &\cr
\midline
&61661&&433&& && && && &\cr
\dbotline
\endtable}

Hand in a SCHEME transcript (photo session) of an encryption-decryption
execution.

\section{Exercise 7}

The KGB sent out their 6002 agent Osczellas Gope and stole your solution
(assuming that if your solution has bugs, they could fix it), and installed
it onto their system.  Bloch Schtraktur, who specialized in algorithms in
Moscow University when he was 16 years old, immediately noticed that the
{\tt improve} procedure is very inefficient, in that it doesn't skip over
even
numbers.  Modify it so that it skips over even numbers, and then run
the same timing tests as you did in exercise 6 and put the results in the
``time2'' column.   You should find out how much more effective the new
{\tt improve} procedure is and report it directly to Alyssa P. Hacker.

Hand in a listing of your new {\tt improve} procedure.

\section{Exercise 8}

Ms. Hacker now learned from you what the the KGB has done.  ``Now, for
something tough to crack....'', said Ms. Hacker, as she increased the bounds of the
variable $n$ in {\tt select-prime} to 40 and 10 respectively.  ``This way,
our RSA system will be generating bigger prime numbers, which should take
anyone considerably longer time to crack the decryption keys'', she
explained with great confidence.

Modify the {\tt select-prime} procedure in your code according to
Alyssa,
and generate a few new keys, then use {\tt crack-rsa} to find the $d$
number
of these keys.  You'll realize that Alyssa is right (don't forget
that she is from M.I.T.)

The KGB is very upset.  6002, 6003 are so ashamed that they are
both ready to hand in their ``drop-cards'' (resignation letters).  At this
point of despair, they learned that Louis Reasoner is going to France for
vacation.  They immediately set up agent 6034 R. T. Fishle to
seduce
Louis.  Hopefully, the KGB could get some useful information out of Louis in
the process.  Ms. Fishle was very successful (and lucky) - Louis had
just developed a serious case of the sleep-talking syndrome very recently
due to too many
all-nighters.  While in a hotel with Ms. Fishle, Louis
not only gave away the new bounds in the {\tt select-prime} procedure,
he also gave away the equation $n^2+n+41$, which he uses
in his prime selection procedure.  Mr. Grofiev exclaimed, ``Let's improve
our {\tt improve} procedure using this equation.''  You are asked to do
just that in this exercise.  Rewrite the {\tt improve} procedure such
that it
systematically returns guesses which satisfy the formula $n^2+n+41$.

Hand in a listing of your new {\tt improve} procedure, and a SCHEME
transcript
of an encryption-decryption execution on large keys generated by the
{\tt select-keys} procedure, and do the timing test as you did in
exercise 6,
putting the results in the ``time3'' column, and hand in the completed
chart.

\section{Exercise 9}

Ordr O. Grofiev is a man with tremendous foresight.  One evening, when
he was taking a walk in the woods, he suddenly had the following idea,
``Suppose the CIA would like to change the bounds of the variable $n$ in
the select-prime procedure again, and suppose they would also like to
change the $n^2+n+41$ formula to some other formula for security
purposes. We would have to modify the body of the procedure {\tt improve}
very frequently,
getting frustrated due to bugs caused by forgetting to close a couple
of parentheses here and there.  If we can develop a generalized
{\tt improve}  procedure, which takes 3 extra arguments: $u$, $l$, and $f$
(standing
for upper-bound, lower-bound and function), we wouldn't have to touch
the {\tt improve} procedure ever again.  Of course, that requires
{\tt crack-rsa}
to also take these 3 extra arguments, so that they can be passed to
{\tt prime-component}, which can then pass them to {\tt improve}.''

Implement Ordr O. Grofiev's idea, and hand in a listing of your new
{\tt crack-rsa, prime-component}  and {\tt improve} procedures.  Also
hand in
a photo session of how you use your new {\tt crack-rsa} to crack
the $d$ number of the following keys: $n = 87953, e = 697$.

\section{Exercise 10}

Finally, the CIA is ready to use their system to send their first
message to
Great Britain.  They actually manage to sell some warheads to the British.
The first message consists of 3 numbers, which stand for the month, date and
year of the transaction.  The CIA encrypted them using the British's
public keys: $n = 99091, e = 225$.  The encrypted message is as follows:

\math{44287 \qquad   6521\qquad    58399}

Using the new {\tt crack-rsa} procedure you implemented in exercise 9,
find out the decryption key $d$ of the British.  Then, find out
the date of this business transaction by decrypting the above message.

Hand in a photo-session of the cracking of keys and the decryption.

\section{Exercise 11}

The KGB tapped the wires and decoded the message also.  They also happen
to know that the next message will be sent from the British to the CIA,
informing them about the exact location of the transaction.   Since the 
encryption keys are public, Ordr and Bloch decided to use the CIA's
public keys to encode a forgery message, which discloses a location where
the KGB, instead of Her Majesty's Secret Service,  will be waiting to
receive the shipment of warheads.

It turns out that the British are not stupid when it comes to such things
as secret business deals.  The British developed a scheme which prevents
forgery.  (This is, by the way, plagiarism - Rivest, Shamir and Adelman
invented it.)
It works as follows:

\math{\eqalign{B:& msg \Rightarrow_{encrypt} (msg)^{Bd}\Rightarrow_{encrypt}
(msg)^{Bd\cdot Ae} \Rightarrow_{decrypt}\cr
& \Rightarrow_{decrypt} (msg)^{Bd\cdot (Ae\cdot Ad)}
\Rightarrow_{decrypt} (msg)^{Bd\cdot Be} = msg :A\cr}}

Let $B$ stand for the British, and $A$ stand for the CIA.  $B$ first
encrypts its message by $B$'s secret key $Bd$.  Then it encrypts it
again using $A$'s public key $Ae$.  When $A$ receives this doubly encrypted
message, he can first decrypt it using his secret key $Ad$, and then
decrypt it once
again using $B$'s public key $Be$, which should yield the original message.
Since only $B$ knows $Bd$, nobody else could have encrypted a message
using $Bd$, which is only decodable by $Be$.  Therefore if the message
can be decrypted by $Be$, it must be encrypted using $Bd$, and thus must
be genuinely from $B$.

For this exercise, use the following information for keys:

\begintable [lll]
$Bn$ = 99091  &     $Be$ = 225 &       $Bd$ = what you found out in
exercise 10\cr
$An$ = 1189543  &  $Ae$ = 757 &       $Ad$ = (use {\tt crack-rsa} to find
out) \cr}}

24 hours later, the CIA received two messages:

\centerline{First message: 754786\ \    140376\ \   481610\ \   901930}

\centerline{Second messa out where the British want the transaction to be, and
where the Russians want the transaction to be - use a real world atlas
to do this.

The conclusion of the story depends, of course, on whether you get the
last exercise right or not.  Good Luck and Skill!


\end

