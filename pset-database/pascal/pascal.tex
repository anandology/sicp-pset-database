%% -*- LaTeX -*-
% Copyright (c) 1990 Massachusetts Institute of Technology
% 
% This material was developed by the Scheme project at the Massachusetts
% Institute of Technology, Department of Electrical Engineering and
% Computer Science.  Permission to copy this material, to redistribute
% it, and to use it for any non-commercial purpose is granted, subject
% to the following restrictions and understandings.
% 
% 1. Any copy made of this material must include this copyright notice
% in full.
% 
% 2. Users of this material agree to make their best efforts (a) to
% return to the MIT Scheme project any improvements or extensions that
% they make, so that these may be included in future releases; and (b)
% to inform MIT of noteworthy uses of this material.
% 
% 3. All materials developed as a consequence of the use of this
% material shall duly acknowledge such use, in accordance with the usual
% standards of acknowledging credit in academic research.
% 
% 4. MIT has made no warrantee or representation that this material
% (including the operation of software contained therein) will be
% error-free, and MIT is under no obligation to provide any services, by
% way of maintenance, update, or otherwise.
% 
% 5. In conjunction with products arising from the use of this material,
% there shall be no use of the name of the Massachusetts Institute of
% Technology nor of any adaptation thereof in any advertising,
% promotional, or sales literature without prior written consent from
% MIT in each case. 



\input /zu/6001/6001mac

\special{twoside}

\evensidemargin 35pt

\begin{document}

\psetheader{Fall Semester, 1989}{Problem set 2}

\medskip

\begin{tabular}{ll}
Issued: & Tuesday 19 September \\
Due: & recitation, Wednesday 27 September \\
Reading: & From text, Chapter 1, sections 1.2 and 1.3.
(This completes chapter 1.)
\end{tabular}

\noindent
{\small Problem sets should always be handed in at recitation.  Late work will
not be accepted.}

\section{Homework exercises}

Write up and turn in the following exercises from the text:

\begin{itemize}
\item Exercise 1.11: Iterative exponentiation

\item Exercise 1.24: Iteration

\item Exercise 1.25: Products

\item Exercise 1.28: Perversity
\end{itemize}

Also do the following exercise:

For each of the following expressions, what must {\cf f} be in order for the
evaluation of the expression to not cause an error?  For each expression, give
a definition of {\cf f} such that evaluating the expression will not cause an
error, and say what the expression's value will be, given your definition.

\beginlisp
f

(f)

(f 3)

((f))

(((f)) 3)
\endlisp

\section{Programming Assignment}

Hand in your work on all labelled problems.

\subsection{Pascal's Triangle}

Imagine that we have a triangular maze, with paths connecting an entrance at
the top to a set of bins at the bottom as in the diagram below.  Eventually the
path ends up in one of the bins at the bottom of the lattice.  (Perhaps you
have seen a demonstration of such a maze in an exhibit on probability.)  We
illustrate one typical path on the diagram by labelling it with ``*''.

\begin{minipage}[t]{\linewidth}
\begin{verbatim}
level

0     *
      |*  
      | * 
      |  *
1     |   *
      |\  *\
      | \ * \
      |  \*  \
2     |   *   |
      |\  |*  |\
      | \ | * | \
      |  \|  *|  \
3     |   |   *   |
      |\  |\  *\  |\ 
      | \ | \ * \ | \
      |  \|  \*  \|  \
4     |   |   *   |   |
      |\  |\  *\  |\  |\  
      | \ | \ * \ | \ | \ 
      |  \|  \*  \|  \|  \      
5     |   |   *   |   |   |
      |\  |\  |*  |\  |\  |\  
      | \ | \ | * | \ | \ | \ 
      |  \|  \|  *|  \|  \|  \
6     |   |   |   *   |   |   \
    |   |   |   | * |   |   |   |
    +---+---+---+---+---+---+---+
bin = 0   1   2   3   4   5   6 
\end{verbatim}
\end{minipage}

There are two ways to get to level 1.  There are four ways to get to level 2
(one way to get to level 2 bin 0, two ways to get to level 2 bin 1, and one way
to get to level 2 bin 2).

\problem{Problem 1}

In general, how many ways are there to get to the $n$th
level of the maze?  (This answer is easy.  Work it out before even
thinking about going to lab.)

It is a more complicated problem to determine how many different paths
there are from the vertex to a particular bin at the bottom of the
maze.

We observe that if we are at level zero we can get into the maze only
at bin zero.  If we are not at level zero there are only two ways to get
to bin $k$ at level $n$ --- we must come from either bin $k$ of level $n-1$ or
bin $k-1$ at level $n-1$.  Thus the number of paths to bin $k$ at level $n$
must be the sum of the number of paths to the contributing bins.  We
can formalize this description as a Scheme program as follows:

\beginlisp
(define (n-paths level bin)
  (if (= level 0)
      (if (= bin 0) 1 0)
      (+ (n-paths (- level 1) (- bin 1))
	 (n-paths (- level 1) bin))))
\endlisp

Compute, by hand, the table of values of {\cf (n-paths n k)} for $n, k = 0$, 1,
2, 3, 4, 5.  This takes very little work (fewer than five minutes!) if you
organize it carefully.  Start with small values and you will soon see the
pattern.  Check your answer on a Chipmunk if you like.

You have probably noticed that the numbers you computed in the previous
exercise are the binomial coefficients (Pascal's triangle).  These numbers
represent the number of combinations of $n$ objects taken $k$ at a time,
ignoring order.

\begin{center}
{\cf (n-paths n k)} $= C(n, k)$
\end{center}

Our algorithm for computing these numbers is one of the worst that a
sensible person might devise.

\problem{Problem 2}

In general, how many calls to {\cf n-paths} are required to
compute the value of {\cf (n-paths n k)} using the algorithm indicated
above?  The answer is a simple function of $n$ (independent of $k$).

An alternative (and far better way) to compute binomial coefficients
is to use the formula:
\begin{center}
$C(n, k) = \frac{n!}{k! (n-k)!}$
\end{center}
Unfortunately, factorials become very large very fast, and arithmetic
with large numbers is expensive.  Also, computing $C(n, k)$ in this way
requires a number of redundant multiplications (because some of the
multiplications in the computation of n! are just to be undone by the
division by either $k!$ or $(n-k)!$).

\problem{Problem 3}

Write a short, elegant program for {\cf (n-paths level bin)},
that avoids the redundant multiplications indicated above.  Hand in
this program, along with examples of its use, 
showing that it works for n up to 6.  Give an estimate
of the order of growth of time and space used by your program.

\subsection{Smoothing}

When random errors occur in measurements it is common to average data to reduce
their effect.  The following exercises are concerned with a form of averaging
called smoothing.  Consider measurements of the height of a ball thrown upwards
with velocity V at time -T, with T chosen such that the ball reaches its
maximum height (0) at time $t=0$.  The true position of the ball is
$y(t)=-\frac{1}{2}gt^2$, the measured position $m(t)=y(t)+e(t)$, where $e(t)$
is the measurement error.

\problem{Problem 4}

Construct a procedure {\cf true-pos} with arguments V, g and t that 
computes the true height of the ball at time t.

To reduce the effect of measurement errors, we might attempt to smooth the data
by averaging adjacent values: $S[m(t)] = \frac{1}{2}(m(t+h)+m(t-h))$, where $h$
is a small time increment.

We can express the idea of smoothing as a procedure (analogous to
the {\cf deriv} procedure discussed on p. 68 of the text) as follows:
\beginlisp
(define (smooth f h)
  (lambda (x)
    (average (f (+ x h))
	     (f (- x h)))))

(define (average x y)
  (/ (+ x y) 2))
\endlisp

\problem{Problem 5}

Even in the absence of random error ($e(t)=0$) $S[m(t)]$ differs from the true
position, $y(t)$.  We call this difference the systematic error introduced by
smoothing.  Determine how the systematic error depends on $h$.  You are to
develop an expression for the systematic error in the measurement of $y(t)$ as
a function of $h$.

\problem{Problem 6}

Verify, by numerical experiment, that the systematic error
introduced by applying {\cf smooth} to {\cf true-pos} has the 
dependence on $h$ that you determined above.

When the random error term is non-zero, $S[m(t)]$ differs from $y(t)$ by the
sum of the systematic error determined above and a term equal to
$\frac{1}{2}(e(t-h)+e(t+h))$.  In many cases this is smaller than $e(t)$.  For
example, when $e(t-h)$, $e(t)$, and $e(t+h)$ are roughly of the same size,
$\frac{1}{2}(e(t-h)+e(t+h))$ is of the same size when $e(t-h)$ and $e(t+h)$ are
of the same sign, but smaller when they differ in sign. In most practical
situations, $h$ is chosen by compromise since reducing the size of $h$
decreases the systematic error, but also reduces the likelihood that the
contributions of $e(t-h)$ and $e(t+h)$ to $S[m(t)]$ will cancel out.

When the random error of measurement is large it is often advantageous
to repeat the smoothing operation a number of times, though this will
increase the systematic error.  So, for example, {\cf (smooth (smooth m h)
h)} may be less noisy than {\cf (smooth m h)} which may be less noisy than $m$.

One useful abstraction for repeatedly applying a procedure is the
{\cf repeated} procedure, defined below:
\beginlisp
(define (repeated f n)
  (if (= n 0)
      identity
      (compose f (repeated f (- n 1)))))

(define (identity x) x)

(define (compose f g)
  (lambda (x) (f (g x))))
\endlisp

For example, using this procedure, one can write:

\begin{verbatim}
==> ((repeated square 3) 5)
390625
\end{verbatim}

\problem{Problem 7}

Define and test a procedure, {\cf nth-smooth}, of three arguments, {\cf n}, the
number of times to apply the smoothing procedure; {\cf f}, the function to
smooth; and {\cf h}, the smoothing interval.  {\cf nth-smooth} must return the
smoothed procedure.  For example, {\cf (nth-smooth 2 m h)} should return a
procedure equivalent to {\cf (smooth (smooth m h) h)}.  You should be sure to
work this out carefully before testing it on a Chipmunk.  The required
procedure is not long, but some thought is required to get it right.

%\beginlisp
%   ***** Answer
%   (define (nth-smooth n f h)
%     ((repeated (lambda (g) (smooth g h)) n) f))
%\endlisp

{\cf ((smooth f h) t)} is equivalent to the algebraic expression
$\frac{1}{2}(f(t-h)+f(t+h))$.  Write algebraic expressions equivalent to {\cf
((smooth (smooth f h) h) t)} and to {\cf ((smooth (smooth (smooth f h) h) h)
t)}.  You should see that the coefficients of {\cf f} in these expressions are
closely related to the numbers in Pascal's triangle, discussed above.

\problem{Problem 8}

How many times must {\cf f}, the procedure that implements the
function being smoothed, be applied when evaluating {\cf (nth-smooth n f h)}?

\problem{Problem 9}

Write a different implementation of {\cf nth-smooth} that applies the procedure
that implements the function being smoothed only $n+1$ times when evaluating
{\cf (nth-smooth n f h)}?  Make use of your work from Problem 3.

\end{document}
