\input ../formatting/6001mac.tex

\begin{document}

\psetheader{Spring Semester, 1989}{Problem Set 1}

\medskip

\begin{flushleft}
Issued:  February 7, 1989 \\
\smallskip
Due:
\begin{tightlist}
\item for sections 1--7 in Recitation on Wednesday, February 15
\item for sections 8--14 in Recitation on Friday, February 17
\end{tightlist}
\end{flushleft}

\noindent
{\small
Problem sets should always be handed in at recitation.  Late work will
not be accepted.  We are staggering the due dates for problem sets to
minimize crowding in the lab.  The Wednesday/Friday dates will be
reversed during the second half of the semester to assure equitable
treatment of students in both morning and afternoon sections.
Specific due dates will be announced as each problem set is handed
out. }

\begin{flushleft}
Reading: 
\begin{tightlist}

\item ``6.001 General Information'': Review this if you haven't
already done so.  Note in particular the hours when the lab is open.

\item Text: section 1.1.

\item  ``A 6.001 User's Guide to the Chipmunk System'': Chapters 1
and 2.  Bring this manual to lab with you to use as a reference.
\end{tightlist}
\end{flushleft}

Before you can begin work on this assignment, you will need to get a
copy of the textbook and pick up the manual package.  To obtain the
manual package, and follow the instructions given in the General
Information memo.

The laboratory assignments for 6.001 have been designed with the
assumption that you will do the required reading and textbook
exercises before you come to the laboratory.  You should also read
over and think through the entire assignment before you sit down at
the computer.  It is generally much more efficient to test, debug, and
run a program that you have planned before coming into lab than to try
to do the planning ``online.''  Students who have taken 6.001 in
previous terms report that failing to prepare ahead for laboratory
assignments generally ensures that the assignments will take much
longer than necessary.


\section{Part 1: Homework exercises}

Do the following exercises before going to the lab.  Write up your
solutions and include them as part of your homework.

\paragraph{Exercise 1.1}
Below is a sequence of expressions.  What is the result printed by the
interpreter in response to each expression?  Assume that the sequence
is to be evaluated in the order in which it is presented.  You should
determine the answers to this exercise without the help of a Chipmunk,
then check your answers when you come to the lab.

\beginlisp
==> (+ 7 8 9)
\null
==> (> 10 9.7)
\null
==> (- (* 2 5) (/ 16 10))
\null
==> (define double (lambda (x) (* x 2)))
\null
==> double
\null
==> (define c 4)
\null
==> c
\null
==> (double c)
\null
==> c
\null
==> (double (double (+ c 5)))
\null
==> (define times-2 double)
\null
==> (times-2 c)
\null
==> (define d c)
\null
==> (= c d)
\null
==> (if (= (* c 6) (times-2 d))
        (< c d)
        (+ c d))
\null
==> (cond ((>= c 2) d)
          ((= c (- d 5)) (+ c d))
          (else (abs (- c d))))
\null
==> (define try (lambda (x) (x 3)))
\null
==> (try double)
\null
==> (try (lambda (z) (* z z)))
\endlisp

Observe that some of the examples shown above are indented and
displayed over several lines for readability.  An expression may be
typed on a single line or on several, and redundant spaces and
carriage returns are ignored.  It is to your advantage to format your
work so that you (and others) can read it easily.

\paragraph{Exercise 1.2}
For each of the following four expressions, show the successive
steps---using the substitution model of evaluation---that reduce the
expression to a number.  In each expression, assume that $exp_1$
and $exp_2$ are subexpressions that evaluate to 5 and 12,
respectively.  (Of course, you can't show the details of evaluating
these two subexpressions.)  Also, for each expression, tell whether
applicative-order evaluation stipulates that $exp_1$ is evaluated
before $exp_2$, or vice versa, or whether the order of evaluating
these two subexpressions is unspecified.
\beginlisp
((lambda (x) ((lambda (y) (- y x)) $exp_1$)) $exp_2$)
\null
((lambda (y) ((lambda (x) (- y x)) $exp_2$)) $exp_1$)
\null
((lambda (x y) (- y x)) $exp_2$ $exp_1$)
\null
((lambda (y) (lambda (x) (- y x)) $exp_1$) $exp_2$)
\null
(((lambda (x) (lambda (y) (- y x))) $exp_1$) $exp_2$)
\endlisp

\paragraph{Exercise 1.3}
Translate the following arithmetic expression into Lisp prefix notation:
\begin{displaymath}
\frac{3+2+(1-(5-(2+\frac{3}{4})))}{4*(7-3)*(1-6)}
\end{displaymath}

\paragraph{Exercise 1.4}
Do exercise 1.4 of the text.


\section{Part 2: Getting started in the lab and using the debugger}

When you come to the lab, find a free computer and log in and
initialize one of your disks, as described at the beginning of chapter
1 of the Chipmunk manual.  You should also write your name and address
on a label affixed to the jacket of the floppy disk.  Floppy disks are
notoriously unreliable storage media, and it is a good idea to copy
your data onto a second disk (that is used only for this purpose) when
you have completed each laboratory assignment.  See the description of
how to copy disks in the Chipmunk manual, and do not hesitate to ask
the lab assistants for help.

After you have successfully logged in, load the material for problem
set 1, using the method described in chapter 1 of the Chipmunk
manual---namely, press \key{EXTEND} (i.e., the key marked
\kkey{2}) then type {\tt load problem set} and press \key{ENTER}.
When the system asks for the problem set number, type {\tt 1} and
press \key{ENTER}.  The system will load some files, and leave you
connected to the Scheme interaction buffer, with Scheme prompting you
for an expression to evaluate.

\subsection{Evaluating expressions} 

To get used to typing at the Chipmunks, check your answers to exercise
1.1.  After you type each expression, press the \key{EXECUTE} key (at
the lower right side of the keyboard) to evaluate the expression and
see the result.  Notice that when you type a right parenthesis, the
matching left parenthesis is briefly highlighted.  You can correct
typing errors with the \key{BACKSPACE} key.\footnote{Edwin is a
version of Emacs, the editor used on Athena workstations.  It has many
powerful editing features, which you can read about in chapter 2 of
the Chipmunk manual.  Although simple cursor motion will be your
primary editing tool at first, it will be greatly to your advantage to
learn to use some of the more sophisticated editing commands as the
semester progresses.}

To type expressions that go over a single line, use the \key{ENTER}
key (not the \key{EXECUTE} key) to move to the next line.  Notice that
the editor automatically ``pretty prints'' your procedure as you type
it in, indenting lines to the position determined by the number of
unclosed left parentheses.

\subsection{Using the debugger}

When you program, you will certainly make errors, both simple typing
typing mistakes and more significant conceptual bugs.  Even expert
programmers produce code that has errors; superior programmers make
use of debuggers to identify and correct the errors efficiently.  At
some point over the next month\footnote{Don't do it now---you have
enough to do now.} you should read chapter 3 of the Chipmunk manual,
which describes Scheme's debugging features in detail.  For now, we've
provided a simple exercise to acquaint you with the debugger.

Loading the code for problem set 1, which you did above, defined three
tiny procedures called {\tt p1, p2} and {\tt p3}.  We won't shown you
the text of these procedures---the only point of running them is to
illustrate the debugger.

Evaluate the expression {\tt (p1 1 2)}.  This should signal an error.
The screen splits into two windows, with the ordinary interaction
buffer at the top, and a {\tt *Scheme Error*} window at the bottom.
The error window contains the message
\beginlisp
Wrong Number of Arguments 1
within procedure \#[COMPOUND-PROCEDURE P2]
\endlisp
At the bottom of the screen is a question asking whether or not you
want to debug the error.

Don't panic.  Beginners have a tendency, when they encounter an error,
to immediately respond ``No'' to the offer to debug, sometimes without
even reading the error message.  Let's instead see how Scheme can be
coaxed into producing some helpful information about the error.

First of all, there is the error message itself.  It says that the
error was caused by a procedure being called with 1 argument, which is
the wrong number of arguments for that procedure.  The next line tells
you that the error occurred within the procedure {\tt P2}, so the
actual error was that {\tt P2} was called with 1 argument, which is
the wrong number of arguments for {\tt P2}.

Unfortunately, the error message alone doesn't say where in the code
the error occurred.  In order to find out more, you need to use the
debugger.  The debugger allows you to grovel around, examining pieces
of the execution in progress in order to learn more about what may
have caused the error.

Type {\tt y} in response to the system's question.
Scheme should respond:
\beginlisp
Subproblem Level: 0  Reduction Number: 0
Expression:
(P2 B)
within the procedure P3
applied to (1 2)
\endlisp

This says that the expression that caused the error was {\tt (P2 B)},
within the procedure {\tt P3}, which was called with arguments 1 and
2.  Ignore the stuff about subproblem and reduction numbers.  You can
read about them in chapter 4 of the Chipmunk manual.

The debugger differs from an ordinary Scheme command level, in that
you use single-keystroke commands, rather than typing expressions and
pressing \key{EXECUTE}.  One thing you can do is move ``up'' in the
evaluation sequence, to see how the program reached the point that
signaled the error.  To do this, type the character {\tt u}.  The
debugger should show:
\beginlisp
Subproblem level: 1 Reduction number: 0
Expression:
(+ (P2 A) (P2 B))
within the procedure P3
applied to (1 2)
\endlisp
Remember that the expression evaluated to cause the error was {\tt (P2
B)}.  Now that we have moved ``up,'' we learn that this expression was
being evaluated as a subproblem of evaluating the expression {\tt (+
(P2 A) (P2 B))} still within procedure {\tt P3} applied to 1 and 2.

So we've learned from this that the bug is in {\tt P3}, where the
expression {\tt (+ (P2 A) (P2 B))} calls {\tt P2} with the wrong
number of arguments.  At this point, one would normally quit the
debugger and edit {\tt P3} to correct the bug.

Before leaving the debugger, let's explore a little more.  Press {\tt
u} again, and you should see
\beginlisp
Subproblem Level: 2  Reduction Number: 0
Expression:
(+ (P2 X Y) (P3 X Y))
within the procedure P1
applied to (1 2)
\endlisp
which tells us that the program reached the place we just saw above as
a result of trying to evaluate an expression in {\tt P1}.

Press {\tt u} again and you should see some mysterious stuff.  What
you are looking at is some of the guts of the Scheme system---the part
shown here is a piece of the interpreter's read-eval-print loop.
In general, backing up from any error will eventually land you in the
guts of the system.  At this point you should stop backing up unless,
of course, you want to explore what the system looks like.  (Yes:
almost all of the system is itself a Scheme program.)

In the debugger, the opposite of {\tt u} is {\tt d}, which moves you
``forward.'' Go forward until the point of the actual error,
which is as far as you can go.

Besides {\tt b} and {\tt f}, there about a dozen debugger
single-character commands that perform operations at various
levels of obscurity.  You can see a list of them by typing {\tt ?}
at the debugger.  For more information, see the Chipmunk manual.

Type {\tt q} to quit the debugger and return to ordinary Scheme.

\section{3. Programming Assignment: The snowflake curve}

Now that you've gained some experience with the Chipmunks, you should
be ready to work on the programming assignment.  When
you are finished in the lab, you should write up and hand in the
numbered problems below.  You may want to include listings and/or
pictures in your write-up.  Chapter 1 of the Chipmunk manual explains
how to use the lab printers.

In the programming assignment for this week, you will be experimenting
with simple procedures that draw variants of a curve called the {\em
snowflake curve} or {\em Koch curve}, after the mathematician H. von
Koch who first described such curves in 1904.

\begin{figure}
\vskip 2 in
{\small
\begin{verbatim}
         level 0             level 1            level 2             level  3
\end{verbatim}
}
\vskip 1 in
\caption{{\protect\footnotesize
Snowflake curves at levels 0 through 3, and a recursive scheme for
drawing a side of the snowflake.}}
\label{snowflake}
\end{figure}

To generate the snowflake curve, you begin with a triangle and apply,
over and over again, the process of replacing each side by a curve
that consists of four smaller sides; each of these smaller sides is in
turn replaced by four smaller sides, and so on.  If you do this an
infinite number of times, the result is a strange topological beast
called a {\em fractal}.\footnote{A fractal curve is a ``curve'' which,
if you expand any small piece of it, you get something similar to the
original.  Fractals have striking mathematical properties.  The Koch
curve, for example, is neither a true 1-dimensional curve, nor a
2-dimensional region in the plane, but rather something in between.
In the mathematics of fractals, the dimension of the Koch curve turns
out to be $\log{4}/\log{3} \approx 1.26$.} Fractals have received a
lot of attention over the past few years, partly because they tend
to arise in the theory of nonlinear differential equations, but also
because they are pretty, and their finite approximations can be easily
generated with recursive computer programs.

Figure~\ref{snowflake} shows some approximations to the snowflake
curve, where we stop replacing sides after a certain number of levels:
a level-0 side is simply a straight line, a level-1 side consists of
four level-0 sides, a level-2 side four level-1 sides, and so on.  The
figure also illustrates a recursive strategy for drawing a side of
the snowflake: a level-$n$ side is a made up of four level-$(n-1)$ sides,
each 1/3 the length of the original side.  Two of the four sides are
parallel to the original, one is rotated by $\pi/3$ (60 degrees) and
one is rotated by $-\pi/3$.

We can translate this strategy into a pair of
procedures that draw side of level $n$ and length $L$, oriented at
a given angle: To draw a side, draw either a single line (at level 0)
or else draw four sides of level $n-1$ and length $L/3$:
\beginlisp
(define (side angle length level)
  (if (= level 0)
      (plot-line angle length)
      (4sides angle (/ length 3) (- level 1))))
\null
(define (4sides angle length level)
  (side angle length level) 
  (side (+ angle (/ pi 3)) length level)
  (side (- angle (/ pi 3)) length level)
  (side angle length level))
\endlisp
The procedure {\tt plot-line} is a primitive graphics command.  It
draw a line by moving a graphics ``pen'' through a specified distance,
in the direction of a specified angle. (Angle 0 is towards the left;
$\pi/2$ is straight up.)  Other graphics primitives are {\tt
clearscreen}, a procedure of no arguments that clears the graphics
screen and returns the pen to the center of the screen; and {\tt
set-pen-at}, a procedure of two arguments, $x$ and $y$, that sets the
pen at the point $(x,y)$ without leaving any trace.\footnote{If you
consult the Chipmunk manual, you will not find {\tt plot-line}, {\tt
clearscreen}, and {\tt set-pen-at} documented in the list of Scheme
primitives.  That is because they are implemented as procedures that
were loaded into Scheme when you loaded the code for problem set 1.
If you want to play, you can look at these procedures.  (Figure out
how to do this.)  They illustrate programming techniques that we will
study near the middle of the semester.  You might dispute whether
these are actually primitives, since they are implemented as ordinary
Scheme procedures.  Start getting used to the notion that often a
``primitive'' is simply something that we choose not to look inside
of.  This illustrates a major idea that we will see throughout 6.001:
One does not create complex systems by focusing on how to construct
them out of ultimate primitive elements.  Rather, one proceeds in
layers, erecting a series of levels, each of which is the
``primitives'' upon which the next level is constructed.}

Finally, we can draw the snowflake curve by drawing three sides,
oriented at 60 degrees, $-60$ degrees, and 180 degrees:
\beginlisp
(define (snowflake length level)
  (side (/ pi 3) length level)
  (side (- (/ pi 3)) length level)
  (side pi length level))
\endlisp

\subsection{Trying the program}

Start a new file on your disk to hold the code for your program by
pressing {\key{FIND FILE} (\shift \kkey{0})} and enter a name for this
file: {\tt ps1.scm}.  You can use some other name besides {\tt ps1} if
you like, but you should end the name in {\tt .scm}.  This tells the
editor that the file contains Scheme code.  Edwin will inform you that
this is a new file, and will create an empty edit buffer for it.
Subsequent laboratory assignments will include large amounts of code,
which will be loaded automatically into an edit buffer when you begin
work on the assignment.  This time, however, we are asking you to type
the code yourself, to give you practice with the editor.

Type in the definitions of the procedures {\tt snowflake}, {\tt side},
and {\tt 4sides}.  Use the \key{ENTER} key to go from one line to the
next, so that Edwin will automatically indent the code.  If Edwin's
indentation does not match what you expected, you have probably
omitted a parenthesis---observing Edwin's indentation is an excellent
way to catch parenthesis errors.  It's also a good idea to leave a
blank line between procedure definitions.

You will also need to define a value for the symbol {\tt pi}:
\beginlisp
(define pi (* 4 (atan 1 1)))
\endlisp
This expression defines {\tt pi} to be 4 times the arctangent of 1,
which is $\pi/4$.

``Zap'' your procedures from the editor into Scheme, using the
\key{EVALUATE BUFFER} key ({\shift \kkey{7}}).\footnote{\key{EVALUATE
BUFFER} transmits the entire buffer to Scheme, and is the easiest
thing to use when you only have one or two procedures in your buffer.
With larger amounts of code, it is better to use other ``zap''
commands, which transmit only selected parts of the buffer, such as
individual procedure definitions.  For a description of these
commands, see Chapter 3 of the Chipmunk manual, under the heading
``Scheme Interaction.''}.  If you get an error at this point, you
probably have a badly formed procedure definition.  Ask for help.
Otherwise, test your program by evaluating
\beginlisp
==> (clearscreen)
\null
==> (snowflake 200 4)
\endlisp
to draw a snowflake curve of side 200 and level 4.

To see the drawing, press the key marked \key{GRAPHICS} at the upper
right of the keyboard.  The Chipmunk system enables you to view either
text or graphics on the screen, or to see both at once.  This is
controlled by the keys marked \key{GRAPHICS} and \key{ALPHA}.
Pressing \key{GRAPHICS} shows the graphics screen superimposed on the
text.  Pressing \key{GRAPHICS} again hides the text screen and shows
only graphics.  \key{ALPHA} works similarly, showing the text screen
and hiding the graphics screen.

Try drawing some snowflake curves at various sizes and levels.  Use
{\tt clearscreen} to clear the screen and {\tt set-pen-at} to change
the initial position from which the curve is drawn.

To save your work on your disk, press {\key{SAVE FILE} (\kkey{0})}.
It's a good idea to save your work periodically, to protect yourself
against system crashes.

\subsection{Varying the parity}

The recursive scheme at the bottom of figure~\ref{snowflake} suggests
some simple variations.  For instance, in drawing the 4 small sides
rather than turning $\pi/3$ (``outward'') and then turning $-\pi/3$
(``inward'') you could first turn inward and then outward.  This would
orient the bulge formed by the middle two small sides towards the
inside of the snowflake rather than towards the outside.  More
generally, the decision to turn first $\pi/3$ and then $-\pi/3$,
versus turning first $-\pi/3$ and then $\pi/3$, could be made on a
level by level basis.  One way to implement such a choice is by means
of a procedure {\tt f} that takes the level number as argument and
returns either 1 or $-1$.  Then, in lines 2 and 3 of the procedure
{\tt 4sides}, the calls to {\tt side}
\beginlisp
(side (+ angle (/ pi 3)) $\ldots$
(side (- angle (/ pi 3)) $\ldots$
\endlisp
should be changed to
\beginlisp
(side (+ angle (* (f level) (/ pi 3))) $\ldots$
(side (- angle (* (f level) (/ pi 3))) $\ldots$
\endlisp
If {\tt f} returns 1 for that level, the bulge will be oriented
outward, while if {\tt f} returns $-1$ the bulge will be oriented
inward.

\paragraph{Problem 1}
Make the change to {\tt 4sides} indicated above.  Also change {\tt
snowflake}, {\tt side}, and {\tt 4sides} so that all three procedures
take {\tt f} as an extra argument.  To test your code, call your new
{\tt snowflake} procedure with an {\tt f} that always returns 1
\beginlisp
(snowflake 200 4 (lambda (side) 1))
\endlisp
and verify that you obtain the same snowflake curve as before.  What
happens if {\tt f} always returns $-1$?  Try an {\tt f} that returns 1
at level 0 and $-1$ otherwise, and see if you get the first shape
shown in figure~\ref{snowflake-variants}.  Explore other choices for
{\tt f}, and look for interesting shapes.  For this problem you should
turn in listings of your modified procedures.  Do not turn in any
pictures (but see the optional contest at the end of this assignment).

\begin{figure}
\vskip 2.5 in
{\small
\begin{verbatim}
level: 3                            level: 3                        level: 4
f: (lambda (side)                   f: (lambda (x)                  f: (lambda (x)
     (if (= side 0)                      (+ (* x x)                      (+ (* x x)
         1                                  (* -3 x)                        (* -.4 x)
         -1))                               5)                              5)
\end{verbatim}
}
\caption{{\protect\footnotesize
Variations on the snowflake curve.}}
\label{snowflake-variants}
\end{figure}


\subsection{Varying the angle}

Actually, there is no reason to restrict the choices for {\tt f} to
procedures that return 1 or $-1$.  Pretty much {\em any} function will
lead to symmetric designs.  Figure~\ref{snowflake-variants} shows some
examples.  Try some choices for {\tt f} and see what interesting designs
you can come up with.


\subsection{Varying the side length}

Going further, you can scale the lengths of the sides as a function of
the level by using a procedure {\tt g} and changing the final line of
{\tt side} to
\beginlisp
(4sides angle (* (g level) (/ length 3)) (- level 1) $\ldots$
\endlisp

\paragraph{Problem 2} Make this change, adding {\tt g} as a new
argument to each of your procedures and changing the calls to the
procedures accordingly.  To test your code, call {\tt snowflake} with a
{\tt g} that always returns 1, and verify that you get the same
pictures as before.  Now try other choices for {\tt g}.  You may need
to change the initial drawing point (using {\tt set-pen-at}) to get
nicely scaled pictures that fit on the screen.  Turn in a listing of
your modified procedures.


\subsection{Total length of the snowflake curve}

\paragraph{Problem 3} Write three new procedures {\tt
snowflake-length}, {\tt side-length}, and {\tt 4sides-length} that take
the same arguments as your snowflake program procedures (as modified
in problem 2).  However, instead of drawing anything, calling {\tt
snowflake-length} should return the {\em total length} (i.e., the
total distance moved by the pen) of the curve that {\tt snowflake}
draws, given the same arguments. (Hint: The three new procedures
should call each other in the same way as the original three and each
should return the length of the part of the curve that would be drawn.
Then {\tt snowflake-length} should return the sum of the results
generated by the three calls to {\tt side-length}, which should return
the sum of the results generated by the four calls to {\tt
4side-length} and so on.)  Turn in a listing of these procedures.

\paragraph{Problem 4} Give a formula for the length of the (original,
ordinary) snowflake curve of side $L$ and level $n$, and give a short
proof that formula is true.  (Hint: How many short line segments are
there? How long is each one?)  Test your formula against your answer
for problem 3 by computing the length of some snowflake curves (taking
{\tt f} and {\tt g} to be procedures that always return 1.

\paragraph{Problem 5} Use your procedures to compute the total length
of curves with side 100, levels 0 through 4, and the following choices
for {\tt g} (note that the choice of {\tt f} doesn't matter):
\begin{tightlist}
\item {\tt (lambda (x) (+ (* x x) (* 2 x) 3))}
\item {\tt sqrt}
\item The function of $x$ that is 1 if $x=0$ and $1/x$ otherwise.
\end{tightlist}
For each choice, show the answer, along with the call to Scheme that
you typed in order to compute it.

\paragraph{Problem 6} You've undoubtedly noticed that, whenever you draw
a snowflake curve, Scheme always prints {\tt PEN-MOVED} after drawing
the curve.  Explain why.

\subsection{Contest (Optional)}

Hopefully, you generated some interesting designs in doing this
assignment.  If you wish, you can make graphics prints of your best
designs and enter them in the 6.001 snowflake design contest.  The
Chipmunk manual explains how to use the graphics printers; or ask the
lab assistants for help.  Turn in your design collection together with
your homework, but {\em stapled separately}, and make sure your name
is on the work.  For each design, show the length, level, {\tt f}, and
{\tt g} that produced the result.  In order minimize demand on the
graphics printers {\em do not turn in more than four pictures}.
Designs will be judged by 6.001 staff and other internationally famous
snowflake design experts, and prizes will be awarded in lecture.


\subsection{Finally}

Please indicate the names of any students you cooperated with in doing
this assignment.

How much time did you spend in lab on this assignment?  How much time
did you spend in total on this assignment?



\end{document}
