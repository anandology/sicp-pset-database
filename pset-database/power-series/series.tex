% for yTeX
\typesize=11pt
\hsize=34pc
\vsize=50pc
\parskip 6pt plus 2pt

\def\psetheader{
\centerline{MASSACHUSETTS INSTITUTE OF TECHNOLOGY}
\vskip .25truein
\centerline{Department of Electrical Engineering and Computer Science}
\centerline{6.001 Structure and Interpretation of Computer Programs}
\centerline{Fall Semester, 1986}}

\def\code#1{\beginlisp
#1
\endlisp

\vskip .1in}

\rectoleftheader={6.001 -- Fall Semester 1986}
\rectorightheader={Problem Set 7}
\onheaders
\onfooters

\null
\vskip 1truein

\psetheader

\vskip .25truein

\centerline{Problem Set 7}

\vskip 0.25truein

\vpar
Issued:  28 October 1986


\vpar
Due: in recitation on 5 November 1986

\vpar
Reading:
\beginbullets
\bpar
Textbook: section 3.4

\bpar
Code files: {\tt PS7-PS.SCM, PS7-PSUTIL.SCM, PS7-RATNUM.SCM} (attached)
\endbullets


\chapter{1. Homework Exercises}

\vpar
Write up and turn in the following exercies from the text:

\beginbullets

\bpar Exercise 3.38 -- Left vs. right accumulation

\bpar Exercise 3.43 -- Tortuous details about delayed evaluation.  Feel
free to check your answer by running this on the computer.  But, in your
answer, explain what is going on -- don't just say what the computer
prints.

\bpar Exercise 3.62 -- Ramanujan numbers

\endbullets

\chapter{2. Laboratory:  Streams}

Streams are useful for modeling infinite mathematical objects, such as
the sequence of samples of a continuous-time signal or the sequence of
coefficients of an infinite power series.

For example, in the text (pp. 273-280) we use streams of values to model samples of
continuous-time signals.  The successive values of a signal stream are
just the successive samples of the signal; the $N$th element of the
stream is identified with the value of the signal at a time $N*DT$
(where $DT$ is the size of the time interval between samples).  We
define stream operations, such as {\tt add-streams}, {\tt scale-stream}, and
{\tt integral} that model the normal mathematical manipulations of such
signals.  Given these operations we can construct streams of samples
that approximate the solutions of differential equations---we just
translate the signal-flow diagram for the differential equation into a
set of equations involving streams.  The key to the usefulness of
streams in this application is the fact that the sequence of values
can be developed incrementally and causally---stream values depend
only on previously developed values.

Another powerful application of streams to this kind of mathematics is
to identify the $N$th element in a stream with the $N$th coefficient of an
infinite power series.  Such series can be used for all sorts of
applications, including the representation of a local solution of a
differential equation.  As with the time sequences discussed above,
the coefficients of a power series can be developed incrementally.  To
compute the $N$th coefficient of a power series from power series
ingredients (say by addition or multiplication) requires no more than
$N$ coefficients of the ingredients.

In the file {\tt ps7-ps.scm} we represent the series
$a_0 + a_1*x + a_2*x^2 + a_3*x^3 + ..$.

by the stream {\tt a0 a1 a2 a3 ...}.

Thus, as explained in section 3.4.4 of the book, we can define an
infinite stream of ones:

\beginlisp
(define ones (cons-stream 1 ones))
\endlisp

\vpar
We can then use this to represent a geometric series:

{\tt ones} represents $1 + x + x^2 + x^3 + ...$

We can form the sequence of partial sums of a series evaluated at a
given point by the following procedure (supplied in the file {\tt ps7-ps.scm}).
The sequence of partial sums is represented by an infinite stream whose first 
element is 0 and whose $N$th element is the sum of the first $N-1$ terms of the
given series.

\beginlisp
(define partial-sums
  (let ((partial-sums-helper
         (lambda (x)
           (define loop
             (lambda (series sum xn)
               (cons-stream sum
                            (loop (tail series)
                                  (add sum (mul xn (head series)))
                                  (mul xn x)))))
           loop)))
    (lambda (series x)
      ((partial-sums-helper x) series 0 1))))
\endlisp

Note that in our series operations the coefficients are manipulated
with our procedures {\tt add}, {\tt mul}, {\tt sub}, {\tt div},
rather than the system
procecures {\tt +}, {\tt *}, {\tt -}, {\tt /}.
This allows us to define generic arithmetic
operations for the coefficients in terms of the standard operations.
It is often useful to use rational number arithmetic for the
coefficients of series, as we shall see.  The file {\tt ps7-ratnum.scm}
contains
the definitions of {\tt add}, {\tt mul}, {\tt sub}, {\tt div},
etc., forming a generic
extension to Scheme arithmetic for rational numbers.

\section{Exercise 1.}
Start up a Scheme and load the files {\tt ps7-ratnum.scm},
{\tt ps7-psutil.scm}, and
{\tt ps7-ps.scm}.  Type in the definition of {\tt ones}, given above.  Use the
{\tt print-stream} procedure, defined in the file {\tt ps7-psutil.scm},
to print out
the sequence of partial sums of the series {\tt ones} evaluated at $1\over2$ 
(as a rational number). Is
the answer what you would expect from what you learned about geometric
series in high school?


As with time sequences, we may add two series together or scale a
series by a constant.  Because these operations may be performed one
coefficient at a time, we may define them as mapping operations:

\beginlisp
;;; $c*(a_0 + a_1*x + a_2*x^2 + a_3*x^3 + ...)$
;;;  $= c*a_0 + c*a_1*x + c*a_2*x^2 + c*a_3*x^3 + ...$

(define scale-series
  (lambda (c s)
    (map-stream (lambda (x) (mul c x))
                s)))
\endlisp

\beginlisp
;;; $(a_0 + a_1*x + a_2*x^2 + ...) + (b_0 + b_1*x + b_2*x^2 + ...)$
;;;  $= (a_0+b_0) + (a_1+b_1)*x + (a_2+b_2)*x^2 + ...$

(define add-series
  (lambda (s1 s2)
    (map-2-streams add s1 s2)))
\endlisp

\vpar
where the mappers (and other useful utilities) are defined in the file
{\tt ps7-psutil.scm}.  We are restricting ourselves to infinite series, so the
mappers need no end tests:

\beginlisp
(define (map-stream f s)
  (cons-stream (f (head s))
               (map-stream f (tail s))))

\pbrk
(define (map-2-streams f s1 s2)
  (cons-stream (f (head s1) (head s2))
               (map-2-streams f (tail s1) (tail s2))))
\endlisp

\section{Exercise 2.}
Define {\tt sub-series}, by analogy to {\tt add-series}.  Demonstrate that your
procedure works.


Given a stream representing an infinite series we can produce a stream
representing the indefinite integral of that series.  The indefinite
integral of a series

$$a_0 + a_1*x + a_2*x^2 + a_3*x^3 + ...$$
is
$$a_0*x + a_1*x^2/2 + a_2*x^3/3 + ...$$

We return the stream {\tt a0 a1/2 a2/3 a3/4 ...} to represent this.  Note:
this is not a legitimate representation of the correct series because
all series must begin with a constant term to make evaluation (partial
sums) start up right.  This choice means that definite integrals (such
as the examples in {\tt cos-series} and {\tt sin-series},
below) must be started
up with {\tt cons-stream} to add in the constant of integration.

\beginlisp
(define indefinite-integral
  (let ()
    (define integrate-helper
      (lambda (s n)
        (cons-stream (div (head s) n)
                     (integrate-helper (tail s)
                                       (+ n 1)))))
    (lambda (series)
      (integrate-helper series 1))))
\endlisp

\section{Exercise 3.}
    Define a procedure, {\tt derivative}, that takes a stream representing a
series as input and produces the stream representing the derivative of
that series.


Given addition, scaling, and integration of series, one can write some
remarkably concise definitions.  For example, the series expansions
for the cosine and sine functions around zero are just that they start
up with $1$ and $0$ and that each is an integral of the other (with a
random scale factor of $-1$ in one case.)

\beginlisp
(define cos-series
  (cons-stream 1
               (indefinite-integral (scale-series -1
                                                  sin-series))))
\pbrk
(define sin-series
  (cons-stream 0
               (indefinite-integral cos-series)))
\endlisp

\section{Exercise 4.}
By analogy to the construction of the sine and cosine series, above,
construct the series for $e^x$.  Remember that the key is that $de^x/dx$
is $e^x$.  Demonstrate that your series works by printing out the series
you get and by using {\tt partial-sums} to obtain the value of $e$.




\vskip 20pt

In section 3.4.5 of the text we develop methods of translating a
signal flow diagram into a procedure that returns the solution of a
differential equation.  Such a procedure returns a stream of
successive values of the function that is the solution to the
differential equation. In
fact, this gets rather complex, involving extra, unintuitive {\tt delay}s.
These problems do not show up when using streams to model the
coefficients of the power-series solution.  

By a power-series solution, we mean that given a differential equation
in $y$ (as a function of $t$), we will consider $y$ to be represented by
a power series in $t$, say

\math{y(t) = a_0 + a_1*t + a_2*t^2 + a_3*t^3 + ...}

\vpar
Then, we need to determine for what values of the coefficients the
differential equation is satisfied.


For example, consider the differential equation, $dy/dt+y=0$.  This is
equivalent to 

\math{\eqalign{\int dy&=\int -y\,dt\cr
y(t) &= y_0 - \int y(t)\,dt\cr}}
where $y_0$ is some constant initial value.  If $y$ is represented as a
power series, we may write the solution of this differential equation as:

\beginlisp
(define (decay y-init)
  (define y
    (cons-stream y-init
                 (indefinite-integral (scale-series -1 y))))
  y)
\endlisp

This procedure works (you may try it).

\section{Exercise 5.}
The analogous procedure on page 277, using streams to model
successive values of {\tt y} rather than successive coefficients of {\tt y},
has
problems that require introductions of {\tt delay} to solve.  The procedure
shown above does not have these problems.  Why?  Write a short essay
explaining this situation.  Can we do something to fix our problem on
page 277 using the ideas shown here?


\section{Exercise 6.}
In Exercise 3.56 (page 278) we consider the problem of modeling
the signal-processing system for the equation $y''-a*y'-b*y=0$.  Let us
do this with series solutions.  Write the procedure, {\tt 2nd}, that takes
as arguments the constants {\tt a}, {\tt b}, and the initial values {\tt y0} 
and {\tt dy0} and
generates the stream for the series solution for $y$.


In the file {\tt ps7-ps.scm} there is a procedure {\tt mul-series}
that produces the
product of two series.  If we have two power series, $U(z)$ and $V(z)$

$$U(z) = U_0 + U_1*z + U_2*z^2 + ...$$
$$V(z) = V_0 + V_1*z + V_2*z^2 + ...$$

\vpar
then $W(z) = U(z)*V(z)$ can be computed by Cauchy's product rule:

$$W(z) = W_0 + W_1*z + W_2*z^2 + ...$$
where
$$W_n = U_0*V_n + U_1*V_{n-1} + ... + U_n*V_0$$

Surely we could solve this for $U(z) = W(z)/V(z)$ equally well:

$$U_n = ( W_n - U_0*V_n - ... - U_{n-1}*V_1 )/V_0$$

\section{Exercise 7.}
Use this idea to write the procedure {\tt div-series} that takes two
series and produces the series expansion of the function that is the
quotient of the functions represented by the arguments.


The series solution to the simple equation, $dy/dt=f(y)$, with the
signal-flow diagram of figure 3.32 (page 277) is just

\beginlisp
(define (solve f y-init)
  (define y (cons-stream y-init (indefinite-integral dy)))
  (define dy (compose-series f y))
  y)
\endlisp

\vpar
assuming that {\tt f} is given as a power series and {\tt compose-series}
produces
the series that is the result of substituting the series that is its
second argument for the independent variable in the series that is its
first argument.

\section{Exercise 8.}
Write the procedure, {\tt compose-series}, required to make the
procedure above work.  Demonstrate that your procedure is sensible.
Explain how you derived it and illustrate its use with an example.


You may have noticed that in this problem set we have written some of
our procedures in a rather weird style.  For example, consider the
{\tt indefinite-integral} procedure above.  Surely it is equivalent to
write: 

\beginlisp
(define indefinite-integral
  (lambda (series)
    (define integrate-helper
      (lambda (s n)
        (cons-stream (div (head s) n)
                     (integrate-helper (tail s)
                                       (+ n 1)))))
    (integrate-helper series 1)))
\endlisp

Or is it?

\section{Exercise 9.}
Write a short essay explaining why we have chosen to write some of
our procedures in an obscure way.  Please do not be sarcastic -- the
answer is rather sensible.  Hint: think about space taken up by the
computation.  Use the environment model to try to understand what is
going on.
\end
