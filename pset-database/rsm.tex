\input formatting/6001mac.tex

\begin{document}

\psetheader{February 8, 1989}{A Revised Substitution Model}

\medskip
 
Section 1.1.5 of the textbook presents the {\bf Substitution Model} 
of expression evaluation. Although this model can be used to understand the
evaluation of many expressions, it does not distinguish explicitly 
between expressions and the results of evaluating the expressions. To make
the evaluation process clearer, these notes introduce a 
{\bf Revised Substitution Model (RSM)}. Bear in mind that neither 
the Substitution Model, nor the Revised Substitution Model, 
is a full description how the Scheme interpreter actually performs 
evaluations. Many details of the evaluation process---some of 
which we will discuss later in the semester---are suppressed in 
these models.

\paragraph{1. Representations of Values}

The evaluation of every (valid) Scheme expression returns a value,
which we will often refer to as an object. In the RSM, we use a box
to represent an object. The characters that are the contents of the
box specify the particular object. For example, \fbox{5} is used to
represent the object that is the number $5$. This is not what is
printed by the computer if you type {\tt 5} at it, but merely an artifice
for helping us follow the process of evaluation.

\begin{itemize} 

\item {\bf Primitive Data:} The value of a numeral is the number that
it names. To distinguish between the numeral and its value, we will
use the notation \fbox{5} for the value returned by evaluating the
numeral {\tt 5}. 

\item {\bf Primitive Procedures:} The Scheme interpreter provides a
set of primitive procedures, which form the building blocks for our
own procedures. These include the addition procedure, the
multiplication procedure, etc. We will represent a primitive
procedure object by a box containing some name for the primitive
procedure. For example, \fbox{addition} would represent the 
primitive addition procedure.

\item {\bf Compound Procedures:} 
Evaluating a {\tt lambda} special form\footnote{That is, a compound
expression that begins with the keyword {\tt lambda} followed by a
list of the formal parameters of the procedure, and ends with
the body of the procedure.} creates a procedure object. We will represent
this object by a box containing the characters {\bf Proc}, a list of
formal parameters of the procedure, and the body of the procedure.
For example, the object obtained by evaluating

\beginlisp
(lambda (x) (* x x))
\endlisp
will be  represented as \fbox{Proc (x) ($\ast$ x x)}.
\end{itemize} 

\paragraph{2. Rules of Evaluation} 

Scheme has only a few kinds of expressions.  A {\bf simple expression}
is either a constant (such as a numeral) or a symbol.

\begin{itemize} 
\item{\bf P-1.}  The value of a numeral is the number that it denotes.
The value of the numeral {\tt 5} is the number $5$, represented by \fbox{5}.

\item{\bf P-2.} The value of a symbol (or name) is the object associated
with the symbol. This value is obtained by looking up the name in a
dictionary or table and extracting the object associated with (or
bound to) that symbol. Note that we can create new bindings in the
table by evaluating the special form {\tt define}.

\item{\bf P-3.}  The value of an object (represented by characters enclosed
in a box) is the same object. In other words, once an expression has
been evaluated, and the value substituted into another expression, no
further evaluation of that value occurs.

\item{\bf P-4.} The value returned when a primitive procedure 
is applied to a set of values is the result obtained by performing
a particular built-in operation on that set of values.
\end{itemize}

A {\bf compound expression} is a sequence of expressions (simple or
compound) enclosed in parentheses. Unless it starts with one of a
very few reserved keywords used to identify special forms, a compound
expression denotes a {\bf combination} (also called a {\bf procedure
application}). The leftmost subexpression of a combination is called the
{\it operator} of the expression; the other subexpressions, the
{\it operands}. 

The Scheme interpreter evaluates combinations by the recursive
application of two basic rules. These rules serve to unwind the
evaluation of combinations until primitive procedures are applied to
objects. The resulting values are then substituted for the
corresponding expressions. Specifically, the rules are:

\begin{itemize} 

\item{\bf C-1.} To {\bf evaluate a combination}, first evaluate the
subexpressions of the combination, then apply the procedure that is
the value of the leftmost subexpression to the arguments that are the
values of the other subexpressions.

\item{\bf C-2.}  To {\bf apply a procedure} to a set of arguments,
evaluate the body of the procedure with each formal parameter of the
procedure replaced by the value of the corresponding 
argument.\footnote{If the body consists of a sequence of expressions,
each expression is evaluated in turn. The value of the final
expression is the result of the procedure application.}
\end{itemize} 

Note that Rule~C-1 does not specify an order for evaluating the
subexpressions of a combination.

\noindent A {\bf Special Form} is a compound expression
whose leftmost subexpression is a reserved keyword
such as {\tt define}, {\tt if}, {\tt cond}, and {\tt lambda}.

\begin{itemize} 
\item {\bf S.} Each special form has its own evaluation rule.
\end{itemize} 

We have already seen that evaluation of the {\tt lambda} special form
creates a procedure object that incorporates a specification of the
formal parameters and body of the procedure. 
Rules for the {\tt define} and {\tt if} special forms are presented 
in Sections 4 and 6.

\paragraph{3. A Simple Example}

In describing the evaluation of a procedure application using these
rules, we will use a special notation to distinguish the evaluation
stages from the application stages. An expression that is being
evaluated will be enclosed within curly braces $\left\{, \right\}$.
Square brackets $\left[, \right]$ are used to enclose the application
of a procedure object to its arguments.

To demonstrate the Revised Substitution Model, we first consider the
evaluation of the expression:

\beginlisp
((lambda (x) (* x x)) 5)
\endlisp
which we know should result in the process of squaring the number 5.

The following table denotes the full evaluation.  We will discuss
the process by which the evaluation evolves below.

\vskip 20pt

\begin{center} 
\begin{tabular}{lcr|c}
\{ & ((lambda (x) ($\ast$ x x)) 5)                      & \} & (1) \cr
   & \{ (lambda (x) ($\ast$ x x)) \} \qquad \{ 5 \}     &    & (2) \cr
 [ & \fbox{Proc (x) ($\ast$ x x)} \qquad \fbox{5}       & ]  & (3) \cr
\{ & ($\ast$ \qquad \fbox{5} \qquad \fbox{5} )          & \} & (4) \cr
   & \{$\ast$\} \qquad \{\fbox{5}\} \qquad \{\fbox{5}\} &    & (5) \cr
[ &   \fbox{mult}      \qquad \fbox{5}  \qquad \fbox{5} & ]  & (6) \cr
  &                         \fbox{25}                   &    & (7) 
\end{tabular} 
\end{center} 

The parenthesized numbers on the right are 
not part of the RSM itself, but merely identify the steps in the 
evaluation process so that we can refer to them. 

We begin with line~(1). Since this is an evaluation of a combination,
rule~C-1 says we must evaluate the subexpressions as shown on line
(2). The value obtained by evaluating the {\tt lambda} special form is
a procedure object, denoted by \fbox{Proc (x) ($\ast$ x x)} and the
value of the numeral {\tt 5} is the number $5$ represented by
\fbox{5}. Thus, the RSM reduces the evaluation of the expression in 
line~(1) to the application of a procedure object to a set of values, 
shown in line~(3).

Now, rule~C-2 applies: the value of the procedure argument is {\it
substituted} (hence the name of the model) for the formal parameter x
in the body of the procedure, as shown in line~(4). Note that we have
substituted the representations of the numbers directly into the
expression. The result of this substitution is a compound expression,
in particular a combination that must now be evaluated.

To evaluate this combination, rule~C-1 applies again: the
subexpressions of the combination must be evaluated, as shown in line
(5). We already have values (or objects) for the second and third
subexpressions, as indicated by the boxes. Thus rule~P-3 holds,
implying that no further evaluation of these expressions occurs.
The operator (first subexpression) of the combination is the symbol
{\tt $\ast$} which is evaluated using rule~P-2. The value of the
operator is the primitive multiplication procedure, denoted by
\fbox{mult}, and is found by using rule~P-2.

Hence, the evaluation of the expression on line~(4) reduces to the
primitive procedure application shown in line~(6). Rule~P-4 now
holds, and the entire expression reduces to \fbox{25}, as expected.

Notice how the process involves the alternation of evaluation and
application stages. We will return to this in detail later in the term.

\paragraph{4. The Define Special Form}

When evaluating a compound expression beginning with {\tt define}, such as

\beginlisp
(define {\it expression-1} {\it expression-2})
\endlisp
first evaluate {\it expression-2}, then associate its value with the
symbol that is {\it expression-1}. This association is then placed 
in a dictionary. The symbol itself is {\it not} evaluated.
The result of evaluating a {\tt define} special form is the symbol 
that is {\it expression-1}.

If the symbol is subsequently evaluated, the dictionary is searched 
and the object associated with the symbol is 
returned.\footnote{Additional details of this process will be discussed
later in the term.} For example, after we evaluate the following expressions:

\beginlisp
(define a 2)
\null
(define square (lambda (x) (* x x)))
\endlisp
the dictionary contains a binding of the symbol {\tt a} to the
number $2$ (represented by \fbox{2}), and a binding of the symbol 
{\tt square} to the procedure represented by \fbox{Proc (x) ($\ast$ x x)}.
No new evaluation
rule~is needed for the ``sugared'' variety of procedure definition:

\beginlisp 
(define (square x) (* x x))
\endlisp 
is evaluated as if it had been written
\beginlisp 
(define square (lambda (x) (* x x)))
\endlisp 

\paragraph{5.  A More Complicated Example}

After evaluating the following two expressions:

\beginlisp
(define a 2)
\null
(define square (lambda (x) (* x x)))
\endlisp

evaluation of the expression

\beginlisp
(square (1+ a))
\endlisp

proceeds as follows:

\newsavebox{\addone}
\savebox{\addone}{
\begin{tabular}{lccr}
  & \{ 1+ \}     & \{ a \}       &   \cr
[ & \fbox{add-1} & \fbox{2}      & ] \cr
  & \multicolumn{2}{c}{\fbox{3}} &   
\end{tabular} }

\vskip 10pt

\begin{center} 
\begin{tabular}{lccr|c}
\{ & ( square   &    (1+ a) )                                 & \} & (1) \cr
   & \{square\} & \{ (1+ a) \}                                &    & (2) \cr
   & & $ \left\langle \usebox{\addone} \right\rangle $        &    & (3) \cr
[  & \fbox{Proc (x) ($\ast$ x x )}  &  \fbox{3}               & ]  & (4) \cr
\{ & \multicolumn{2}{c}{($\ast$ \fbox{3} \fbox{3} )}          & \} & (5) \cr
   & \multicolumn{2}{c}{\{$\ast$\} \{\fbox{3}\} \{\fbox{3}\}} &    & (6) \cr
[  & \multicolumn{2}{c}{\fbox{mult} \fbox{3} \fbox{3}}        & ]  & (7) \cr
   & \multicolumn{2}{c}{\fbox{9}}                             &    & (8) 
\end{tabular} 
\end{center} 

Rule~C-1 governs the evaluation of line~(1): each of the two
subexpressions must be evaluated, as shown on line~(2). 
The second subexpression is a compound expression. The stages in 
the evaluation this subexpression are
grouped inside brackets and designated step~(3). Although we have included 
all the evaluation steps in this case, such detail can be omitted
if the steps are apparent.

Note that the symbols {\tt square}, {\tt a} and {\tt 1+} are 
evaluated using rule~P-2:
the value of each symbol is determined by looking up the object 
bound to it in the dictionary. The binding for {\tt 1+} is part of
the basic Scheme interpreter; bindings for {\tt square} and {\tt a} 
were created by evaluating the {\tt define} special form. 
The rest of the evaluation is similar to the example given above.

Also note that each subexpression of a combination is completely
evaluated (i.e. reduced to an object) before the application stage.
This is because Scheme uses {\it applicative} order of evaluation. 
If Scheme used {\it normal} order of evaluation, the evolution
of the evaluation process would be very different, as discussed on
p 15 of the text.

\paragraph{6. The If Special Form}

When evaluating a compound expression beginning with {\tt if} such as

\beginlisp
(if {\it predicate} {\it consequent} {\it alternative})
\endlisp
first evaluate the {\it predicate} expression. If the value
returned is true (i.e. not false), then evaluate the {\it
consequent} expression; otherwise, evaluate the {\it alternative}
expression. Never evaluate both the {\it consequent} and 
{\it alternative} expressions.

For example, consider

\beginlisp
(define abs
  (lambda (x)
    (if ($<$ x 0)
        (- x)
        x)))
\endlisp
then evaluating the expression
\beginlisp
(abs -2)
\endlisp

proceeds as follows:

\vskip 10pt

\newsavebox{\iftest}
\savebox{\iftest}{
\begin{tabular}{lcccr}
\{ & ( $<$        & \fbox{-2} & 0 )       & \} \cr
[  &  \fbox{LESS} & \fbox{-2} & \fbox{0}  & ]  \cr
   &  \multicolumn{3}{c}{\fbox{TRUE}}     &
\end{tabular} }

\begin{center}
\begin{tabular}{lccccr|c}
\{ &     \multicolumn{4}{c}{( abs  -2 )}                     & \} & (1) \cr
 [ & & \fbox{Proc (x) (if ($<$ x 0) (- x) x)} & \fbox{-2} &  & ]  & (2) \cr 
\{ & ( if & ($<$ \fbox{-2} 0) & (- \fbox{-2} ) & \fbox{-2} ) & \} & (3) \cr 
   & & $ \left\langle\usebox{\iftest} \right\rangle $  &  &  &    & (4) \cr
\{ & &                                 & ( - \fbox{-2} )  &  & \} & (5) \cr
[  & &                          & \fbox{negate} \fbox{-2} &  & ]  & (6) \cr
   &  \multicolumn{4}{c}{\fbox{2}}                           &    & (7) 
\end{tabular} 
\end{center} 

In describing this evaluation, some of the simpler steps, such as
evaluating subexpressions of combinations, have been omitted. 
The {\tt if} special form is evaluated when the body of the
{\tt abs} procedure is evaluated (step~3). Note that the predicate 
clause is evaluated first (step~4). Since this particular
predicate clause has a true\footnote{We have used \fbox{TRUE} to
denote the Boolean object corresponding to a true logical value;
\fbox{FALSE} would represent the object corresponding to a false
logical value.} value, the consequent
clause is evaluated and the alternative clause is ignored. 

The rule~for evaluating the {\tt cond} special form is 
similar to that for the {\tt if} special form. As an exercise, you
should formulate this rule~and use it to evaluate the expression 
{\tt (abs -2) } assuming that {\tt abs} had been defined as

\beginlisp 
(define (abs x)
  (cond ((> x 0) x)
        ((= x 0) 0)
        (else (- x))))
\endlisp 

\eject
\paragraph{7.  An Example with Higher Order Procedures}

Procedures that manipulate procedures are called higher order procedures.
After evaluating the following definition:

\beginlisp
(define do-5 (lambda (f) (f 5)))
\endlisp

evaluating

\beginlisp
(do-5 square)
\endlisp

proceeds as follows:

\vskip 10pt

\begin{center} 
\begin{tabular}{lcccr|c}
\{ &  ( do-5                   & square )                   & & \} & (1) \cr
[  & \fbox{Proc (f) (f 5)}   & \fbox{Proc (x) ($\ast$ x x)} & & ]  & (2) \cr
\{ & ( \fbox{Proc (x) ($\ast$ x x)} &  5 )                  & & \} & (3) \cr
[  & \fbox{Proc (x) ($\ast$ x x)} & \fbox{5}                & & ]  & (4) \cr
\{ & ( $\ast$                     & \fbox{5}  & \fbox{5} )    & \} & (5) \cr
[  & \fbox{mult}                  & \fbox{5}  & \fbox{5}      & ]  & (6) \cr
   &    \multicolumn{3}{c}{\fbox{25}}                         &    & (7)
\end{tabular} 
\end{center} 

Each subexpression in line~(1) is a symbol that has a 
procedure object bound to it in the dictionary. The 
procedure bound to {\tt do-5} is then applied to the procedure
bound to {\tt square} as shown on line~(2). Since procedures are objects,
they can be directly substituted into the bodies of other
procedures. Thus in line~(3) the procedure bound to {\tt square}
is substituted for the formal parameter {\tt f} in
the body of the procedure bound to {\tt do-5}. The remaining steps
are similar to those in the first example.

\paragraph{8. Internal Definitions}

The rules we have stipulated so far cover many
expressions. However, we have not discussed the consequences of
using block structure. Consider the following example from page 31 of
the text:

\beginlisp
(define factorial
  (lambda (n)
     (define (iter p c)
       (if (> c n)
           p
           (iter (* c p) (+ c 1))))
     (iter 1 1)))
\endlisp

The body of the procedure bound to the symbol {\tt factorial} 
consists of a sequence of two compound expressions, the first of
which is a {\tt define} special form. Each time the {\tt factorial} 
procedure is applied to an argument, these expressions are evaluated 
in order. When the internal {\tt define} special form is evaluated, 
a {\it new} procedure object is created\footnote{This procedure has 
two formal parameters {\tt p} and {\tt c}. The argument of 
the {\tt factorial} procedure application replaces the occurrence 
of parameter {\tt n} in its body.} and bound to the symbol 
{\tt iter}. To avoid confusion between the different procedure objects that
result from different applications of {\tt factorial}, 
each evaluation of the internal {\tt define} form can be regarded 
as creating its binding for {\tt iter} in a private or
local dictionary used for the particular application of the {\tt
factorial} procedure.\footnote{That is, there is not just one dictionary, 
but rather a hierarchy of dictionaries. When a symbol is evaluated, this
hierarchy is searched in a specified order and the first binding
found is used. Details of this process will be discussed in Chapter~3.}
Thus, when the second expression in the body of the {\tt factorial} 
procedure is evaluated, the binding used to determine the value of
{\tt iter} is the one contained in the local dictionary used for the 
particular application of {\tt factorial}.

\end{document}


