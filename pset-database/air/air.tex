% Copyright (c) 1990 Massachusetts Institute of Technology
% 
% This material was developed by the Scheme project at the Massachusetts
% Institute of Technology, Department of Electrical Engineering and
% Computer Science.  Permission to copy this material, to redistribute
% it, and to use it for any non-commercial purpose is granted, subject
% to the following restrictions and understandings.
% 
% 1. Any copy made of this material must include this copyright notice
% in full.
% 
% 2. Users of this material agree to make their best efforts (a) to
% return to the MIT Scheme project any improvements or extensions that
% they make, so that these may be included in future releases; and (b)
% to inform MIT of noteworthy uses of this material.
% 
% 3. All materials developed as a consequence of the use of this
% material shall duly acknowledge such use, in accordance with the usual
% standards of acknowledging credit in academic research.
% 
% 4. MIT has made no warrantee or representation that this material
% (including the operation of software contained therein) will be
% error-free, and MIT is under no obligation to provide any services, by
% way of maintenance, update, or otherwise.
% 
% 5. In conjunction with products arising from the use of this material,
% there shall be no use of the name of the Massachusetts Institute of
% Technology nor of any adaptation thereof in any advertising,
% promotional, or sales literature without prior written consent from
% MIT in each case. 

% for yTeX
\typesize=11pt
\hsize=34pc
\vsize=50pc
\parskip 6pt plus 2pt
\def\v#1{\hbox{\bf #1}}
\def\unit#1{{\v{\^#1}}}

\def\psetheader{
\centerline{MASSACHUSETTS INSTITUTE OF TECHNOLOGY}
\centerline{Department of Electrical Engineering and Computer Science}
\centerline{6.001 Structure and Interpretation of Computer Programs}
\centerline{Spring Semester, 1988}}

\def\code#1{\beginlisp
#1
\endlisp

\vskip .1in}

\rectoleftheader={6.001 -- Spring Semester 1988}
\rectorightheader={Problem Set 5}
\onheaders
\onfooters

\null
\vskip 1truein

\psetheader

\vskip .25truein

\centerline{Problem Set 5}

\vskip 0.25truein

\vpar
Issued: March 1, 1988.


\vpar
Due: in recitation on March 16 {\it for sections meeting at 9, 10 and
11} and on March 18 {\it for sections meeting at 12, 1 and 2}.

\vpar
Note: you have two weeks for this assignment because of the quiz on
March 9th.  This assignment is the same length as a normal week long
assignment.  We suggest you do the homework exercises before the quiz
and the laboratory exercises afterwards.

\vpar
Reading assignment: Finish Chapter 2.

\vskip 20pt

{\bf Quiz Announcement:}  Quiz 1 is on Wednesday, March 9.  The quiz
will be held in Walker Memorial Gymnasium (50-340) from 5--7PM or
7--9PM.  You may take the quiz during either one of these two periods,
but students taking the quiz during the first period will not be allowed
to leave the room until the end of the period.  The quiz is {\bf
partially} open book, meaning that you may bring a copy of the Scheme
manual, plus {\bf one} sheet of 8.5 by 11 inch paper, on which you may
place any crib notes you think will be useful.  The quiz will cover
material from the beginning of the semester through the material
presented in recitation on March 2, and through section 2.2 
of the textbook.  

\chapter{1. Homework exercises}

Write up and turn in the following exercises:

1. Exercise 2.27 from text

2. Exercise 2.28 from text

3. Exercise 2.29 from text


\chapter{2. Laboratory Assignment: The Airline Mergers}

This assignment asks you to integrate three different file systems
using dispatch-by-type and data-directed programming techniques.
You should read section 2.3 of the book before doing this assignment.
In addition, the structure of one of the file systems is very
similar to a Huffman-coding-tree, which is described in detail in
section 2.2.6 of the book.  You should refer to that section if
you do not understand it from reading the code we give you in the
appendix.

Use the ``load problem set'' command to load the code in the appendix
from the system onto your Chipmunk.  You will not need to modify
any of the code, but you will be adding a substantial amount of
new code.  Therefore we suggest that you put your new code in a
separate file.  Since one can only ``zap'' the code into the Scheme
environment one file at a time, you can first ``zap'' all the code
in the appendix into Scheme, then set up a new buffer and work
from there.  The ``zapped'' code stays in Scheme unless you logout
from your machine.

\section{An adaptation of a True Story of Our Time}

Recent chronic economic depressions have had significant impact upon
the airline industry.  Many small and medium-sized airlines are being
forced to merge to form conglomerates (take 14.01 - you'll
learn about these buzz words) in order to survive the high cost of
business operation.  Among these unfortunate souls are three airlines
that have decided to merge to form one single airline.
They are {\bf People-Delay Airline} (slogan - {\it fly the executive
style}
(i.e. always late)), {\bf New Jersey Air} (slogan -  {\it something SPECIAL
in the air}) and {\bf Epsilon Airline}, a Delta Airline spin-off
(slogan - {\it we achieve perfection so long as the error lies in
the neighborhood of epsilon}).  These three airlines are, however, very worried about
integrating their personnel files.   Their file systems are structured
differently, and they have so many employees that designing a
completely
new file system would require them to hire 10 data entry clerks for
3 years to enter all the data into the new system.  The CEO at 
People-Delay Airline, Al Cheapo, thinks that this is too expensive.
He has appointed you to think up a way of integrating the three
file systems without having to modify the internals of any one
system.  You, of course, know a good trick for doing this, if you have
gone to the recent lectures.

\section{Part 1}

Just when you were ready to begin work, a small fire broke
out in the computer lab at People-Delay.  The procedures for looking
up, inserting and deleting employee records for People-Delay's
personnel file were destroyed from magnetic tapes as well as from
computer memory.  Your first task is to reconstruct those
procedures.  You begin by analyzing the
structure of People-Delay's personnel file, which looks like the diagram
shown in Figure 1 (verify it in your mind!).

Notice that the names of the employees are alphabetized from
left to right.  In fact, the personnel file is structured
exactly the same as New Jersey Air's file structure, except
for one difference - the employees are alphabetically
arranged in People-Delay Airline's file, while the employees
are randomly arranged in New Jersey Air's file structure.
Reconstruct the {\tt lookup, insert} and {\tt delete} procedures for
People-Delay's personnel file.  Try to make your procedures
efficient by using the fact that employee names are alphabetized.
Name these procedures {\tt lookup-ordered, insert-ordered} and
{\tt delete-ordered}.  Test each procedure using the miniaturized file
system provided in the appendix.
When you test out the {\tt insert-ordered}
procedure, use the given constructor {\tt make-record-table} to make a
new record with salary information for Alyssa, who makes 35 thousand
dollars a year at People-Delay.

Hand in a listing of your procedures and a photo file of
a sample execution of each procedure.

\section{Part 2}


Just out of curiosity,  you want to find out what the structure
of the personnel file of Epsilon Airline looks like.  The personnel
file presented in the code is again a minaturized version of the
real file (remember that the size of the real file has caused
Al Cheapo to ask you to handle this job in the first place).
Draw a box and pointer diagram of this sample miniature file.
In 2 sentences or less, summarize how insertion of an
employee record is done.  Similarly, summarize how deletion of an
employee record is done.

\section{Part 3}

Now, you are ready to integrate the three file
systems. Remember that what you want to achieve is to have a generic
{\tt lookup}, a generic {\tt insert} and a generic {\tt delete} operator,
which, when given the right employee name and the right file (and
the right employee record, in the case of insertion), will
select the right procedure to perform the operation, depending on the
requirements of the file system.
You may realize that between the two
integration methods - {\bf dispatch-by-type} and
{\bf data-directed-programming},
one is more suitable for combining a small number of systems,
and the other is more suitable for combining a large number
of systems.  You are to decide, for this part, which
method to use, and implement that method.

Hand in a listing of your new code and a photo session
demonstrating that the generic operators work as desired.

\section{Part 4}

Al Cheapo is very pleased with your work.  
``If integrating these personnel files is such an easy task'', he says,
``then I am going to persuade more airlines to merge with us!!''
This means that you now have to use the other method
to implement the integration.

Hand in a listing of your new code and a photo session
demonstrating that the generic operators under this
new system also work.  If you picked the wrong methods
for part 3 and 4 (i.e., if you reversed the two),
but got the implementation of each one right, you
will get half the credit for part 3 and full credit
for part 4.

\section{Part 5}

You have probably noticed that for each company, the
structure of each employee record is exactly the same
as the structure of the file itself.  That is, the
structure of an employee record for the unordered
type of file is again a list of pairs, where the
car of each pair points to an identifier, and the
cdr of each pair points to the information associated
with that identifier, and the identifiers are also
unordered like the names.  Similarly, the employee
record of the ordered type has the same structure as
the ordered file itself.  The employee record of
the tree type file has also a tree structure.
In other words, we have a recursion of structure
in the files.

Taking advantage of this observation,  write three
procedures: {\tt record-lookup, record-insert} and
{\tt record-delete}, which are also generic operators.

{\tt Record-lookup} takes the name of an employee, 
the appropriate personnel file and an identifier,
which could be either the symbol {\tt salary} or
{\tt address}, and returns the salary or address of
the employee if the employee is found in the given
personnel file.

{\tt Record-insert} takes the name
of an employee, the appropriate personnel file,
an identifier and the information associated
with that identifier, and returns the new
personnel file with the information associated
with the identifier inserted in the employee's
record, if the employee is found in the
given personnel file.

{\tt Record-delete} takes the name of an employee,
the appropriate personnel file and an identifier,
and returns the new personnel file with the
information associated with the identifier
deleted (together with the identifier itself),
if the employee is found in the given personnel
file.

(Hint: Make use of the already existing
generic {\tt lookup, insert} and {\tt delete} operators.)

\section{Part 6}

After the new system has been in place for a while,  payroll
department head Ms. I. Luva Payday complains to
you that she is tired of having to remember which employee is in
which personnel file.  She asks you to link all the files together
into a master file, as diagrammed in Figure 2, and have an operator
{\tt lookup-global}, which, when given the name of the employee and the
masterfile, would automatically search through each personnel file
until it finds the employee, and return the employee's record.


Hand in a listing of your {\tt lookup-global} procedure, and a
photo session of its execution.

Thus the three airlines live happily ever after with their new
company name: No-frill Airline (slogan: ``The generic airline of the
21st Century'').


\end
