% -*- Mode: Tex -*-
% for yTeX
% Copyright (c) 1990 Massachusetts Institute of Technology
% 
% This material was developed by the Scheme project at the Massachusetts
% Institute of Technology, Department of Electrical Engineering and
% Computer Science.  Permission to copy this material, to redistribute
% it, and to use it for any non-commercial purpose is granted, subject
% to the following restrictions and understandings.
% 
% 1. Any copy made of this material must include this copyright notice
% in full.
% 
% 2. Users of this material agree to make their best efforts (a) to
% return to the MIT Scheme project any improvements or extensions that
% they make, so that these may be included in future releases; and (b)
% to inform MIT of noteworthy uses of this material.
% 
% 3. All materials developed as a consequence of the use of this
% material shall duly acknowledge such use, in accordance with the usual
% standards of acknowledging credit in academic research.
% 
% 4. MIT has made no warrantee or representation that this material
% (including the operation of software contained therein) will be
% error-free, and MIT is under no obligation to provide any services, by
% way of maintenance, update, or otherwise.
% 
% 5. In conjunction with products arising from the use of this material,
% there shall be no use of the name of the Massachusetts Institute of
% Technology nor of any adaptation thereof in any advertising,
% promotional, or sales literature without prior written consent from
% MIT in each case. 


\typesize=11pt
\hsize=34pc
\vsize=50pc
\parskip 6pt plus 2pt
\def\gobble#1{}

\def\fbox#1{%
  \vtop{\vbox{\hrule%
              \hbox{\vrule\kern3pt%
                    \vtop{\vbox{\kern3pt#1}\kern3pt}%
                    \kern3pt\vrule}}%
        \hrule}}

\def\illustratefile#1#2#3{
  \hbox to #1{
     \hfil
     \vbox to #2{
       \vfil
       \special{psfile=#3 hoffset=-306 voffset=-396}
       \vfil}
     \hfil}}

\def\psetheader{
\centerline{MASSACHUSETTS INSTITUTE OF TECHNOLOGY}
\centerline{Department of Electrical Engineering and Computer Science}
\centerline{6.001 Structure and Interpretation of Computer Programs}
\centerline{Fall Semester, 1988}}

\def\code#1{\beginlisp
#1
\endlisp

\vskip .1in}

\rectoleftheader={6.001 -- Fall Semester 1988}
\rectorightheader={Problem Set 4}
\onheaders
\onfooters

\null
\vskip 1truein

\psetheader

\vskip .25truein

\centerline{Problem Set 4}

\vskip 0.25truein

\vpar
Issued: October 6, 1988


\vpar
Due: in recitation on October 14, 1988 {\it for all sections}

\vpar
Reading: From text, Chapter 2, Sections 2.1 and 2.2, also sections
3.4--3.4.1. 

\vskip 20pt

{\bf Quiz Reminder:}  Quiz 1 is on Wednesday, October 19.  The quiz will
be held in Walker Memorial Gymnasium (50-340) from 5-7 PM or 7-9 PM.
You may take the quiz during either one of these two periods, but
students taking the quiz during the first period will not be allowed to
leave the room until the end of the period.  The quiz will cover
material from the beginning of the semester through recitation on
October 7.    This quiz will be ``semi-''open book.  This means that you
may bring a Scheme manual to the exam, and you may bring a single 8.5 by
11 inch sheet on paper, on both sides of which you may write any notes
that you think will help you during the exam. 


\chapter{1. Homework exercises}

Work the following exercises.  Write up your
solutions and submit them as part of your homework to your TA in
recitation.

\beginbullets

\bpar Exercise 4.1:  List structures.  Give combinations of {\tt car}s
and {\tt cdr}s that will pick 2 from each of the following lists:

\beginlisp
(6 5 4 3 2 1 0)
\pbrk
((6 5) (4 3) (2 1) (0))
\pbrk
(6 (5 (4 (3 (2 (1 (0)))))))
\pbrk
(6 ((5 ((4 3)) 2) 1) 0)
\endlisp

\bpar Exercise 4.2:  Finger exercises.  Suppose we define {\tt a } and
{\tt b} to be the following two lists:

\beginlisp
(define a (list 1 2))
(define b (list 3 4 5))
\endlisp

What result is printed by the interpreter in response to evaluating
each of the following expressions:

\beginlisp
(cons a b)
(append a b)
(list a b)
\endlisp

\bpar Exercise 4.3: More finger exercises.  Suppose we have defined 
{\tt a b c} and {\tt d} to have the values {\tt 4 3 2} and {\tt 1}
respectively.
 What would the interpreter print in response to evaluating each of the
following expressions?

\beginlisp
(list a 'b c 'd)
(list (list a) (list b))
(cons (list d) 'b)
(cadr '((a b) (c d) (a d) (b c)))
(cdr '(a))
(atom? (caddr '(Welcome to MIT)))
(memq? 'sleep '(where getting enough sleep))
(memq? 'time '((requires good) (management of time)))
\endlisp

\bpar Exercise 4.4: do exercise 2.29 on page 103 of Abelson and
Sussman.

\endbullets

\chapter{2. Programming Assignment: Food for Thought}

Alyssa P. Hacker has taken on a part time job as a manager for Pitch-it-Kitsch'ns,
a local fast food outfit.  Alyssa's job includes the construction of a
data base system for keeping track of information about the meals
prepared at Pitch-it.  To deal with this problem, Alyssa has begun to
design a data base system, and in the problem set you are going to help
Alyssa with this project.

The main goals of this problem set are to give you experience in
manipulating data structures, and to think about the manipulation of
list structures.  In particular, we are going to explore the use of
the concepts of {\tt map, accumulate} and {\tt filter} as general
techniques for deriving information about and from list structures.  One
of the things you should observe in doing this, is how such general
tools can help you solve problems involving lists of objects, without
worrying about the details of those lists.

Now to the details.
The basic components of Alyssa's system are two different kinds of data
structures or abstractions.  The first is a data structure for a dish,
which includes:  a name, some ingredients, the
number of calories in the dish, and the cost of the dish.  To build such
abstractions, Alyssa writes the following constructor.
\beginlisp
(define make-dish
  (lambda (name ingredients cals cost)
    (list name ingredients cals cost)))
\endlisp

An example of using this constructor is given below:
\beginlisp
(make-dish 'hamburger '((beef .25) (bun 1)) 0 0)
\endlisp

Here, the {\tt
name} is a symbol identifying the dish, {\tt ingredients} is a list of
entries, each of which
is a list of two elements, the name of the ingredient, and the amount of
a standard unit of that ingredient used in the dish.  For example, a
hamburger uses a quarter pound of beef and one bun.  Initially, the
{\tt calories} and {\tt cost} of the dish are set to zero.

The second structure to be used is one that represents information about
ingredients, and includes: a name, the number of calories
in a standard unit, and the cost per standard unit of the ingredient.
Alyssa uses the following constructor.
\beginlisp
(define make-ingredient
  (lambda (name cals-unit cost-unit)
     (list name cals-unit cost-unit)))
\endlisp

An example of using this constructor:
\beginlisp
(make-ingredient 'beef 200 .20)
\endlisp
i.e. 200 calories per pound, and 20 cents per pound.

{\bf Exercise 1:}  Although we have introduced the notion of {\bf
defabs} as a mechanism for building constructors and selectors, in this
problem set, we want to you help us build such mechanisms by hand, to
ensure that you understand their role.  Hence, complete the definitions
below to provide the needed selectors for the constructors described
above.

\beginlisp
(define dish-name \gobble{cadr)}
\pbrk
(define dish-ingredients \gobble{caddr)}
\pbrk
(define dish-cals \gobble{cadddr)}
\pbrk
(define dish-cost \gobble{caddddr)}
\pbrk
(define ingredient-name \gobble{cadr)}
\pbrk
(define ingredient-cals-unit \gobble{caddr)}
\pbrk
(define ingredient-cost-unit \gobble{cadddr)}
\endlisp


\vskip 20pt

Now, we are going to help Alyssa build procedures for analysing and
manipulating libraries of such structures.  To do this, we will make use
of some standard procedures for manipulating list structure.  These
include the following:

{\tt Map} is a general procedure that takes as arguments a procedure and
a list, and returns a new list, each of whose elements is obtained by
applying the supplied procedure to each element of the original list.

\beginlisp
(define map
  (lambda (proc lst)
    (if (null? lst)   ; if lst is empty
        '()           ; then return an empty list
        (cons (proc (car lst))   ; otherwise apply procedure to first element
              (map proc (cdr lst)))))) ; and add it to front of list
                                       ; obtained by processing rest of list
\endlisp


{\tt Filter} is a general procedure that takes as arguments a predicate
procedure and a list, and returns a new list.  The new list is obtained
by applying the predicate to each element of the original list in turn,
and including in the new list only those elements for which the
application of the predicate returned a Boolean value of true.

\beginlisp
(define filter
  (lambda (pred lst)
    (cond ((null? lst)   ; if lst is empty
           '())          ; then return an empty list
          ((pred (car lst))  ; if first element satisfies predicate
           (cons (car lst) (filter pred (cdr lst))))  ; then add to new list
          (else  ; otherwise discard from new list, and handle remainder
           (filter pred (cdr lst))))))
\endlisp

{\tt Accumulate} is a general procedure that takes as arguments a
procedure for combining values, an initial value and a list, and returns
a value, obtained by recursively applying the combining procedure to
each element in the list.

\beginlisp
(define accumulate
  (lambda (combiner init lst)
     (if (null? lst)  ; if lst is empty
         init         ; return initial value
         (combiner (car lst)  ;; otherwise combine first element
                   (accumulate combiner init (cdr lst))))))  ; with rest
\endlisp


When you load problem set 4, copies of these procedures are loaded for
you, as is an example set of dishes and ingredients.  In particular, the
symbol {\tt dishes} is bound to a list of dish structures, and the
symbol {\tt ingredients} is bound to a list of ingredient structures. 

{\bf Exercise 2:} The first thing Alyssa needs to do is make a list of
all the dishes that are served in Pitch-it.  She knows that that
information is contained within the list of dish abstractions. Using
some or all of the procedures {\tt map, filter} 
and {\tt accumulate}, write a procedure called {\tt dish-names}, which
when applied to a list of dishes, returns a list of their names.  Apply
this procedure to {\tt dishes} to test that it works correctly.

\gobble{
\beginlisp
(define dish-names
  (lambda (dish-list)
    (map (lambda (d) (dish-name d)) dish-list)))
\endlisp
or
\beginlisp
(define dish-names
  (lambda (dish-list)
   (map dish-name dish-list)))
\endlisp
}

{\bf Exercise 3:}  Alyssa also wants to be able to generate a list of
the names of the ingredients of a particular dish.  Using some or all
of the procedures {\tt map, filter} and {\tt accumulate}, write a
procedure called {\tt dish-ingred-names}, which when applied to a data
structure for a dish, returns a list of the names of the ingredients
used in that particular dish.  Apply this procedure to an element of the
list {\tt dishes} to test that it works correctly.  

\gobble{
\beginlisp
(define dish-ingred-names
  (lambda (dish)
    (map car (dish-ingredients dish))))
\endlisp
}

Using this procedure,  write another procedure, called {\tt
dish-and-ingred-names}, which takes as argument a list of dishes, and
returns a new list, each element of which is a list of two elements, the
first being the name of the dish and the second being the list of
ingredient names returned by {\tt dish-ingred-names}.

\gobble{
\beginlisp
(define dish-and-ingred-names
  (lambda (dishes)
    (map (lambda (d) (list (dish-name d)
                           (dish-ingred-names d)))
         dishes)))
\endlisp
}

{\bf Exercise 4:}  Alyssa wants to keep track of calories and cost for
her dishes. As things stand that information is not directly
available.  For most of the data structures for dishes in her data
base, the calorie and cost information has been initialized to zero.
Each dish contains information about the amount of each ingredient
used in the dish, and each ingredient data structure contains
information about the calories or cost per unit serving of the
ingredient.  Somehow, Alyssa needs to combine that information together.
She wants to write a set of
procedures that can automatically update the information about dishes
from the information about ingredients.  We will help her do this in
several stages.  

First, suppose we are given the name of an ingredient, and a list of
ingredient data structures (e.g. the value of {\tt ingredients}).  We need a
procedure, called {\tt find-ing}
that given these two arguments, will return the ingredient data
structure with the same name as the supplied argument.  Write such a
procedure, using {\tt filter}.

\gobble{
\beginlisp
(define find-ing
  (lambda (name lst)
    (car (filter (lambda (elt) (eq? name (ingredient-name elt))) lst))))
\endlisp
}

Based on this procedure, Alyssa has written another procedure that will
determine the number of calories in a dish.

\beginlisp
(define calories-of-dish
  (lambda (dish ings-list)                      
    ;arguments are a dish structure and a list of ingredient structures
    (let ((dish-ings (dish-ingredients dish)))  ; get the ingredients of the dish
      (accumulate-cals dish-ings ings-list))))  ; determine number of calories
\endlisp

In doing this, Alyssa has assumed that a procedure named {\tt
accumulate-cals} can be created that will actually determine the
number of calories in a dish.  Its arguments are a list of dish
ingredients, each element of which is a list of two elements, the name
of the ingredient and the amount of a unit portion of the ingredient
used in the dish, and a list of all the ingredient data structures.
Write the procedure, {\tt accumulate-cals}, which should recursively
``cdr'' down the list of dish ingredients, adding the product of the
portion of the ingredient used times the calories per unit portion of
the ingredient.  You should use {\tt find-ing} to get the appropriate
data structure for each ingredient.

\gobble{
\beginlisp
(define accumulate-cals
  (lambda (dish-ings ings-list)
   ;arguments are the list of ingredients for a dish,
   ;and a list of ingredient data structures
     (if (null? dish-ings)
         0
         (let ((first-ing (car dish-ings)))
           (+ (* (cadr first-ing)               ;get amount of ingredient
                 (ingredient-cals-unit (find-ing (car first-ing)
                                                 ings-list)))
                                ;get unit calories from structure
              (accumulate-cals (cdr dish-ings) ings-list))))))
\endlisp
}

Now, Alyssa needs a procedure for updating all of the dish data
structures.  Write a procedure called {\tt add-calories-to-dishes} that
takes as arguments a list of dish data structures and a list of
ingredient data structures and returns a new list of dish data
structures which now include information about the number of calories in
a dish.  For those of you who have looked ahead, {\bf DO NOT} use {\tt
set!} to solve this problem.  Instead you should build a list of new
data structures, that maintain the information in the old data
structures, but add the new information as well.

\gobble{
\beginlisp
(define add-calories-to-dishes
  (lambda (dish-list ingredients-list)
    (if (null? dish-list)
        '()
        (let ((next-dish (car dish-list)))
          (cons (make-dish (dish-id next-dish)
                           (dish-name next-dish)
                           (dish-ingredients next-dish)
                           (calories-of-dish next-dish ingredients-list)
                           0)
                (add-calories-to-dishes (cdr dish-list) ingredients-list)))))) 
\endlisp
}

Apply this procedure to {\tt dishes} to test that it works correctly.

\vskip 20pt

{\bf Exercise 5:}  We need to do the same thing for updating information
about the cost of each dish.  Write a set of procedures similar to those
above for doing this.

\vskip 20pt

{\bf Exercise 6:}
If you now look at the procedures of the previous two exercises, you
should observe a common pattern, and by now you know that as a
consequence, we are going to capture that common pattern in a higher
order procedure.  In particular, write a procedure called {\tt
accumulate-entry} which generalizes {\tt accumulate-cals} and the
corresponding procedure for cost, which you wrote for exercise 5.
Show how you would use this generalized procedure to implement {\tt
calories-of-dish}.

\gobble{
\beginlisp
(define accumulate-entry
   (lambda (dish-ings ings-list select)
      (if (null? dish-ings)
          0
          (+ (* (cadr (car dish-ings))
                (select (find-ing (car (car dish-ings))
                                  ings-list)))
             (accumulate-entry (cdr dish-ings) ings-list select)))))
\pbrk
(define calories-of-dish
  (lambda (dish ings-list)
    (let ((dish-ings (dish-ingredients dish)))
      (accumulate-entry dish-ings ings-list ingredient-cals-unit))))
\endlisp
}

\vskip 20pt

{\bf Exercise 7:}  The solutions in the previous exercises built
specialized accumulators for finding out the number of calories or the
cost of the dishes.  We have the general tool of {\tt accumulate}
available to use, however, and here we consider how to use it to do
the same thing.  Note that {\tt accumulate} operates on a list of
values, accumulating them according to the supplied procedure.  To use
{\tt accumulate}, we must first create a list of the calories
associated with each ingredient.  We can do this by first mapping an
appropriate procedure down a list of dish ingredients, and then
applying accumulate to the resulting list.  Using this idea, write a
procedure called {\tt new-accumulate-cals} that takes as arguments a
list of dish ingredients and a list of all ingredient structures, and
determines the calories associated with the dish.

\gobble{
\beginlisp
(define new-accumulate-cals
  (lambda (dish-ings ings-list)
     (accumulate + 0 
                 (map (lambda (ing)
                         (* (second ing)
                            (ingredient-cals-unit (find-ing (first dish-ings)
                                                            ings-list))))
                      dish-ings))))
\endlisp
}


{\bf Exercise 8:}  Alyssa also wants to find all the dishes that contain
a particular ingredient.  She knows that the procedure {\tt
dish-and-ingred-names} created in 
problem 3 will get her a list of elements, each of which is a list of
the name of the dish, and the list of ingredients of the dish.

Alyssa wants a new procedure,
called {\tt contains-ingredient}, that takes two arguments, the name of
an ingredient, and a list of dishes, and returns a new list containing
the names of the dishes containing the ingredient.  
Alyssa's idea is to first create a list of lists of the form
\beginlisp
(list (dish-name dish) (dish-ingred-names dish))
\endlisp 
one for each dish, by using {\tt dish-and-ingred-names}.  Then one can create a new list by
including only
those elements of the first list that contain the desired ingredient.
Finally, the elements of this list can be used to create a new list
containing the name of the dish associated with each element.  Using
some or all of {\tt map, filter} and {\tt accumulate}, write such a
procedure.  In doing this, you may find it useful to use the SCHEME
procedure {\tt memq} described on page 102 of the text. 

\gobble{
\beginlisp
(define contains-ingredient
  (lambda (ing-name dish-list)
    (map car
        (filter
          (lambda (elt)
            (memq ing-name (second elt)))
          (dish-and-ingred-names dishes)))))
\endlisp
}

{\bf Exercise 9:}   Finally, Alyssa is interested in finding all pairs
of dishes that share a common ingredient.  To do this, she needs several
components. First, she needs a procedure, called {\tt get-all-pairs},
which when given a list {\tt l} as argument, generates a list of 2-lists
(i.e. lists of two elements) consisting of all order pairs of elements
of {\tt l}.  For example,
\beginlisp 
(get-all-pairs '(1 2 3))
\endlisp
would return the list 
\beginlisp

\endlisp
Using {\tt map} write the procedure {\tt get-all-pairs}.

Next, Alyssa needs a procedure, called {\tt common-elts}, which given
two lists as arguments, returns a list of the elements common to both
lists.  Write this procedure, using {\tt map} and {\tt filter}.  You may
find it useful to use the following procedure.
\beginlisp
(define flatten
  (lambda (l)
    (accumulate append '() l)))
\endlisp

\gobble{
\beginlisp
(define common-elts
  (lambda (l1 l2)
    (flatten
      (map (lambda (elt) (filter (lambda (x) (eq? x elt))
                                 l2))
           l1))))
\endlisp
}

Finally, using {\tt get-all-pairs} and {\tt common-elts}, together with
{\tt map } and {\tt filter}, write a procedure {\tt commonality}, which
when applied to a list of dishes, returns a new list of dishes that
share a common ingredient.  The elements of the returned list each
should be a list of the two dish names and the list of common ingredients
in the dish.
\gobble{
\beginlisp
(define commonality
  (lambda (ds)
    (filter third
            (map (lambda (p)
                    (list (first (first p))
                          (first (second p))
                          (common-elt (second (first p))
                                      (second (second p)))))
                 (get-all-pairs (dish-and-ingred-names ds))))))
}

\vfill
\eject
\beginlisp
;; code for food problem set 
\pbrk
;; constructor for dishes
\pbrk
(define make-dish
  (lambda (name ingredients cals cost)
    (list name ingredients cals cost)))
\pbrk

;; constructor for ingredients
\pbrk
(define make-ingredient
  (lambda (name cals-unit cost-unit)
     (list name cals-unit cost-unit)))
\pbrk
;; sample list of dishes
(define dishes
   (list (make-dish 'hamburger '((beef .25) (bun 1)) 0 0)
         (make-dish 'cheeseburger '((beef .25) (bun 1) (am-cheese 1)) 0 0) 
         (make-dish 'swissburger '((beef .25) (bun 1) (sw-cheese 1)) 0 0) 
         (make-dish 'hotdog '((frankfurter 1) (hbun 1)) 0 0)
         (make-dish 'reg-coke '((coke-syrup .5)) 0 0)
         (make-dish 'large-coke '((coke-syrup 1)) 0 0)
         (make-dish 'doubleburger '((beef .5) (bun 1)) 0 0)))
;; note that coke has an empty list of ingredients, and that
;; the calories and cost have already been inserted into the structure
\pbrk
;; sample list of ingredients
(define ingredients
  (list (make-ingredient 'beef 200 1.50)
        (make-ingredient 'am-cheese 100 .10)
        (make-ingredient 'sw-cheese 200 .20)
        (make-ingredient 'bun 50 .15)
        (make-ingredient 'hbun 50 .10)
        (make-ingredient 'frankfurter 150 .20)
        (make-ingredient 'coke-syrup 75 .10)))
\pbrk
(define calories-of-dish
  (lambda (dish ings-list)                      
    ;arguments are a dish structure and a list of ingredient structures
    (let ((dish-ings (dish-ingredients dish)))  ; get the ingredients of the dish
      (accumulate-cals dish-ings ings-list))))  ; determine number of calories
\pbrk
;;general procedures for manipulating list structures
(define map
  (lambda (proc lst)
    (if (null? lst)   ; if lst is empty
        '()           ; then return an empty list
        (cons (proc (car lst))   ; otherwise apply procedure to first element
              (map proc (cdr lst))))))
\pbrk
(define filter
  (lambda (pred lst)
    (cond ((null? lst)   ; if lst is empty
           '())          ; then return an empty list
          ((pred (car lst))  ; if first element satisfies predicate
           (cons (car lst) (filter pred (cdr lst))))  ; then add to new list
          (else  ; otherwise discard from new list, and handle remainder
           (filter pred (cdr lst))))))
\pbrk
(define accumulate
  (lambda (combiner init lst)
     (if (null? lst)  ; if lst is empty
         init         ; return initial value
         (combiner (car lst)  ;; otherwise combine first element
                   (accumulate combiner init (cdr lst))))))
\endlisp
\end

