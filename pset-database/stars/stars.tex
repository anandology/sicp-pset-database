% -*- LaTex -*-
% Copyright (c) 1990 Massachusetts Institute of Technology
% 
% This material was developed by the Scheme project at the Massachusetts
% Institute of Technology, Department of Electrical Engineering and
% Computer Science.  Permission to copy this material, to redistribute
% it, and to use it for any non-commercial purpose is granted, subject
% to the following restrictions and understandings.
% 
% 1. Any copy made of this material must include this copyright notice
% in full.
% 
% 2. Users of this material agree to make their best efforts (a) to
% return to the MIT Scheme project any improvements or extensions that
% they make, so that these may be included in future releases; and (b)
% to inform MIT of noteworthy uses of this material.
% 
% 3. All materials developed as a consequence of the use of this
% material shall duly acknowledge such use, in accordance with the usual
% standards of acknowledging credit in academic research.
% 
% 4. MIT has made no warrantee or representation that this material
% (including the operation of software contained therein) will be
% error-free, and MIT is under no obligation to provide any services, by
% way of maintenance, update, or otherwise.
% 
% 5. In conjunction with products arising from the use of this material,
% there shall be no use of the name of the Massachusetts Institute of
% Technology nor of any adaptation thereof in any advertising,
% promotional, or sales literature without prior written consent from
% MIT in each case. 

\input ../formatting/6001mac.tex

% \special{twoside}

\evensidemargin 35pt

\begin{document}

\psetheader{Fall Semester, 1989}{Problem set 3}

\medskip

\begin{tabular}{ll}
Issued: & Tuesday 26 September \\
Due: & Wednesday 4 October \\
Reading: & Text, Chapter 2, sections 2.1 and 2.2.
\end{tabular}

\noindent
{\small Problem sets should always be handed in at recitation.  Late work will
not be accepted.}

\section{Homework exercises}

Write up and turn in the following exercises from the text:

\begin{itemize}
\item Exercise 2.3, page 84.
\item Exercise 2.6, page 86.
\item Exercise 2.7, page 86.
\item Exercise 2.11, page 87.
\end{itemize}

\section{Programming Assignment: Star Gazing}

One of Louis Reasoner's frequent habits is taking walks in the countryside late
at night -- to help him ``clear his head,'' he claims.  Unfortunately, Louis
often loses his orientation, and has some difficulty in returning to his car.
He decides he needs help in orienting himself, and plans to invent a device to
help him use the stars as a guide.

Suppose that Louis wants to know what part of the sky he is seeing.
This can be accomplished by recognizing some familiar pattern of
stars.  There are two parts to the process of using the stars for
orientation:

\begin{itemize}

\item Recognizing spots of light in the sky as known groups of stars.  
Here we match the bright spots in an image of the sky against star
charts representing traditional configurations, called constellations.

\item Determining how a recognized group is oriented with respect to
the matching chart.  Such a relation is called a transformation of
coordinates.

\end{itemize}

These two tasks are closely interrelated: it is hard to do one without
the other -- in other words, if we have solved either of the two
problems we have solved the other.

In this problem set, we are going to help Louis develop a method for
recognizing constellations, given charts of those constellations and images of
the stars.  The methods developed here actually have less ``blue sky''
applications: We can think of the ``spots'' as feature points in an image of
something other than the sky.  These feature points are places in the image
that are especially easy to identify, such as the corners of polygonal regions
of constant brightness.  The images might be of a flat industrial part on a
conveyor belt, for example, taken with a camera that is oriented with its
optical axis perpendicular to the belt.  In fact, the methods we are going to
explore in this problem set are a simple version of a general purpose object
recognition system called RAF, developed by Eric Grimson and Tom\'as
Lozano-P\'erez.

Louis begins by trying to understand his data.  He calls the points of
light in an image of the sky ``spots'' to distinguish them from the
``stars'' on a chart.  Each star on a chart has a name, and Louis chooses
to give each spot on an image an identifying number.

To simplify the algebra, Louis pretends that the sky is flat.  He
describes the position of each spot in the sky and each representation
of a star on a chart using two coordinates.

In addition, every point of light in an image and every star on a chart
has a brightness, measured in magnitudes (to be explained later).

For example, a simple star chart might contain the following entries:
\[
\begin{array}{llll}
Name&X&Y&Magnitude\\
alpha&0.0&1.0&0.0\\
beta&0.0&0.0&0.5\\
gamma&1.5&1.0&1.0\\
delta&1.0&2.0&2.0
\end{array}
\]
and a simple image might look as follows:
\[
\begin{array}{llll}
Number&X&Y&Brightness\\
1&1.03&1.97&0.2\\
2&1.32&3.75&1.2\\
3&0.25&2.75&0.3
\end{array}
\]

Louis begins by defining constructors for these abstractions:

\beginlisp
(define make-star
  (lambda (star-name star-position star-magnitude)
    (list star-name star-position star-magnitude)))
\null 
(define make-spot
  (lambda (spot-number spot-position spot-brightness)
    (list spot-number spot-position spot-brightness)))
\null
(define make-position
   (lambda (x y) (list x y)))
\endlisp

{\bf Exercise 1:}  Louis represents a star chart as a list of stars.
Louis makes a simple test chart, called {\tt trial-stars-1}, follows:
\beginlisp
(define trial-stars-1
  (list (make-star 'alpha (make-position 0.0 1.0) 0.0)
        (make-star 'beta (make-position 0.0 0.0) 0.5)
        (make-star 'gamma (make-position 1.5 1.0) 1.0)))
\endlisp
Draw a box-and-pointer diagram for the list structure
corresponding to {\tt trial-stars-1}.

\vskip .5in

{\bf Exercise 2} Given these constructors, we need to define
appropriate selectors for each of the components of the compound data
items.  Complete the definitions for the selectors below:

\beginlisp
;;; Given a star, we should be able to get its components
\null
(define star-name ...)                               ;should get star's name
% (lambda (star) (car star)))
\null
(define star-position ...)                           ;should get star's position vector
% (lambda (star) (cadr star)))
\null
(define star-magnitude ...)                          ;should get star's magnitude
% (lambda (star) (caddr star))
\null
\endlisp

\beginlisp
;;; Given a spot, we should be able to get its components
\null
(define spot-number ...)                             ;should get spot's number
% (lambda (spot) (car spot)))
\null
(define spot-position ...)                           ;should get spot's position vector
% (lambda (star) (cadr star)))
\null
(define spot-brightness ...)                         ;should get spot's v coord
% (lambda (spot) (caddr spot)))
\endlisp

\null

Louis now attacks the problem of matching a list of spots to a list of stars.
Louis calls a particular matching, mapping every spot observed to a star (from
his chart), an {\it interpretation}.  Louis's idea is that he can first develop
all possible assignments of spots to stars, and later filter them for
reasonableness.  His first draft of a matcher is thus nothing more than a
program that makes (a list of) all possible interpretations of an image (a list
of spots) relative to a chart (a list of stars), without taking into account
any of the constraints implied by positions and brightnesses.  Below we find
Louis's first draft:

\beginlisp
(define (all-interpretations image chart)
  (if (null? image)
      (list the-empty-interpretation)   ;Note1
      (let ((spot-to-match (first image))
            (spots (rest image)))
        (define (match-all stars)
          (if (null? stars)
              the-empty-list            ;Note2
              (append (extend-interpretations
                        (pair-up spot-to-match (car stars))
                        (all-interpretations spots
                                             (delete (car stars) chart)))
                      (match-all (cdr stars)))))
        (match-all chart))))
\null
(define (extend-interpretations pair interpretations)
  (define (extend-all interps)
    (if (null? interps)
        the-empty-list
        (cons (augment-interpretation pair (car interps))
              (extend-all (cdr interps)))))
  (extend-all interpretations))
\null
;;; My data abstraction implementations are below this line.
\null
(define the-empty-list '())     ;Ben B. says this is good magic.
(define first car)
(define rest cdr)
\null
(define the-empty-interpretation the-empty-list)
(define augment-interpretation cons)
\null
(define pair-up list)
(define pair-spot car)
(define pair-star cadr)
\endlisp

The set of interpretations of a set of spots relative to a set of stars is
formed as follows.  The first spot can be assigned to any star.  This
can be the beginning of any interpretation that matches the rest of
the spots to the stars not yet used.

Louis has marked a few subtle points: Note1 reminds us that there is
one interpretation of no spots, the empty interpretation!  Note2
reminds us that if there are spots to be matched but no stars left on
the chart there are no consistent interpretations.

When you load problem set 3 into your Chipmunk, copies of these procedures are
also loaded into your editor, for your convenience, as are some utility
procedures for printing out interpretations, {\tt print-match} and {\tt
print-matches}.  In addition, an example list of stars, called {\tt
trial-stars-1}, and an example list of image spots, called {\tt trial-spots-1},
are also loaded.  You may also find the procedure {\tt print-matches} useful
for printing out the results of finding the matches in a readable form.

You can try it out:
\beginlisp
(all-interpretations trial-spots-1 trial-stars-1)
\endlisp
You will see that Louis's procedures will find all possible ways of
matching stars with spots, without any constraint on the matches.  So
if there are $m$ stars and $n$ spots (with $m\geq n$), we will
generate $\frac{m!}{(m-n)!}$ interpretations.  These are all the ways
of picking spots to pair up with stars.  When $m=n$, this procedure
provides us with a convenient way of generating all $m!$ permutations
of $m$ elements.

Louis needs to pick out the right matching.  He could do this by
checking all of the possible interpretations for the best one, but
because factorials grow very fast he needs some way of reducing the
number of interpretations that will be generated.

One way that Louis can reduce the number of possible interpretations
is to take brightness into account when matching stars and spots.
Stars have different magnitudes (note that in astronomy, the magnitude
of a star is a measure of its brightness, and is measured on a
logarithmic scale, where a difference of five magnitudes corresponds
to a ratio of 100 in brightness).  Correspondingly, spots in the image
will have different brightnesses.  Measurements of brightness cannot
be made with perfect accuracy, so we have to allow for the possibility
of a small difference when we test whether a spot could match a star.

{\bf Exercise 3:} Write a procedure called {\tt similar-magnitudes?},
which takes as arguments a star and a spot, and which evaluates to
true if the absolute difference between the magnitude of the star and
the brightness of the spot is less than the value of a global
variable, {\tt magnitude-tolerance}, (which we have defined initially
to be 1.0 in the code file).  Be sure to use appropriate abstractions
and their relevant selectors and/or constructors.

%(define similar-magnitudes?
%  (lambda (star spot)
%    (let ((star-m (star-magnitude star))
%          (spot-m (spot-brightness spot)))
%      (< (abs (- star-m spot-m)) magnitude-tolerance))))
%
%(define magnitude-tolerance 1.0)

\null

To use this information requires modification of the first-draft
interpretation generator.  The current code assumes that any pairing
is acceptable.  

{\bf Exercise 4:}

You are to change Louis's code to require that for a spot to be paired
with a star, they must have similar magnitudes.  The change required
is very simple, only requiring the addition of a few lines of code to the
procedure {\tt all-interpretations}.  

Try out your new matcher using the lists of stars and spots you
created in Exercise 1.  You should notice that adding the constraint
of similar magnitude can decrease the number of matches explored,
provided that the error in brightness measurements is small enough.
In fact, if brightness could be measured with unlimited accuracy,
Louis's job would be done, since a star's magnitude would then be a
unique identifying characteristic.  In practice, however, measurements
of brightness have limited accuracy and the procedure may find several
potential matches that each pass the brightness comparison tests.

Hand in your modified code and demonstrate, using appropriate
examples, that it does ``the right thing''.

%(define (all-interpretations image chart)
%  (if (null? image)
%      (list the-empty-interpretation)  ;here it is!
%      (let ((spot-to-match (first image))
%            (spots (rest image)))
%       (define (match-all stars)
%         (if (null? stars)
%             the-empty-list            ;There are no interpretations consistent with no stars.
%             (if (not (similar-magnitudes? spot-to-match (car stars))))
%                 (match-all (cdr stars))
%                 (append (extend-interpretations (pair-up spot-to-match (car stars))
%                                                 (all-interpretations spots
%                                                                      (delete (car stars) chart)))
%                         (match-all (cdr stars))))))
%        (match-all chart))))

\null

We need to further prune the number of possible interpretations, and we can do
so by considering information about the spatial relationship between the stars
in a constellation.  So far, we have only used a unary test, one that applies
to a single star and a single spot.  Now let use consider a binary test.  The
distance between stars is not changed by rotation or translation in the image
plane.  So distance is independent of the coordinate transformation we are
seeking, and thus comparison of distances between stars and corresponding spots
gives us a convenient additional filtering test for proposed partial matches.
Note that this binary test is more expensive than the unary brightness test,
since we have to compare the distances between the proposed new star and all of
the stars already included in the partial match with the corresponding
distances between the proposed new spot and all of the spots already included
in the partial match.  If these distances can be measured with high accuracy,
however, then this test will remove almost all erroneous partial matches.

{\bf Exercise 5:}
We need procedures for computing distances between spots and between
stars.  Write a procedure, {\tt distance}, that computes the distance
between two two-dimensional points on a plane.  Also
write a procedure, {\tt similar-distances}, which takes as arguments two
pairs, each of which is a pairing of a spot and a star, and which tests
whether the absolute difference in distances between the respective
stars and spots is less than a globally specified {\tt
distance-tolerance} defined as follows:

\beginlisp
(define distance-tolerance 0.5)
\endlisp

%\beginlisp
%(define distance 
%  (lambda (pos1 pos2)
%     (let ((dx (- (position-x pos1) (position-x pos2)))
%           (dy (- (position-y pos1) (position-y pos2))))
%       (sqrt (+ (* dx dx) (* dy dy))))))
%\pbrk
%(define similar-distances
%  (lambda (pair-a pair-b)
%     (let ((star-a (pair-star pair-a))
%           (spot-a (pair-spot pair-a))
%           (star-b (pair-star pair-b))
%           (spot-b (pair-spot pair-b)))
%       (let ((dstars (distance (star-position star-a)
%                               (star-position star-b)))
%             (dspots (distance (spot-position spot-a)
%                               (spot-position spot-b))))
%         (< (abs (- dstars dspots)) distance-tolerance)))))
%\endlisp

\null

{\bf Exercise 6:} We now want to use the distances between pairs of
stars and the corresponding pairs of spots to prevent the generation
of impossible matches.  Given a partial (or current) interpretation
and a new pairing of a star and a spot, we want to ensure that the
distance between the new star and each of the stars in the
interpretation is consistent with the distance between the new spot
and each of the spots in the interpretation.  Write a procedure called
{\tt distances-check}, that takes a new pair (consisting of a new star
and a new spot) and a current interpretation, (consisting of a list of
such pairs), and returns true if the new pair has similar distances
with all the pairs in the interpretation, and false otherwise.

%(define distances-check
%  (lambda (current-pair interp)
%     (cond ((null? interp) true)
%           ((similar-distances current-pair (first interp))
%            (distances-check current-pair (rest interp)))
%           (else false))))

\null

{\bf Exercise 7:} Using these procedures, we need to modify Louis's
code to incorporate this distance constraint.  The change required is
very simple, only requiring the addition of a few lines of code to the
procedure {\tt extend-interpretations}.  You may also find it
instructive to see how the number of matches changes with the amount
of error allowed in the match.  Try varying the global parameter {\tt
distance-tolerance}, which initially has the value $0.5$, making this
value smaller until only one match of stars to spots is possible.
What is the actual match in this case?

\null

If the accuracy of image position measurement is high, the procedures we have
built will typically find the correct match, and only the correct match, unless
the constellation has an unusually symmetric spatial pattern.  If, on the other
hand, the accuracy is limited, there may be several different matches between
stars and spots that pass the pairwise distance test.  Also, some of the
matches that pass the pairwise distance test may not make sense globally.
Consider, for example, a match between a constellation and some spots arranged
in a pattern that is a mirror-image of the constellation.  There is in general
no rotation and translation that will bring the constellation and the set of
spots into alignment, yet all of the pairwise distances will match.  

We use $u$ and $v$ as coordinates in an image, while $x$ and $y$ are
coordinates in some global coordinate system in which all of the
constellation models are described.  In our flat sky model, the
transformation from one to the other involves a rotation and a
translation:
\[
\begin{array}{lll}
x & = & c\, u - s\, v + x_0 \\
y & = & s\, u + c\, v + y_0
\end{array}
\]
where $c= \cos \theta$ and $s = \sin \theta$ where $\theta$ is the
angle between corresponding axes in the two coordinate systems.
Further, $x_0$ and $y_0$ are the offsets of the origin of the image
coordinate system in the model coordinate system.  This transformation
is illustrated in Figure 1. Part of our matching task is the recovery
of the transformation parameters, $c, s, x_0$ and $y_0$.

\begin{figure}
\vspace{2.5 in}
%\illustratefile{3.27in}{1.02in}{/ab/welg/ps3-fig1.ps}
\caption{To transform a set of points in image coordinates into
model coordinates, we first rotate the points by an angle $\theta$, and
then translate the result.}
\end{figure}

This means
that when we have a possible matching of all the stars to spots, we should
estimate the transformation from the image to the model and then verify that
this transformation maps all of the spots to positions near the stars that they
have been matched with.  A transformation returns $c, s, x_0$ and $y_0$.

We have provided a procedure called {\tt compute-transformation}, that will
compute such a transformation, given as arguments {\tt xa ya xb yb ua va ub
vb}, where {\tt xa, ya} are the x and y coordinates of one star, {\tt ua, va}
are the u and v coordinates of the matching spot, and {\tt xb, yb, ub, vb} are
the coordinates for a second matching star and spot.  {\tt
Compute-transformation} returns a list of {\tt c, s, x0, y0} corresponding to
the components of the computed transform.

Further we have provided a procedure called {\tt map-uv-to-xy} that,
given as arguments {\tt u} and {\tt v} as the coordinates of a point
in an image, plus a transform (as returned by {\tt compute-transformation}), 
returns a pair (the result of using a {\tt cons}) giving the
coordinates
for the transformed point in the sky's
coordinate system, obtained by transforming the image point by the
transform.

{\bf Exercise 8:} To guarantee that the matching of spots to stars is correct,
we need to use the computed transform to test that each image spot, when
transformed by the transformation, is close to its matching star.  Write
a procedure called {\tt check-matched-pairs}, which takes as arguments an
interpretation (i.e. a list of matches of star and spot) and a transformation,
and returns true if for all of the matches, the transformed image spot is
within {\tt distance-tolerance} of its corresponding model star.

\null

{\bf Exercise 9:} Change your matcher to include the procedures {\tt
compute-transformation} and {\tt check-matched-pairs} to filter
the list of proposed interpretations so that any that are returned are
in fact globally correct.

\end{document}
