% Copyright (c) 1990 Massachusetts Institute of Technology
% 
% This material was developed by the Scheme project at the Massachusetts
% Institute of Technology, Department of Electrical Engineering and
% Computer Science.  Permission to copy this material, to redistribute
% it, and to use it for any non-commercial purpose is granted, subject
% to the following restrictions and understandings.
% 
% 1. Any copy made of this material must include this copyright notice
% in full.
% 
% 2. Users of this material agree to make their best efforts (a) to
% return to the MIT Scheme project any improvements or extensions that
% they make, so that these may be included in future releases; and (b)
% to inform MIT of noteworthy uses of this material.
% 
% 3. All materials developed as a consequence of the use of this
% material shall duly acknowledge such use, in accordance with the usual
% standards of acknowledging credit in academic research.
% 
% 4. MIT has made no warrantee or representation that this material
% (including the operation of software contained therein) will be
% error-free, and MIT is under no obligation to provide any services, by
% way of maintenance, update, or otherwise.
% 
% 5. In conjunction with products arising from the use of this material,
% there shall be no use of the name of the Massachusetts Institute of
% Technology nor of any adaptation thereof in any advertising,
% promotional, or sales literature without prior written consent from
% MIT in each case. 

% for yTeX
\typesize=11pt
\hsize=34pc
\vsize=50pc
\parskip 6pt plus 2pt
\def\v#1{\hbox{\bf #1}}
\def\unit#1{{\v{\^#1}}}

\def\psetheader{
\centerline{MASSACHUSETTS INSTITUTE OF TECHNOLOGY}
\centerline{Department of Electrical Engineering and Computer Science}
\centerline{6.001 Structure and Interpretation of Computer Programs}
\centerline{Spring Semester, 1988}}

\def\code#1{\beginlisp
#1
\endlisp

\vskip .1in}

\rectoleftheader={6.001 -- Spring Semester 1988}
\rectorightheader={Problem Set 3}
\onheaders
\onfooters

\null
\vskip 1truein

\psetheader

\vskip .25truein

\centerline{Problem Set 3}

\vskip 0.25truein

\vpar
Issued: February 18, 1988


\vpar
Due: in recitation on Febrary 24 {\it for sections meeting at 9, 10 and
11} and on February 26 {\it for sections meeting at 12, 1 and 2}.


\vpar
Reading assignment:  Chapter 2, Section 2.1, 2.2

\chapter{1. Homework exercises}

Write up and turn in the following exercises:

1. Exercise 2.3 from text

2. Exercise 2.5 from text

3. Exercise 2.6 from text

4. Exercise 2.7 from text

5. Exercise 2.11 from text


\chapter{ 2. Lab Exercises}

It often gets tiresome to write simple data abstractions over and over
again.  Therefore many languages provide a data abstraction facility
which takes must of the repetitiveness out of using abstraction.
We will experiment with such an abstraction facility in the lab.

Begin by installing the code for problem set 3 on your floppy disk,
using the {\tt load-problem-set} operation documented in the Chipmunk
manual.  The file to be loaded contains all of the code listed at the
end of the problem set, in file PS3.SCM.  You will not need to edit or
modify any of it.  The ``modifications file'' for problem set 3 that
will be installed on your disk is an empty file, since for this problem
set you will be writing new procedures, rather than modifying procedures
that we supply.

\section{Part 1}

Read the first page of comments in PS3.SCM to see how the data
abstraction abstraction works.

Help out Johnson Space Center Data Disservices by defining an astronaut
abstraction.  An astronaut has a weight (in pounds), a height (in
inches), an age (in years), and a GPA of 4.0 (usually).  To save
the data enterer's time, make sure that the GPA defaults to 4.0.

\section{Part 2}

Define variables, SALLY, JOHN, BUZZ, KATHY and DEKE whose values
are instances of the astronaut abstraction, and which describe the
following astronauts:

\vskip 20pt

\centerline{\begintable [|l|l|l|l|l|]
\topline
&name&&weight&&height&&age&&GPA&\cr
\midline
&SALLY&&120&&66&&34&&4.0&\cr
\midline
&JOHN&&150&&69&&31&&4.0&\cr
\midline
&BUZZ&&178&&70&&53&&3.5&\cr
\midline
&KATHY&&115&&65&&33&&4.0&\cr
\midline
&DEKE&&183&&69&&57&&4.0&\cr
\botline
\endtable }

\section{Part 3}

All crews on the leaner meaner Shuttle program are going to be
restricted to three Astronauts.  There are strict mass restrictions on
lift off weight so that the engines will not need to run at maximum
thrust.  Write a procedure VALID-MISSION? which takes four arguments:
COMMANDER, PILOT, MISSION-SPECIALIST, and CREW-MASS, and returns
true only if the three chosen astronauts do not exceed the mass
restriction.  Test your procedure with:


\beginlisp
(valid-mission? sally deke john 470)
\endlisp

\noindent and

\beginlisp
(valid-mission? buzz deke kathy 470)
\endlisp

\section{Part 4}

The budget at NASA is really tight, and space suit cost is proportional
to the square of astronaut height.  There will be a restriction on the
total space suit cost for a mission.  The President is also worried
about the world view of the astronuat corps and has ordered that there
will be a certain minimum total GPA for every mission.

Write a Higher Order Procedure MAKE-COSTING-PROCEDURE
that takes a field name and a procedure
to be applied to the contents of that field, and returns a procecedure
that sums over three astronauts.  For instance:

\beginlisp
==> ((make-costing-procedure square 'height) sally deke john)
13878
\endlisp

Now modify VALID-MISSION? to take additional arguments, MAX-SUIT-COST
and ACCUM-GPA.

\section{Part 5}

Data Diservices decides that the salary of each astronaut should be kept
in their record too.  NASA has told them there will be a cap on total
salary costs per mission.  Modify your VALID-MISSION? procedure to take
an additional argument, TOTAL-SALARY, and to take that into account in
validating a crew configuration.  What else needs to be changed to make
this work?

\end
