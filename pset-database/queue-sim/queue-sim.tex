% Copyright (c) 1990 Massachusetts Institute of Technology
% 
% This material was developed by the Scheme project at the Massachusetts
% Institute of Technology, Department of Electrical Engineering and
% Computer Science.  Permission to copy this material, to redistribute
% it, and to use it for any non-commercial purpose is granted, subject
% to the following restrictions and understandings.
% 
% 1. Any copy made of this material must include this copyright notice
% in full.
% 
% 2. Users of this material agree to make their best efforts (a) to
% return to the MIT Scheme project any improvements or extensions that
% they make, so that these may be included in future releases; and (b)
% to inform MIT of noteworthy uses of this material.
% 
% 3. All materials developed as a consequence of the use of this
% material shall duly acknowledge such use, in accordance with the usual
% standards of acknowledging credit in academic research.
% 
% 4. MIT has made no warrantee or representation that this material
% (including the operation of software contained therein) will be
% error-free, and MIT is under no obligation to provide any services, by
% way of maintenance, update, or otherwise.
% 
% 5. In conjunction with products arising from the use of this material,
% there shall be no use of the name of the Massachusetts Institute of
% Technology nor of any adaptation thereof in any advertising,
% promotional, or sales literature without prior written consent from
% MIT in each case. 

\beginone {6.[???]} {Simulation of a Queueing System}
%derived from Spring '82 PS7

\noindent
In this section we shall implement a simulation of a {\it multi-server
queueing system}.  This models phenomena associated with customers
waiting in line to obtain a service that is provided by any
one of a number of identical servers.  New customers are constantly
entering the system with requests for service that take various
amounts of time.  We will model a setup in which a separate queue is
formed for each server, and a customer entering the system joins the
shortest queue.  This is the typical organization of many kinds of
service facilities, for example, most banks, where the servers are the
tellers and the queues are the lines formed at the individual tellers'
windows.  Other such facilities are post offices, supermarkets, and so on.

\parindent10pt
Our simulation will provide information about the amount of time
customers spend waiting in line, and will allow us to determine the
effect upon this waiting time of adding additional servers, changing
the rate at which new customers enter the system, or changing the rate
at which requests are processed.\ft{The branch of mathematics call
{\it queueing theory} has developed purely theoretical techniques for
answering many of these same questions.  A discussion of these methods
is outside the scope of this book.}

\beginthree{Using the simulation}

The simulation is run by calling the procedure {\tt simulate}, which
takes four inputs.  The first is the number of servers in the system.
The second input is the probability (a number between 0 and 1) that a
new customer will enter the system during each time period.  The third
input is a list specifying the distribution of times for service
requests.  The service time for a given customer is obtained by
choosing a number at random from this list.  (For example, a list {\tt (2
3 4 4 8)} implies that the time needed to serve a customer will be 2
with 20\% probability, 3 with 20\% probability, 4 with 40\% probability
and 8 with 20\% probability.)  The fourth input specifies the number of
time units for which the simulation is to run.

Each time a customer leaves the system, the simulator prints a message
giving the time the customer entered the system,
the time the customer left the system,
the amount of service requested, and the time spent waiting in
line.  Here is a typical portion of the output for a system with 3
servers and a 90\% probability of a customer entering during each time
period:

\code
==> (simulate 3 0.90 '(2 3 4 4 8) 50)
(enter 2 leave 5 service 3 wait 0) 
(enter 4 leave 6 service 2 wait 0) 
(enter 7 leave 9 service 2 wait 0) 
(enter 5 leave 13 service 8 wait 0) 
(enter 9 leave 13 service 4 wait 0) 
(enter 6 leave 14 service 8 wait 0) 
(enter 8 leave 15 service 2 wait 5) 
(enter 13 leave 17 service 4 wait 0) 
$\vdotss$
(enter 35 leave 40 service 3 wait 2) 
(enter 37 leave 43 service 3 wait 3) 
(enter 36 leave 44 service 8 wait 0) 
(enter 34 leave 44 service 4 wait 6) 
(enter 38 leave 48 service 4 wait 6) 
(enter 39 leave 48 service 4 wait 5) 
(enter 41 leave 50 service 2 wait 7) 
\uncode

Observe how much worse the waiting time becomes if we
decrease the number of servers to 2:

\code
==> (simulate 2 0.90 '(2 3 4 4 8) 50)
(enter 2 leave 4 service 2 wait 0) 
(enter 1 leave 5 service 4 wait 0) 
(enter 3 leave 9 service 4 wait 2) 
(enter 5 leave 11 service 2 wait 4) 
(enter 4 leave 12 service 8 wait 0) 
(enter 8 leave 14 service 3 wait 3) 
(enter 7 leave 16 service 4 wait 5) 
(enter 9 leave 18 service 4 wait 5) 
$\vdotss$
(enter 21 leave 38 service 4 wait 13) 
(enter 20 leave 41 service 8 wait 13) 
(enter 22 leave 42 service 4 wait 16) 
(enter 24 leave 45 service 4 wait 17) 
(enter 26 leave 46 service 4 wait 16) 
(enter 25 leave 49 service 4 wait 20) 
(enter 29 leave 50 service 4 wait 17) 
\uncode

\beginthree{Structure of the simulation program}

The system we are modeling has two types of communicating
objects---customers and servers---which we will model as procedures with local
state.  In addition to customer and server objects, of which there can
be any number, we will have an object that provides new customers to
the system.

By organizing the model in terms of independent objects, we make it very
flexible.  For instance, it is a straightforward matter to modify the
simulation to allow servers or customers who behave differently from
the typical ones, or to allow servers as well as customers to
constantly enter and leave the system.   We can easily
arrange to print out additional information, such as the average
waiting time of all customers in the system or an indication of which
server served which customer.  We could also have more than one
customer source, each with its own characteristic probability
of a customer entering and distribution of customer request times.

\beginthree{Customers}

{\tt Make-customer} creates a customer object.  Its arguments specify
the time the customer arrives in the system and the service time
needed by the customer.  The customer object itself is a procedure
(the internal procedure {\tt customer} below) with local state (the arguments of
{\tt make-customer}).  After {\tt make-customer} creates the customer
object, it asks the server with the shortest queue to serve the customer.

\code
(define (make-customer arrival-time time-to-service)
  (define (customer message)
    (cond ((eq? message 'how-long-do-you-need)
           time-to-service)
          ((eq? message 'done)
           (let ((time (get-current-time)))
             (print
              (list 'enter arrival-time
                    'leave time
                    'service time-to-service 
                    'wait (- time (+ arrival-time
                                     time-to-service))))))
          (else
           (error "unknown message -- customer" message))))
  (request-service (shortest-queue-server available-servers)
                   customer)
  customer)
\uncode

Servers communicate with customers by using the following procedures:

\code
(define (customer-service-time customer)
  (customer 'how-long-do-you-need))

(define (customer-finished customer)
  (customer 'done))
\uncode\noindent
 A customer replies to the message {\tt how-long-do-you-need} by
giving its required service time.  The {\tt done} message prompts the
customer to print its arrival time, the current time, the service
time, and the time spent waiting (which is computed from the other
three times).

The server with the shortest queue is identified by the following
procedure, which is called with a list of servers as input.  It first
uses {\tt mapcar} to
form a list of pairs, each consisting of a server together with the
length of its queue.  Then it calls a procedure
{\tt pair-with-smallest-car}, which takes a list of pairs as argument
and returns the pair that has the smallest {\tt car} (in this case, the
smallest queue length).  The appropriate server is the {\tt cdr} of the
resulting pair.

\code
(define (shortest-queue-server server-list)
  (cdr
   (pair-with-smallest-car
    (mapcar (lambda (server)
              (cons (server-queue-length server)
                    server))
            server-list))))

(define (pair-with-smallest-car pairs)
  (cond
   ((= (length pairs) 1) (car pairs))
   (else (let ((smallest-in-cdr
                (pair-with-smallest-car (cdr pairs))))
           (cond
            ((< (caar pairs) (car smallest-in-cdr))
             (car pairs))
            (else smallest-in-cdr))))))
\uncode

\beginthree{Servers}

A server is a procedure (the internal procedure {\tt me} below) with
a local state variable {\tt my-queue}, which is a {\it queue} of
customers (initially empty).  (Queues and the operations on queues
are described in section 3.3.2 of the text.)
The first customer on the queue is the one currently being served.

\code
(define (make-server)
  (let ((my-queue (make-queue)))
    (define (serve customer)
      (after-delay (customer-service-time customer)
                   service-completed))
    (define (process-serve-request customer)
      (let ((queue-was-empty (empty-queue? my-queue)))
        (insert-queue! my-queue customer)
        (if queue-was-empty
            (serve (front my-queue)))))
    (define (service-completed)
      (customer-finished (front my-queue))
      (delete-queue! my-queue)
      (if (not (empty-queue? my-queue))
          (serve (front my-queue))))
    (define (me message)
      (cond ((eq? message 'how-long-is-your-queue)
             (length-queue my-queue))
            ((eq? message 'serve-me) process-serve-request)
            (else (error "unknown message -- server" message))))
    me))
\uncode

When a server receives a {\tt how-long-is-your-queue} message
(because a new customer is deciding which queue to join)

\code
(define (server-queue-length server)
  (server 'how-long-is-your-queue))
\uncode\noindent
 it responds with the length of its queue.  (In order to be able to
find the length of a queue, we will add one new operation,
{\tt length-queue}, to the queue
operations of section 3.3.2.)

The server's local procedure {\tt process-serve-request} is called
with a customer object as argument when a customer requests service:

\code (define (request-service server customer)
  ((server 'serve-me) customer))
 \uncode\noindent
 The server first adds the given customer to its queue.  If there was
not already a customer on the queue, it then proceeds to serve the
customer.  To serve a customer, the server finds out how long the
transaction will take and arranges for its local procedure {\tt
service-completed} to be called after that amount of time.  (The
procedure {\tt after-delay} takes a time delay and a no-argument
procedure and calls the procedure after the given delay.)  When the
time is up, the server tells the customer at
the head of the queue that it is done and proceeds to serve the next
customer in the queue, if there is one.

\beginthree{Creating customers}

A source of customers has two local state variables: the probability
(a number between 0 and 1) that a new customer will enter the system
during each time period, and a list specifying the distribution of
times for service requests.  When {\tt make-customer-source} creates a
customer source (the procedure {\tt me} below) it requests that the
source's local procedure {\tt customer-source-action} be called
after 1 unit of time.  This procedure decides (based on the probability)
whether to create a customer; if so, it creates one with a service
time chosen at random from the list of service times.   Whether or not it
creates a customer, it asks to be called again after 1 time unit.
This is how we simulate a given probability that a customer will enter
the system at any time.

\code
(define (make-customer-source chance-customer-enters
                              request-distribution)
  (define (customer-source-action)
    (if (odds chance-customer-enters)
        (make-customer (get-current-time)
                       (pick-random request-distribution)))
    (after-delay 1 customer-source-action))
  (define (me message)
     (error "unknown message -- customer-source" message))
  (after-delay 1 customer-source-action)
  me)
\uncode


The following procedure takes a number between
0 and 1 as input and returns true with that probability
and false otherwise.  It is used  by the customer source
to decide whether or not to generate a new customer.

\code
(define (odds percentage)
  (> percentage (/ (random 100) 100)))
\uncode

The customer source uses the following procedure to choose the
customer's service time at random from the list of service times:

\code
(define (pick-random x)
  (nth (random (length x)) x))
\uncode

\beginthree{Scheduling events}

Servers and customer sources both use a procedure {\tt after-delay}
to request that a procedure be called after a given delay.
The idea here is
that we maintain a data structure, called an {\it agenda}, that contains a
schedule of things to do.
See section 3.3.4 (under the heading ``The agenda'') for a definition
of {\tt after-delay} and a specification of the agenda operations
{\tt empty-agenda?}, {\tt first-agenda-item},
{\tt remove-first-agenda-item!}, {\tt add-to-agenda!}, and
{\tt current-time}.

Since time in the simulation is maintained in the agenda, the
procedure {\tt get-current-time} (used above by customers and customer sources)
can be implemented as follows:

\code
(define (get-current-time) (current-time the-agenda))
\uncode

\beginthree{Initialization and main loop}

There are two global variables that are used throughout the system.

\code
(define the-agenda nil)
(define available-servers nil)
\uncode\noindent
 Initialization consists of creating {\tt the-agenda}, setting up a
list of {\tt available-servers}, and creating a customer source.

The main loop of the simulation repeatedly sends a signal to the first
object on the agenda and removes that object from the agenda, until
the requisite number of time units have elapsed.

All of this is accomplished by the main procedure {\tt simulate}, which is
called with four arguments: the number of servers, the probability
that a new customer will enter the system during the next time
period, the list from which the time to service the customer will be
selected at random, and the number of time steps the simulation should
run.  {\tt Simulate} first performs the initialization and then enters
the main loop:

\code
(define (simulate number-servers
                  chance-customer-enters
                  request-distribution
                  max-time)
  (define (initialize-simulation)
    (set! the-agenda (make-agenda))
    (set! available-servers
          (make-server-list number-servers))
    (make-customer-source chance-customer-enters
                          request-distribution))
  (define (main-loop)
    (cond ((= (get-current-time) max-time) 'done)
          ((empty-agenda? the-agenda) 'done)
          (else (let ((first-item (first-agenda-item the-agenda)))
                  (first-item)
                  (remove-first-agenda-item! the-agenda)
                  (main-loop)))))
  (initialize-simulation)
  (main-loop))
\uncode

{\tt Make-server-list} {\tt cons}es together a list 
of a given number of servers:

\code
(define (make-server-list how-many)
  (cond ((= how-many 0) '())
        (else (cons (make-server)
                    (make-server-list (- how-many 1))))))
\uncode

\beginthree{Queues and agendas}

Queues are described and implemented in section 3.3.2 of the text.
For this application, we have defined an additional queue operation,
{\tt (length-queue \an{queue})}, that returns the number of items in a queue:

\code
(define (length-queue queue)
  (length (front-ptr queue)))
\uncode\noindent
 Note that each call to {\tt make-queue} produces a new queue, and operations
performed on one queue do not affect other queues.

Agendas are described and implemented in section 3.3.4 of the text
(see ``agenda'' in the index).

\beginexsix{
The main loop in {\tt simulate} checks whether {\tt the-agenda}
is empty.  Will this ever occur?  Why or why not?
}

%\endex

\beginexsix{
What would the simulator do if we removed the line}

\excode
(after-delay 1 customer-source-action)
\unexcode\noindent{\ninepoint
from the end of {\tt make-customer-source}?
}

\endex

\beginthree{Changes to the simulation}

Now that we have presented the entire simulator that we demonstrated
at the beginning of this section, we will consider some changes to
the program.

Suppose we wish to know the average service times and waiting times
for customers who go through the system.  We can compute this by
building a ``statistician object'' that we can communicate with
by calling the following procedures:

\code
(define (remember-service-time statistician time)
  ((statistician 'service-time) time))

(define (remember-waiting-time statistician time)
  ((statistician 'waiting-time) time))

(define (average-service-time statistician)
  (statistician 'average-service))

(define (average-waiting-time statistician)
  (statistician 'average-wait))
\uncode\noindent
 The last two messages cause the statistician to output the average of
all the service times and waiting times, respectively, that were
communicated by means of the first two messages.

The following two exercises add a statistician to the simulation.

\beginexsix{
Design the statistician object.
Describe the internal state
information that is kept by your statistician object and how
this information is updated in response to messages received.
A straightforward way for the statistician to compute the
desired averages is to save away all the information received from the
customers in a long list and do all the computation at the end.  But
it is also possible to avoid storing all this information by computing the
average incrementally.  For example, when a new value is received, the
new average can be computed in terms of the new value, the old
average, and the number of values that have been averaged so far.

\parindent10pt
Write a {\tt make-statistician} procedure that returns a
statistician object in accordance with your design.
Don't forget to initialize the object's internal state information.
To test your procedure, create a statistician {\tt stat1}}

\excode
(define stat1 (make-statistician))
\unexcode\noindent{\ninepoint
 and send {\tt stat1} some values for service and
waiting times, then ask {\tt stat1} for the two averages.
}

\beginexsix{ Modify the queuing simulation to include a
statistician.  Each customer should notify the statistician of its
service and waiting times when it leaves the system.  After the
simulation has run for the designated number of time units, the system
should request the average times from the statistician and print them.
(Hint: The only procedures you should need to modify for this exercise
are {\tt simulate} and {\tt make-customer}.)  }

\endex


The next two exercises let us get additional information about
the simulation.

\beginexsix{
The simulator reports on customers only when they leave the system.
When the simulation ends there are typically customers waiting
in queues or in the process of being served, but we do not know about
these.  Add a printout at the end of the simulation that reports on
the customers remaining in the system, telling when each customer
arrived and how much service it needs.  Include some indication
of which customers are being served.
}

\beginexsix{
Modify the simulator so that it prints an indication of which
server served which customer.  (Hint: You can do this by assigning
each server a number.  A customer will need some way of finding
out the number of its server.)
}

\endex

The following exercises implement changes to the treatment of
customers in the simulation.

\beginexsix{ Modify the simulator so that there can be more than
one customer source, each with its own characteristic probability of a
customer entering and distribution of customer request times.
(Hint: {\tt Simulate} will have to accept a list of probabilities and
request-distribution lists so that it can create customer sources
with the desired parameters.)
}

\beginexsix{ Real people waiting on line don't behave quite like the
customers in our model.  Rather than simply selecting the queue with
the fewest customers on it, they might try to take into account the
service times of the customers in a queue.  (For example, they don't
want to get behind a post-office customer with 20 packages if another
customer is just buying stamps.)
Also, people generally keep an eye on the other queues and switch
to another queue if it becomes shorter than the one they are currently on.

\expart{a}
Design a kind of customer that behaves differently from the ones in
our simulation, and write a constructor {\tt make-customer1} that
creates customers of this new kind. (Be sure that your customers
respond to all the right messages.)
To test your procedure, run a simulation using {\tt make-customer1}
in place of {\tt make-customer}.

\expart{b} Modify the simulation so that more than one kind of customer can enter
the system.  If you have implemented multiple customer sources (as
suggested in  the previous exercise), you might want to do this by
specifying which customer constructor should be used by each customer
source.  Alternatively, you might have a single customer source create
more than one kind of customer either by cycling through a list
of customer constructors or by selecting at random from a list of
customer constructors.  In the latter case, you can control the probability
of generating each kind of customer in the same way we controlled
the distribution of service request times---by putting multiple
copies of each customer constructor in the list.
}

\endex

In the next two exercises, we confront the fact that in the real world,
servers go on and off duty.

\beginexsix{
Modify the simulation so that servers, as well as customers,
are added while the simulation is running.  The creation of
servers, like the creation of customers, can be controlled
by a probability.  (Generally the probability of a server
entering the system will be much lower than the probability of
a customer entering.)  Alternatively, you might want to
schedule the addition of servers by giving {\tt simulate} a list
of times at which servers should be created.
}

\beginexsix{Design a means of taking a server out of the system and a
means of deciding when to do so, and add this to the simulation.  You
might prefer to have the servers just take a break and come back to
work after some period of time rather than leaving altogether.  In
either case, be sure that a server doesn't walk off the job in the
middle of serving a customer.  Also make sure that any customers
remaining on the abandoned queue redistribute themselves to other queues.  }

\endex

Instead of having each server form its own queue, we might want a
system in which all new customers enter a single queue and all
servers draw from this queue.  Many banks have started to use this
kind of system, on the grounds that it guards against a customer
getting stuck in a slow line when there are other lines
that are moving more quickly.

\beginexsix{
Redesign the simulation so
that, instead of separate server queues, there will be a single
queue-like object (let's call it {\tt waiting-line}) to which all
customers are added.  Then all servers draw from {\tt waiting-line}.
Here are some problems to watch out for in your design:

\beginlist{
In the original implementation, the customer currently being
served remains at the head of the server's private queue.  If there is
only one queue, then you must find some way to prevent more than one
server from trying to serve the same customer.
}

\beginlist{
You must worry about the problem of a server becoming idle
(e.g., when the waiting line is temporarily empty) and then never
accepting a new customer from the line.
The original implementation avoided this problem by having new customers
send ``serve me'' messages to specific servers.  If a server
was idle, the ``serve me'' message would awaken it.  How will you
address this problem now, when new customers go to the waiting line
rather than to a specific server?  (Hint: One way to solve the
problem is to make {\tt waiting-line} more than just a queue.  Have it also
be able to initiate requests to servers.)
}

\endlist

Write a brief description of your strategy for solving the above
problems.  Describe (in terms of messages sent) the interactions
between customers, servers, and the waiting line.

\parindent10pt
Implement the changes that you designed above.  The only old
procedures you should need to change are {\tt initialize-simulation},
{\tt make-customer}, and {\tt make-server}.  You will, of course, need
to also write some new procedures.  Note that if you want {\tt
waiting-line} to be more than just a queue, you should {\it not}
modify the implementation of queues, but rather use a queue as
part of a waiting line, similar to the way in
which servers had queues in the old implementation.
}

\parindent10pt
Do you think that the single queue arrangement provides better service
(e.g., in a bank) than does the multiple queue arrangement?  What
kinds of test could you run with your simulation to find out?  Are the
averages currently kept by the statistician enough, or might you want
to check other aspects of the behavior of the system?
