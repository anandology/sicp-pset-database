% for yTeX
% Copyright (c) 1990 Massachusetts Institute of Technology
% 
% This material was developed by the Scheme project at the Massachusetts
% Institute of Technology, Department of Electrical Engineering and
% Computer Science.  Permission to copy this material, to redistribute
% it, and to use it for any non-commercial purpose is granted, subject
% to the following restrictions and understandings.
% 
% 1. Any copy made of this material must include this copyright notice
% in full.
% 
% 2. Users of this material agree to make their best efforts (a) to
% return to the MIT Scheme project any improvements or extensions that
% they make, so that these may be included in future releases; and (b)
% to inform MIT of noteworthy uses of this material.
% 
% 3. All materials developed as a consequence of the use of this
% material shall duly acknowledge such use, in accordance with the usual
% standards of acknowledging credit in academic research.
% 
% 4. MIT has made no warrantee or representation that this material
% (including the operation of software contained therein) will be
% error-free, and MIT is under no obligation to provide any services, by
% way of maintenance, update, or otherwise.
% 
% 5. In conjunction with products arising from the use of this material,
% there shall be no use of the name of the Massachusetts Institute of
% Technology nor of any adaptation thereof in any advertising,
% promotional, or sales literature without prior written consent from
% MIT in each case. 


\typesize=11pt
\hsize=34pc
\vsize=50pc
\parskip 6pt plus 2pt
\parindent=0pt
\def\v#1{\hbox{\bf #1}}
\def\unit#1{{\v{\^#1}}}

\def\psetheader{
\centerline{MASSACHUSETTS INSTITUTE OF TECHNOLOGY}
\centerline{Department of Electrical Engineering and Computer Science}
\centerline{6.001 Structure and Interpretation of Computer Programs}
\centerline{Spring Semester, 1988}}

\def\code#1{\beginlisp
#1
\endlisp

\vskip .1in}

\rectoleftheader={6.001 -- Spring Semester 1988}
\rectorightheader={Problem Set 7}
\onheaders
\onfooters

\null
\vskip 1truein

\psetheader

\vskip .25truein

\centerline{Problem Set 7}

\vskip 0.25truein

\vpar
Issued: March 29, 1988


\vpar
Due: in recitation on April8 {\it for sections meeting at 9, 10 and
11} and on April 6 {\it for sections meeting at 12, 1 and 2}.


\vpar
Reading assignment: Section 3.4


\chapter{1. Homework exercises}


\vpar
Exercise 1. Do exercise 3.12 in the book.

\vpar
Exercise 2. Do exercise 3.13 in the book.

\vpar
Exercise 3. Do exercise 3.14 in the book.

\vpar
Exercise 4. Do exercise 3.16 in the book.

\vpar
Exercise 5. Do exercise 3.17 in the book.


\chapter{2. Laboratory Assignment: Mutable Data}

Use the ``load problem set'' command to load the code in the appendix
from the system onto your Chipmunk.

There is a cost associated with using {\tt CONS}.  Every time it is
called a piece of storage (a {\it cons cell}) in the computer's finite
memory must be set aside to store the {\tt CAR} and {\tt CDR} parts. 
Eventually the memory will be exhausted.  In fact this usually happens
quite soon when a lisp program is running.  But by the time it does
happen, it is usually the case that some of the cons cells allocated
earlier are no longer able to be refered to by the user's program.  For
instance in the following example

\beginlisp
(define numbers '(3 2 7 8))

(define product-of-squares-of-odd-elements
  (accumulate *
              1
              (mapcar square
                      (filter odd? numbers))))
\endlisp

\noindent the lists resulting from {\tt filter} and {\tt mapcar} are
no longer needed after their initial use, and in fact there will not
be any pointers to them in any active environment frames.  The lisp
system should be free to reclaim and later re-use the cons cells making
up these temporary lists.  Such action is called {\it garbage
collection} and will be the covered later in the course in chapter 5. 
The point for now is that there is some cost involved with {\tt CONS} as
all machines which implement it also have to periodically run a garbage
collection program---sometimes garbage collection can be the major time
user in a large program.\footnote*{On the chipmunks the status of the
lisp system is indicated in the lower right corner of the screen---when
it is the symbol $\pi$ normal lisp is running, when it is the symbol G
it is garbage collecting.}

One way to cut down on the amount of consing that a program does, and
hence cut down on garbage collection time, is to explicitly re-use cons
cells through the primitive mutators {\tt set-car!} and {\tt set-cdr!}.

In these lab exercises we are going to compare two styles of
programming, one which uses new cons cells always, and one which re-uses
cons cells wherever possible.  We will see certain tradeoffs between the
two styles, and as with many engineering problems we will see that there
is often no clear cut decision as to which is the better style for a
particular application.

\section{Part 1}

In {\tt ps7.scm} you will find definitions of {\tt ins-cons}, {\tt
ins-set-car!}, and {\tt ins-set-cdr!}, which are instrumented versions
of standard primitives.  By using these versions exclusively we can
track how much consing and mutation occurs in a program.  The two
procedures {\tt reset-counters} and {\tt report-counters} are used to
monitor these quantities.

Using the modified primitives write {\tt ins-append} and {\tt
ins-append!}, instrumented versions of {\tt append} and {\tt append!}
(see the first homework exercise above).

Compare the amount of consing done in:

\beginlisp
(ins-append '(a b c) '(d e f))

(ins-append! '(a b c) '(d e f))
\endlisp

\section{Part 2}

One possible definition of {\tt reverse} is

\beginlisp
(define (reverse x)
  (if (null? x)
      nil
      (append (reverse (cdr x)) (list (car x)))))
\endlisp

Write an instrumented version of this, called {\tt ins-reverse}.  Take
care to instrument the consing done by {\tt list} by replacing it with
appropriate calls to {\tt ins-cons}.  Monitor the amount of consing
done by {\tt ins-reverse} on a few examples.  What is the order of
growth of cons cells used by {\tt ins-reverse} as a function of the
length of the list to be reversed?

Write a linear version of {\tt ins-reverse}.
Note that you may well have to change he base case handling that was
used in the original {\tt reverse}.

\noindent {\bf Optional:} Wite {\tt ins-reverse!} that destrucutively
re-uses the cons cells from its argument and so does {\tt no} consing to
reverse a list.

\section{Part 3}

The procedures {\tt reset-counters} and {\tt report-counters} as defined
in the provided code refer to an {\it assq} count.  This is meant to be a
count of how many items are examined by {\tt assq}.  Here is a possible
definition of {\tt assq} without any error checking:

\beginlisp
(define (assq x y)
  (cond ((null? y) nil)
        ((eq? (caar y) x) (car y))
        (else (assq x (cdr y)))))
\endlisp

\noindent Write an instrumented version of this, named {\tt ins-assq}. 
Note that it should indicate three items examined for both of the
following cases:

\beginlisp
(ins-assq 'c '((a 1) (b 2) (c 3) (d 4) (e 5)))

(ins-assq 'd '((a 1) (b 2) (c 3)))
\endlisp

\section{Part 4}

In the code handout you will find the standard code for {\tt put} and
{\tt get} taken from page 216 of the text.  Modify this code to define
{\tt ins-put} and {\tt ins-get}.  Be careful not to miss the consing
done in the two places that {\tt list} is currently used, and be sure to
use the instrumented version of {\tt assq}.

To test out your new version start by typing:

\beginlisp
(ins-put 'a 'x 3)
(ins-put 'b 'y 2)
\endlisp

\noindent Then use the {\tt assq} instrumentation to see how much work
is done in retrieving these two entries from the table.

Now repeatedly do the following four operations:

\beginlisp
(ins-put 'a 'b 'c)
(ins-put 'b 'a 'd)
(ins-put 'a 'b 'd)
(ins-put 'b 'a 'd)
\endlisp

\noindent How does the amount of consing grow over time?  What happens
to the access time for the original two table entries as the above
operations are repeated?

\section{Part 5}

Now write instrumented versions of {\tt put} and {\tt get}, call them
{\tt ins-nomutate-put} and {\tt ins-nomutate-get}, which do not use
mutators at all.  The idea is to cons new entries on to the front of
subtables, and to cons new versions of subtables on to the front of the
table, shadowing old entries from {\tt ins-assq}.

\noindent {\tt HINT:} One way to do this is to modify your version of
{\tt ins-make-table} by changing the initial binding for {\tt
local-table} and by simplifying {\tt insert!}.  A minor change is
necessary in {\tt lookup} to accomodate a headerless representation of
tables.  Note that you will have to use the {\tt set!} special form, but
not the cons cell mutators.

Now answer the same questions for this version of {\tt put} and {\tt
get} as were asked in part 4.

\section{Part 6}

The non mutating versions of {\tt put} and {\tt get} certainly appear to
be losers.  But they have an advantage over the old version.  {\bf Every
version of the table which ever existed is still around!}  Thus we can
easily backup to an earlier version of the database.

Modify {\tt ins-nomutate-make-table} so that the dispatch procedure it
defines handles two more messages: {\tt table} and {\tt restore-table}
which let us backup to saved checkpoints of the table as in the
following example:

\beginlisp
(define ins-nomutate-operation-table (ins-nomutate-make-table))
(define ins-nomutate-get (ins-nomutate-operation-table 'lookup-proc))
(define ins-nomutate-put (ins-nomutate-operation-table 'insert-proc!))

...do some putting and getting...

(define saved-table (ins-nomutate-operation-table 'table))

...do some more putting and getting...

;;; now backup to an earlier version of the table

((ins-nomutate-operation-table 'restore-table) saved-table)
\endlisp

Try some examples to ensure that your new version does no consing at
all when saving and restoring a table.

\section{Part 7}

Now modify {\tt ins-make-table} to have the same capabilities.  Note
that in saving a table you will have to copy it to guard against later
mutations.  You can make the copying procedure an internal procedure at
the same depth as {\tt lookup} and {\tt insert!}. 

Now compare the consing behavior of this version of saving tables with
the version in Part 6.

\section{Part 8}

Given the normal behavior in terms of consing of the two new verions of
{\tt put} and {\tt get} is it the case that one of the two versions is
clearly better in a situation where the capablity for saving and backing
up to an earlier version of the database is required? If it is clear,
please explain.  If it is not clear, say why not, and describe scenerios
(e.g., in terms of relative frequencies of changing existing entries,
adding new entries and backing up the data base) in which each of the
approaches is superior. 


\end

