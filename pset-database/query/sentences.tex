% -*- LaTex -*-
% Copyright (c) 1990 Massachusetts Institute of Technology
% 
% This material was developed by the Scheme project at the Massachusetts
% Institute of Technology, Department of Electrical Engineering and
% Computer Science.  Permission to copy this material, to redistribute
% it, and to use it for any non-commercial purpose is granted, subject
% to the following restrictions and understandings.
% 
% 1. Any copy made of this material must include this copyright notice
% in full.
% 
% 2. Users of this material agree to make their best efforts (a) to
% return to the MIT Scheme project any improvements or extensions that
% they make, so that these may be included in future releases; and (b)
% to inform MIT of noteworthy uses of this material.
% 
% 3. All materials developed as a consequence of the use of this
% material shall duly acknowledge such use, in accordance with the usual
% standards of acknowledging credit in academic research.
% 
% 4. MIT has made no warrantee or representation that this material
% (including the operation of software contained therein) will be
% error-free, and MIT is under no obligation to provide any services, by
% way of maintenance, update, or otherwise.
% 
% 5. In conjunction with products arising from the use of this material,
% there shall be no use of the name of the Massachusetts Institute of
% Technology nor of any adaptation thereof in any advertising,
% promotional, or sales literature without prior written consent from
% MIT in each case. 

\input ../formatting/6001mac.tex

% \special{twoside}

\evensidemargin 35pt

\begin{document}

\psetheader{Spring Semester, 1988}{Problem set 11}

\medskip

\begin{tabular}{ll}
Issued: & May 3, 1988 \\
Due: & Optional \\
Reading: & Text, Chapter 4, Sections 4.4 and 4.5.
\end{tabular}

\section{The Query Language}

The code for the query language is taken from the text, pp. 362 -- 380.  The
code for the IBM Co. database is attached for your reference.  Instructions
relevant to the loading and editing of code (as required for the exercises)
will be posted in lab.

Once you have installed the query system by means of the {\tt load problem set}
command, you should be able to start the system by evaluating

\beginlisp
(query-driver-loop)
\endlisp

This starts the query system evaluator, which has its own driver loop for 
reading expressions from the terminal and evaluating them.  You will notice
that it provides you with its own prompt, {\tt query->}.  As with the normal
Scheme {\tt ==>} prompt, you should end anything you type to the {\tt query->}
prompt with the \fbox{EXECUTE} key.

In order to make it easier to type complicated queries, we have added
a command to the query language that will let you use the editor.  If
you type {\tt (edit)} to the {\tt query->} prompt, it will return you
to the editor. You can use the editor to compose a query and {\em zap}
one query down at a time.  Use the \fbox{ZAP DEFINE} key to mark
exactly one query to be executed; then use the \fbox{SCHEME} key to
get back to the query language.  Do not use the \fbox{EDIT} key on the
terminal; that is like typing ctrl-G and then {\tt (edit)}, which will
get you out of the query language (back into Scheme) and then into the
editor.

\exercise Work text exercises 4.28, 4.30, and 4.31.  The first exercise
requires you only to enter queries.  For the second and third exercises,
you will need to define rules.  Test your rules on the IBM Co. database.

\medskip

In the following exercises you are to write rules for the Query Language
that can generate and test simple English sentences.  Consider only 
sentences that consist of the word ``the'' followed by a {\em noun phrase}
and a {\em verb phrase}.

\begin{itemize}

\item A noun phrase consists of either a noun, or an adjective followed by 
a noun phrase.

\item A verb phrase consists of either a verb or a verb followed by an adverb.

\end{itemize}

First, install a set of assertions into the database to define the words in the
vocabulary:

\beginsmalllisp
(noun dog)
(noun cat)
(noun professor)
(noun student)
(noun rat)
(adjective red)
(adjective slow)
(adjective dead)
(verb ate)
(verb ran)
(verb slept)
(verb drank)
(adverb quickly)
(adverb happily)
(adverb well)
\endsmalllisp

\exercise Given these assertions, it is straightforward to write a pair
of rules for inferring verb phrases.  Complete the rules below by 
filling in the missing expressions.

\beginlisp
(rule (verb-phrase (?x))
  \fillin{exp1})
\null
(rule (verb-phrase (?x ?y))
  \fillin{exp2})
\endlisp

Test your rules by determining what verb phrases can be inferred from
the database.  Does it matter what order you enter these rules into
the database?  To check these rules on a Chipmunk, you will have to
define a database for sentences analogous to the IBM database.

\exercise Now write rules for inferring noun phrases by filling in the
missing expressions below:

\beginlisp
(rule (noun-phrase (?x))
  \fillin{exp3})
\null
(rule (noun-phrase (?x . ?y))
  \fillin{exp4})
\endlisp

Test your rules by determining what noun phrases can be inferred from the database.
Does it matter what order you enter these rules into the database?

\exercise Finally, create a rule for inferring sentences from noun and verb phrases.
Enter the rules for {\tt append-to-form} (p. 347 of the text) into the database.
{\tt append-to-form} can be used to split a list of words into two parts.  A sentence
consists of the word ``the'' followed by a noun phrase followed by a verb phrase.
Complete the following rule to infer sentences:

\beginlisp
(rule (sentence (the . ?s))
  (and (append-to-form \fillin{exp5})
        \fillin{exp6}
        \fillin{exp7}))
       
\endlisp

Verify that ``The slow dog ran well'' is an acceptable sentence.  Under what reorderings
of the {\tt and} clauses is this sentence not accepted?  Explain.

\exercise What sentences of the form ``The red \ldots slept \ldots'' satisfy 
the rules?

\exercise What sentences of the form ``The \ldots rat \ldots slowly'' satisfy
the rules?

\exercise What query would you use to generate sentences?

\end{document}
