% for yTeX
% Copyright (c) 1990 Massachusetts Institute of Technology
% 
% This material was developed by the Scheme project at the Massachusetts
% Institute of Technology, Department of Electrical Engineering and
% Computer Science.  Permission to copy this material, to redistribute
% it, and to use it for any non-commercial purpose is granted, subject
% to the following restrictions and understandings.
% 
% 1. Any copy made of this material must include this copyright notice
% in full.
% 
% 2. Users of this material agree to make their best efforts (a) to
% return to the MIT Scheme project any improvements or extensions that
% they make, so that these may be included in future releases; and (b)
% to inform MIT of noteworthy uses of this material.
% 
% 3. All materials developed as a consequence of the use of this
% material shall duly acknowledge such use, in accordance with the usual
% standards of acknowledging credit in academic research.
% 
% 4. MIT has made no warrantee or representation that this material
% (including the operation of software contained therein) will be
% error-free, and MIT is under no obligation to provide any services, by
% way of maintenance, update, or otherwise.
% 
% 5. In conjunction with products arising from the use of this material,
% there shall be no use of the name of the Massachusetts Institute of
% Technology nor of any adaptation thereof in any advertising,
% promotional, or sales literature without prior written consent from
% MIT in each case. 


\typesize=11pt
\hsize=34pc
\vsize=50pc
\parskip 6pt plus 2pt
\def\v#1{\hbox{\bf #1}}
\def\unit#1{{\v{\^#1}}}

\def\psetheader{
\centerline{MASSACHUSETTS INSTITUTE OF TECHNOLOGY}
\centerline{Department of Electrical Engineering and Computer Science}
\centerline{6.001 Structure and Interpretation of Computer Programs}
\centerline{Fall Semester, 1987}}

\def\code#1{\beginlisp
#1
\endlisp

\vskip .1in}

\rectoleftheader={6.001 -- Fall Semester 1987}
\rectorightheader={Problem Set 9}
\onheaders
\onfooters

\null
\vskip 1truein

\psetheader

\vskip .25truein

\centerline{Problem Set 9}

\vskip 0.25truein

\vpar
Issued: November 12, 1987.


\vpar
Due: in recitation on November 25, 1987 {\it for all sections}


\vpar
Reading assignment: Sections 4.4 and 4.5 of the text

\vskip 20pt

{\bf Quiz Announcement:}
Remember that Quiz 2 is on Wednesday, November 18.  The quiz 
will be held in Walker Memorial Gymnasium (50-340) from 5--7PM or
7--9PM.  You may take the quiz during either one of these two periods,
but students taking the quiz during the first period will not be allowed
to leave the room until the end of the period.  The quiz is {\bf
partially} open book, meaning that you may bring a copy of the Scheme
manual, plus {\bf one} sheet of 8.5 by 11 inch paper, on which you may
place any crib notes you think will be useful.  The quiz will cover
material through that presented in lecture on November 10, and through
section 4.2 of the text.

\chapter{1. Homework exercises}

Do the following exercises in the textbook:

\beginbullets

\bpar Exercise 1. - exercise 4.32 in the textbook

\bpar Exercise 2. - exercise 4.33 in the textbook (Note: the mini-database
      for this exercise is included in the code at the end of this handout.
      You do not need to type it in in order to try out your answers.)

\chapter{2. Laboratory Assignment: The Query Language}

Except for one exercise in this part, you will not be doing any programming
in SCHEME for this problem set.  Instead, you will be programming in the
"query language" described in Chapter 4.4 in the book.  Use the
``Load Problem Set'' command to load the query language code and a
simple data base.  As you will notice, this version of the language
is not very efficient.  To avoid slowing it down, be sure NOT to ``zap''
the source code of the query language into SCHEME -- a compiled
version, which is more efficient, is being loaded automatically into SCHEME
when you issue the ``Load Problem Set'' command.

\vpar
Don't forget to end everything you type to the query prompt with the
{\tt execute} key, just as you would at the normal Scheme prompt.

\vpar
Once you have read the files in, you must do two things before you can
start the system.  First, you must ``zap'' in the data base definition.
Second, you must initialize the system by typing in:

\beginlisp
(initialize-data-base pi-data-base)
\endlisp
\vpar
After you have done that, you can start the system by typing in:

\beginlisp
(query-driver-loop)
\endlisp

\section{What Happens Next with Paranoid International Securities Systems?}

Paranoid International's Mobot business was not a great success.  Their
clients complained that the Mobots are not smart enough to handle campus
security (campuses are becoming more and more violent places nowadays.)
PI discovered that the main reason is that they don't have enough computer
science people in the R-and-D department, and they ended up forcing a lot of
the EE people to write software to program the MOBOTS (no offense to the
EE's).  Since it's recruiting season again, PI decided to hire more
computer science people besides the usual marketing and MIS (management
information systems) people they hire every year.  To keep track of the
people they interviewed, chief campus recruiter Mr. Han Shakin maintained
a data base of his interviewees.  A data base query system is supplied
by the MIS department of PI, but with minimal documentation on how to
use the system (as is always the case with the products that MIS departments
deliver.)  Your task is to help Mr. Han Shakin to use the system.

\section{Part 1}

Give simple queries that retrieve the following information from the
data base:

\bpar a. Everyone who went to school at MIT

\bpar b. Everyone who is interested in some area in AI

\bpar c. The addresses of everyone who lives in Cambridge

Test out your queries and hand in a photo session of it.

\section{Part 2}

Formulate compound queries that retrieve the following information from
the data base:

\bpar a. All those who got an MBA degree from ``Harvahd''.

\bpar b. All those whose GPA is between 4.0 and 5.0, or whose communication
         skill is rated ``good''.  

\bpar c. All those who have a computer science degree which is not an
         SB degree and didn't get that degree from MIT.  (For this one,
         you have a chance to use the NOT special form.  Does the query
         system give you a different response if you put the NOT statement
         as the first clause in the AND statement?  Why?)

Test out your queries and hand in a photo session of it.

\section{Part 3}

Define four rules described as follows:

\bpar Define a rule which says that a person is a ``byte-head'' if
         he/she has a PHD in computer science and is interested in any
         area in AI.

\bpar  Define a rule which says that a person can be hired in the
        marketing division if he/she doesn't have a PHD degree and
         has either an MBA degree or his/her communication skill is
         rated as ``best''.  (Mr. Han Shakin thinks that paying a
         PHD to do marketing is too expensive.)  Notice that the
         name of Cy D. Fect gets printed twice.  Why?

\bpar  Define a rule which says that a person can be hired in the
         software development division if he/she has a degree in computer
         science, has a GPA greater than 4.5, and does not have a
         PHD degree (PHD's are reserved for the research division.)

\bpar  Define a rule which says that a person should be paid more
         in order for PI to attract him/her if the person has a computer
         science degree from MIT and his/her GPA is greater than 4.8.

Test out your rules and hand in photo sessions of it.

\section{Part 4}

For this part, you must define rules in terms of other rules:

\bpar Define a rule which says that a person can be hired in the
         AI division if he/she is a ``byte-head'' or has an SM degree
         in computer science while interested in some area in AI.

\bpar Define a rule which says that a person can be hired in the
         MIS division if he/she can be hired in the software development
         division and is also interested in some area in MIS.

\bpar Define a rule which says that a person can be hired in the
         R-and-D division if he/she can be hired in either the AI division
         or the software-development division.

\bpar Define a rule which says that a person can be hired under the
         company's affirmative action policy if the person can be hired
         in either marketing, MIS or R-and-D, and has a degree from the
         school Swellesley.  Could you explain why the name of
         Alyssa P. Hacker gets printed twice?

\bpar Define a rule which says that a person is cheaper to be hired
         in R-and-D if the person can be hired in R-and-D but doesn't need to
         to be paid more for PI for attract him/her.

\endbullets
\vpar
Test out the rules and hand in photo sessions of it.

\section{Part 5}

Try out the rule {\tt know-each-other} given in the data base.  Notice
that every pair of people who know each other gets printed twice.
Why? Devise a way to fix this problem.

\section{Part 6}

Do exercise 4.41 in the text book.  You need not make any changes to the
query language code in order to implement {\tt unique}.  Merely set up a
small file of your own with the new code to be added to the query interpreter.
Note that during the debugging of your program, you will probably be 
switching back and forth between the editor and SCHEME.  Each time you
modify the definition of {\tt uniquely-asserted?} and read it back into
SCHEME, be sure to also read in and execute the {\tt put} command.
Otherwise, you will change the procedure definition, but {\tt qeval} will
still dispatch to the old procedure.

Test out your procedure by issuing the following queries:

\beginlisp
(unique (pay-more-to-attract ?x))
\pbrk
(unique (can-hire-marketing ?x))
\endlisp

The first one should return the only frame where {\tt x} is bound to
Ben Bitdiddle, and the second one should return nothing.

\section{Part 7}

The rule {\tt can-supervise} given in the data base is not complete.
It uses the criteria {\tt better-qualification}.  Write a rule that
says that a person has better-qualification than another if he/she
is from either MIT, Harvahd or Swellesley, and has a degree in the
same subject as the other person, but can communicate better than
the other person.  You might find it useful to define a rule
{\tt can-communicate-better}.  Notice that since there are only
four ratings of communication skills: best, better, good and ok, and
they are lexicographically ordered, one can compare them lexicographically
to determine which person's communication skill is better.

Test out your rule and make sure it works.  This rule takes a long
time to run, so be patient with it.

If we switch the order of the two clauses in the {\tt and} statement
in {\tt can-supervise}, the system goes into an infinite loop.  Explain
why.

\section{Part 8 - Extra Credit}

(This is a hard one).  Many of logic systems have expert systems
built on top of them.  The expert systems make use of the power of the
logic systems to do logical deductions while interacting with the user.
Specifically, an expert system usually asks the user questions if the
logic system cannot make deductions due to a lack of information in the
data base.  The following is an example of an expert system which can
recommend a very limited set of drugs for a patient:

\vpar
Suppose our data base consists of the following facts:

\beginlisp
(suppresses aspirin headache)
(suppresses lomotil diarrhea)
(aggravates aspirin ulcer)
(aggravates lomotil bad-liver)
\endlisp

\vpar
A computer with the expert system on it would behave as follows:

\begintable[ll]
user:& (should-take John ?x)\cr
computer:& (has-complained-of John ?s)\cr
&           What is the value of ?s\cr
user:& headache\cr
computer:& (has-condition John ulcer)\cr
user:& no\cr
computer:& (should-take John aspirin)\cr
\endtable

\vpar
We can imagine having a rule that looks like the following:

\beginlisp
(rule (should-take ?p ?d)
      (and (is-reported (has-complained-of ?p ?s))
           (suppresses ?d ?s)
           (not (unsuitable-for ?d ?p))))
\endlisp

\vpar
{\tt is-reported} is a very complex rule that first searches the data base
to determine whether the patient has previously reported a symptom.  If
not, it makes the computer ask the patient.

\vpar
With all this information in mind, how much of the existing query
language implementation has to be modified in order to have it behave
like the trivial expert system outlined above?  Give a paragraph or
two of your thoughts.  Details on the implementation are not important.
We are interested in the basic changes one must make to the system.

\end
