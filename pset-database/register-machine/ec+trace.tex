% for yTeX
% Copyright (c) 1990 Massachusetts Institute of Technology
% 
% This material was developed by the Scheme project at the Massachusetts
% Institute of Technology, Department of Electrical Engineering and
% Computer Science.  Permission to copy this material, to redistribute
% it, and to use it for any non-commercial purpose is granted, subject
% to the following restrictions and understandings.
% 
% 1. Any copy made of this material must include this copyright notice
% in full.
% 
% 2. Users of this material agree to make their best efforts (a) to
% return to the MIT Scheme project any improvements or extensions that
% they make, so that these may be included in future releases; and (b)
% to inform MIT of noteworthy uses of this material.
% 
% 3. All materials developed as a consequence of the use of this
% material shall duly acknowledge such use, in accordance with the usual
% standards of acknowledging credit in academic research.
% 
% 4. MIT has made no warrantee or representation that this material
% (including the operation of software contained therein) will be
% error-free, and MIT is under no obligation to provide any services, by
% way of maintenance, update, or otherwise.
% 
% 5. In conjunction with products arising from the use of this material,
% there shall be no use of the name of the Massachusetts Institute of
% Technology nor of any adaptation thereof in any advertising,
% promotional, or sales literature without prior written consent from
% MIT in each case. 


\typesize=11pt
\hsize=34pc
\vsize=50pc
\parskip 6pt plus 2pt

\def\psetheader{
\centerline{MASSACHUSETTS INSTITUTE OF TECHNOLOGY}
\centerline{Department of Electrical Engineering and Computer Science}
\centerline{6.001 Structure and Interpretation of Computer Programs}
\centerline{Fall Semester, 1988}}

\def\code#1{\beginlisp
#1
\endlisp

\vskip .1in}

\rectoleftheader={6.001 -- Fall Semester 1988}
\rectorightheader={Problem Set 9}
\onheaders
\onfooters

\null
\vskip 1truein

\psetheader

\vskip .25truein

\centerline{Problem Set 9}

\vskip 0.25truein

\vpar
Issued: Tuesday, 22 November 1988

% NOTE: missing # in {\tt (LOAD " 5:/SCHEME/PSETS/PS10-ECEVAL.SCM") }
% should be {\tt (LOAD "#5:/SCHEME/PSETS/PS9-ECEVAL.SCM") }, but
% the # won't go through ytex.  There is surely a way to do this in
% ytex, but for now it is written in by hand.

% The code that goes with this handout is in PS9-ECEVAL.SCM, PS9-SYNTAX.SCM,
% and PS9-ADDEND.SCM.  

\vpar
Due: 
on Friday, 2 December

\vpar
Reading Assignment: Chapter 5, 121 pages! -- start early.

\vpar
{\bf FINAL EXAM}: The final exam will be held on Tuesday, 20 December 1988,
from 1:30 PM to 4:30 PM in Dupont Gymnasium.  It will cover
material from the entire semester.  The final is open-book.

\vskip 1truein

{\bf NOTE}: 
This problem set covers two of the three major programs discussed in chapter
5: the register-machine simulator and the explicit control evaluator.
Problems 1 and 2 cover register machines, 3 and 4 involve using and
modifying the explicit control evaluator.  The lectures on this material
will cover these programs in the same order as the problem set, so
you will have no trouble starting the problem set early.

The following files are essentially the same as the
code in the book with a few modifications to help you see what is
going on.  You will not need to know the details of register machine
simulator code, but you should (from reading the text) get 
a general idea of what is going on.  The code for the explicit control 
evaluator is included at the end of this handout.  A {\tt load-problem-set} 
command will load PS9-ANSWER.SCM into the editor and the other files 
into Scheme.

\ftpar{\tt PS9-REGSIM.SCM}
This is the register machine simulator discussed in section 5.1.5.  It
has been modified to include a monitored stack, as suggested in the
subsection ``Monitoring machine performance'' (p.  417), and to
complain about various errors in machine descriptions, as suggested in
exercise 5.9. In addition we added a simple tracing facility, a
counter for the number of machine operations executed, and a dynamic
display of the stack depth using chipmunk graphics.

\ftpar{\tt PS9-SYNTAX.SCM }
These are Scheme procedures that define the representation of
expressions and environments.  This is essentially the same syntax as
was used in the meta-circular evaluator, with a few additions required
by the explicit-control version.

\ftpar{\tt PS9-ECEVAL.SCM }
This is the explicit-control evaluator described in
section 5.2. All of the code has been collected here in the form of a
definition to be processed by the register machine simulator when the
file is loaded, so do not be surprised if the loading takes quite a
while. This version of the evaluator also has hooks in it to handle
code produced by the compiler.

\ftpar{\tt PS9-ANSWER.SCM }
This is the file where you can enter your answers.  It contains tables
for entering statistics, which can be edited or filled in by hand,
and Scheme procedures that you will want to zap into Scheme. Use the 
Scheme primitive {\tt list-file} to print a copy of the file.

\chapter{Part I -- The register machine simulator}

Register machines are built with {\tt define-machine} and can be run by 
typing {\tt (start <machine-name>)}.  See pages 394-405 for some example 
machines definitions, taking note of the syntax for {\tt define-machine}.

To make use of the facilities for monitoring machine performance, you will need
to use the initialization procedures between each run of a machine.  The {\tt
initialize-stack} and {\tt initialize-ops-counter} instructions are included to
print statistics and to reset some of machine's local state variables that were
used in statistics gathering.  (Page 413 of the book describes {\tt
initialize-stack}, and an example below shows how you will want to use
both instructions.)

\section{Problem 1 -- Building register machines}

Consider the following procedure for counting atoms in a data structure.

\beginlisp
(define (count-atoms lst)
        (cond ((null? lst) 0)
              ((atom? lst) 1)
              (else (+ (count-atoms (car lst))
                       (count-atoms (cdr lst))))))
\endlisp

Design a register machine for computing {\tt count-atoms} by filling
in the following program shell.  Assume the argument to {\tt count-atoms} is 
in {\tt lst} register when the machine starts, and the result is placed in
{\tt value} after it has finished.  The initializations should be the
last instructions performed by the machine.

\beginlisp
(define-machine count-atoms-machine
  (registers value continue lst)
  (controller
        .
        .
        .
    (perform (initialize-stack))
    (perform (initialize-ops-counter))))
\endlisp

The above shell is included in the file {\tt PS9-answer.scm } along
with a procedure {\tt try-count-atoms} to help you test your machine.  
Note the statistics initializations as the last machine intructions.  

In addition to statistics for stack usage and total number of operations, we 
have added some procedures to help you debug your machines.  The two 
procedures {\tt remote-trace-on} and {\tt remote-trace-off} take a single
argument, a register machine, and control the printing of machine intructions 
as they are executed.  In addition, there is a facility for tracing a specific 
register of a machine using {\tt remote-trace-reg-on} and 
{\tt remote-trace-reg-off}; both procedures take a register machine and a 
register name.  A message will be printed every time that register is set to 
a new value.  For example:

\beginlisp
1 ==> (remote-trace-reg-on count-atoms-machine 'value)
()
\pbrk
1 ==> (try-count-atoms '(a . b))
(REG-TRACE: VAL NEW-VALUE: 1) 
(REG-TRACE: VAL NEW-VALUE: 1) 
(REG-TRACE: VAL NEW-VALUE: 0) 
        .
        .
        .
(REG-TRACE: VAL NEW-VALUE: 2) 
(TOTAL-PUSHES: 6 MAXIMUM-DEPTH: 4) 
(MACHINE OPERATIONS: 48) 
2 
\endlisp

Turn in a listing of your machine.

\section{Problem 2 -- Running hand-coded count-atoms}

Assume you are going to run your machine only on balanced binary trees,
e.g., a pair of atoms, {\tt (cons 'a 'b)}, is such a tree, as well as
{\tt (cons t1 t2)} where {\tt t1} and {\tt t2} are themselves balanced binary 
trees of the same size.  Fill out the table in {\tt PS9-answer}, which 
contains more examples of balanced binary tress.  Derive formulas for the 
number of operations, total stack pushes and maximum stack depth as a function 
of $n$, where $n$ is the depth of a tree.  (The tree printed as {\tt (a . b)} 
is of depth 1.)  


\chapter{Part II -- The explicit control evaluator}

The explicit control evaluator is implemented as a register machine,
so you will have Scheme embedded in the simulator embedded in
Scheme!!! To start the machine, type {\tt (go)}. On the bottom of the screen
you will see a graphical representation of the depth of the stack.
Note that the graph "wraps around" back to the left side for long
runs.  At the end of a run, the system will print a few statistics.
For example:

\beginlisp
1 ==> (go)
(TOTAL-PUSHES: 0 MAXIMUM-DEPTH: 0) 
(MACHINE OPERATIONS: 5) 
EC-EVAL==>(define (f x) (+ x (* x x)))
F 
(TOTAL-PUSHES: 3 MAXIMUM-DEPTH: 3) 
(MACHINE OPERATIONS: 35) 
EC-EVAL==>(f 10)
110 
(TOTAL-PUSHES: 21 MAXIMUM-DEPTH: 8) 
(MACHINE OPERATIONS: 179) 
EC-EVAL==>
Quit!
\endlisp

Notice that the explicit control evaluator has the prompt {\tt EC-EVAL==>}
to show you that you are not talking to Scheme. Try playing around
with a few expressions. Use the "alpha" and "graphics" to help see the
graph of the stack depth. When it gets too cluttered you can clear the
screen with the {\tt clear-graphics} procedure which has been installed as
a primitive operation. The other primitives are:

\beginlisp
    car cdr cons atom? eq? null? + - * / > < = 
\endlisp

\vpar
but it is easy to add more. See the procedure {\tt setup-environment} in
the ECEVAL file.

As you will probably notice, there is no error handler for the
evaluator, so the slightest mistake will bounce you back out to
Scheme. Just hit a {\tt <cntl>G} and type {\tt (go)} again (your old 
definitions will still be defined).

\section{Problem 3 -- Running interpreted count-atoms}
Crank up the explicit control evaluator and carefully type in the
following procedure definitions:

\beginlisp
(define (count-atoms-r lst)
        (cond ((null? lst) 0)
              ((atom? lst) 1)
              (else (+ (count-atoms-r (car lst))
                       (count-atoms-r (cdr lst))))))

(define (count-atoms-i lst)
        (define (loop l1 l2 result)
                (cond ((null? l1) 
                          (cond ((null? l2) result)
                                (else (loop (car l2) (cdr l2) result))))
                      ((atom? l1) (loop (car l2) (cdr l2) (+ 1 result)))
                      (else (loop (car l1) (cons (cdr l1) l2) result))))
        (loop lst '() 0))
\endlisp

Fill in the values of the two tables for interpreted versions of 
{\tt count-atoms} in {\tt PS9-answer.scm}.  Observe how the shapes of the 
stack usage differ between {\tt count-atoms-r} and {\tt count-atoms-i}.  
Derive formulas for number of operations, total stack pushes and maximum 
stack depth for each of the interpreted procedures.   Write a brief 
explanation (one paragraph) of your observations.  For example, can 
you make any general statements about the relative performance, in terms of 
stack space and number of operations, between the recursive and iterative 
versions?  

\chapter{Part III -- Debugging from the explicit control evaluator}

To find bugs in programs, it is often useful to be able to watch the
program execute on a particular input.  One way of doing this is to trace 
the operation of the evaluator by using {\tt remote-trace-on} and 
{\tt remote-trace-off} on the evaluator register machine.  Once tracing
has been turned on, every register machine instruction of the evaluator will
automatically be traced, since the code runs on the explicit-control-evaluator 
machine.  For example, here is a trace (via {\tt photo}) of the evaluation of 
{\tt 'a} : 

\beginlisp
1 ==> (remote-trace-on explicit-control-evaluator)
() 
\pbrk
1 ==> (go)
(PERFORM (INITIALIZE-STACK))
(TOTAL-PUSHES: 0 MAXIMUM-DEPTH: 0) 
(PERFORM (INITIALIZE-OPS-COUNTER))
(MACHINE OPERATIONS: 2) 
(PERFORM (NEWLINE))
                                                 ; Newline performed
(PERFORM (PRINC 'EC-EVAL==>))EC-EVAL==>
(ASSIGN EXP (READ))'a                            ; User input here
\pbrk
(ASSIGN ENV THE-GLOBAL-ENVIRONMENT)
(ASSIGN CONTINUE PRINT-RESULT)
(GOTO EVAL-DISPATCH)
(BRANCH (SELF-EVALUATING? (FETCH EXP)) EV-SELF-EVAL)
(BRANCH (QUOTED? (FETCH EXP)) EV-QUOTE)
(ASSIGN VAL (TEXT-OF-QUOTATION (FETCH EXP)))
(GOTO (FETCH CONTINUE))
(PERFORM (USER-PRINT (FETCH VAL)))
A                                                ; The result!
(GOTO READ-EVAL-PRINT-LOOP)
(PERFORM (INITIALIZE-STACK))
(TOTAL-PUSHES: 0 MAXIMUM-DEPTH: 0) 
(PERFORM (INITIALIZE-OPS-COUNTER))
(MACHINE OPERATIONS: 14) 
(PERFORM (NEWLINE))
                                                 ; Newline performed
(PERFORM (PRINC 'EC-EVAL==>))EC-EVAL==>
(ASSIGN EXP (READ))
Quit!
\endlisp

As you can see, the traces can quickly get out of hand! Furthermore,
it is not at all clear which occurence of an instruction is being
executed.  {\tt (GOTO (FETCH CONTINUE))}, for example,
occurs at many different places in the evaluator machine.  By
comparing the surrounding instructions in both the trace and the machine
definition, you may be able to determine which instruction in the
machine is being executed.  In general, however, making sense out of your 
register machine trace is similar to doing reverse compilation.
Therefore, we would like to trace application of 
{\tt EC-EVAL} Scheme procedures, rather than register machine instructions.  

The {\tt trace} procedure on the Chipmunks lets you trace Scheme procedure
applications; the next problems develop a modification to the explicit
control evaluator to provide a similar tracing mechanism.  You will be
modifying the explicit control evaluator to handle the {\tt trace} special
form.  {\tt Trace} takes one argument, a procedure, and informs that procedure
that whenever it is applied, it should print out its own name and the list of
actual arguments to which it's being applied.  The following shows an example
of the use of {\tt trace}:

\beginlisp
EC-EVAL==>(define (factorial n)
                  (cond ((= n 0) 1)
                        (else (* n (factorial (- n 1))))))

FACT 
(TOTAL-PUSHES: 3 MAXIMUM-DEPTH: 3) 
(MACHINE OPERATIONS: 37) 
EC-EVAL==>(trace fact)

FACT 
(TOTAL-PUSHES: 0 MAXIMUM-DEPTH: 0) 
(MACHINE OPERATIONS: 18) 
EC-EVAL==>(fact 2)

(CALLING PROCEDURE FACT WITH (2)) 
(CALLING PROCEDURE FACT WITH (1)) 
(CALLING PROCEDURE FACT WITH (0)) 
2 
(TOTAL-PUSHES: 80 MAXIMUM-DEPTH: 14) 
(MACHINE OPERATIONS: 715) 
EC-EVAL==>
\endlisp

\section{Problem 4 -- Tracing in the explicit control evaluator}

Implement {\tt trace} by following each of the steps below.  One
of the goals of the implementation should be to avoid slowing the
execution of untraced procedures.  Note that you are not being asked 
to implement {\tt untrace}, so there is no easy way to stop a procedure 
from being traced.  

There are a number of possible approaches you could take in the implementation.
One would be to modify the Scheme code in the procedure object being traced by
adding a print expression at the beginning of the procedure body.  However,
this would not work well for tracing primitive (or, as you will see later,
compiled) procedures, since the Scheme code of the procedure body is not 
available in those cases.  Instead, we will add a new kind of procedure object 
called a {\it traced procedure object}, which will contain the real
procedure object.  In addition to the real procedure object, a traced 
procedure object will contain a special symbol so it can recognized as a 
traced procedure, and the procedure name to be printed when the procedure is 
called.

\section{Step 1: a trace expression}

Write two new procedures for working with expressions represented
as lists:  {\tt tracing?} takes an expression and returns a true value 
when the expression is a trace expression; {\tt traced-proc} returns the 
procedure part of a trace expression.  These should be analogous to
the procedures in section 4.1.2 of the book and {\tt PS9-SYNTAX.SCM}.
Turn in a listing of your procedures.

\section{Step 2: representing traced procedures}

Write procedures for changing a procedure object (as represented by
the evaluator) into a traced procedure object.  You will need to
work with the representation of procedure objects, so review
{\tt make-procedure} (page 305) and {\tt primitive-procedure-objects}
(page 311).  

{\tt (change-to-traced-procedure! <name> <procedure-object>)}
will mutate the procedure object to be a traced procedure object.  
Hint: This is tricky. You will need to mutate the old procedure object, 
not build a new one, but you must also keep all the pieces of 
the old one in the new one.  In particular, you will have to store the 
lamba expression, environment, and tag ({\tt primitive} or {\tt procedure})
along with the name.  All this is necessary in order to make the other
three procedures in this step behave correctly.	

{\tt (traced-procedure? <procedure-object>)} is a true value 
whenever the procedure is being traced, i.e., whenever it is a traced 
procedure object.

{\tt (procedure-part-traced <traced-proc>)} returns the (untraced)
procedure object that is {\it almost} the procedure object used when
{\tt change-to-traced-procedure!} was applied.  By {\it almost} we
mean that it is a different procedure object of the same tag
(i.e., primitive or compound) and with the same ({\tt eq?})
lambda-expression and environment.

{\tt (name-part-traced <traced-proc>)} returns the name of the 
procedure object used when {\tt change-to-traced-procedure!} was applied.

Test your procedures by building some procedure objects using 
{\tt make-procedure}, changing them into traced procedures, and
applying the selector procedures.  Turn in a listing of your
procedures.

\section{Step 3: print routines}

We have provided you with a procedure, {\tt print-call}, that takes a 
procedure name and a list of arguments and prints a line of the form:
\beginlisp
        (CALLING PROCEDURE * WITH (3 2))
\endlisp
Don't be mislead by the call to {\tt reverse} inside {\tt print-call}.
This is just to make your life easier, since argument lists are stored
in reverse order within the evaluator.

In addition, the {\tt user-print} routine in {\tt PS9-MCEVAL.SCM}
must be modified to handle traced procedures in some reasonable way.
We have included a modified version of {\tt user-print} that you will
want to use with your modified evaluator.  Zap both procedures into
Scheme.

\section{Step 4: building traced procedures in the evaluator machine}

Modify the explicit-control-evaluator definition to handle the {\tt trace} 
special form.  First, add a new branch in the {\tt eval-dispatch} part of the 
code to branch to a new label {\tt ev-trace-command}.  Next, add a section 
of code beginning with the new label {\tt ev-trace-command} that contains 
instructions to change the current procedure object to a traced procedure.  
These changes should make the evaluator handle expressions of the form
{\tt (trace <procedure>)}. 

\section{Step 5: applying traced procedures in the evaluator machine}

Now that you have traced procedure objects in the environment, you need to
modify the code that handles procedure application to deal with these new 
objects.  Add a branch to the {\tt apply-dispatch} part of the code so that 
traced procedures can be treated specially.  Next, add the corresponding code 
for traced procedure application under the label {\tt traced-apply} to first 
print the trace information and then apply the real procedure part of the 
traced procedure as usual. Turn in a listing of the evaluator, marking your 
changes.  

\section{Step 6: using your tracing mechanism}

Your new evaluator should be now able to correctly trace compound and 
primitive procedures.   Try tracing some procedures, including some
that are called from within compound procedures.  Turn in a {\tt photo} 
session from running the {\tt factorial} program above, but this time trace 
the call to the primitive "{\tt -}" procedure.

\end

%stuff below was used in an earlier (spring 87) version of the pset

\chapter{Part IV -- The compiler!!}

The compiler translates Scheme expressions into register machine code
which can be run as part of the explicit control evaluator machine
(check out the book!). You can look at the output of the compiler by
using the {\tt compile} procedure from Scheme. For example:

\beginlisp
1 ==> (pp (compile '(* x y)))
((ASSIGN FUN (LOOKUP-VARIABLE-VALUE '* (FETCH ENV)))
 (ASSIGN VAL (LOOKUP-VARIABLE-VALUE 'X (FETCH ENV)))
 (ASSIGN ARGL (CONS (FETCH VAL) '()))
 (ASSIGN VAL (LOOKUP-VARIABLE-VALUE 'Y (FETCH ENV)))
 (ASSIGN ARGL (CONS (FETCH VAL) (FETCH ARGL)))
 (GOTO APPLY-DISPATCH))
\endlisp

To actually run the code, use the procedure {\tt compile-and-go} which
compiles an expression and runs the result via the explicit control
evaluator:

\beginlisp
1 ==> (compile-and-go '(* 5 (+ 4 6)))
(TOTAL-PUSHES: 0 MAXIMUM-DEPTH: 0) 
50 
(TOTAL-PUSHES: 4 MAXIMUM-DEPTH: 4) 
(MACHINE OPERATIONS: 37) 
EC-EVAL==>
Quit!
\endlisp

You can compile definitions with {\tt compile-and-go} which will run the
define and make the compiled procedure available from the evaluator.
(Since {\tt compile} and {\tt compile-and-go} are top-level to Scheme, you can
{\tt zap} explicit calls to them from NMODE instead of typing them by hand!).

\beginlisp
1 ==> (compile-and-go '(define (f x) (+ x 5)))
(TOTAL-PUSHES: 0 MAXIMUM-DEPTH: 0) 
F 
(TOTAL-PUSHES: 1 MAXIMUM-DEPTH: 1) 
(MACHINE OPERATIONS: 15) 
EC-EVAL==>(f 10)
15 
(TOTAL-PUSHES: 5 MAXIMUM-DEPTH: 3) 
(MACHINE OPERATIONS: 66) 
EC-EVAL==>
Quit!
\endlisp

By the way, becuase of the design of the tracing feature in Problem 4, and
because compiled code is executed from the evaluator, you should be
able to trace compiled procedures as well as procedures called from compiled
procedures!  However, in this problem you will again be gathering some 
performance statistics, and although your changes to the evaluator in
problem 4 should not significantly affect these statistics, reload the
original verion of the evaluator before doing problem 5.  To do this, type
{\tt (LOAD "\#5:/SCHEME/PSETS/PS9-ECEVAL.SCM") }

\section{Problem 5 -- Compiled runs of count-atoms}

Run the compiled version of the recursive {\tt count-atoms} procedure 
(using {\tt compile-and-go}).  There is a definition in {\tt
PS9-ANSWER.SCM} that gives the code for a recursive procedure
called {\tt count-atoms-comp}.  After zapping this definition, you
can compile the code by typing {\tt (compile-and-go count-atoms-code)},
and then run the procedure {\tt count-atoms-comp}.

Fill out the table in {\tt PS9-ANSWER.SCM} and derive formulas as before.  
Write a paragraph explaining a few of the reasons why the compiled run of 
{\tt count-atoms-comp} is more efficient than the interpreted run of the same 
procedure (i.e., {\tt count-atoms-r}).

\section{Problem 6 -- Compiled code for count-atoms}

Make a listing of the compiled code for {\tt count-atoms} by making a
{\tt photo} of the evaluation of: 

\beginlisp
(pp (compile count-atoms-code))
\endlisp

Annotate your transcript like the version of {\tt factorial} on
pages 478-479 in your book. In particular, for each jump to
{\tt apply-dispatch} you should indicate which procedure is about to be
applied. Include your annotated printout in the problem set.

\end